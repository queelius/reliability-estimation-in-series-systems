% Options for packages loaded elsewhere
\PassOptionsToPackage{unicode}{hyperref}
\PassOptionsToPackage{hyphens}{url}
%
\documentclass[
]{article}
\usepackage{lmodern}
\usepackage{amssymb,amsmath}
\usepackage{ifxetex,ifluatex}
\ifnum 0\ifxetex 1\fi\ifluatex 1\fi=0 % if pdftex
  \usepackage[T1]{fontenc}
  \usepackage[utf8]{inputenc}
  \usepackage{textcomp} % provide euro and other symbols
\else % if luatex or xetex
  \usepackage{unicode-math}
  \defaultfontfeatures{Scale=MatchLowercase}
  \defaultfontfeatures[\rmfamily]{Ligatures=TeX,Scale=1}
\fi
% Use upquote if available, for straight quotes in verbatim environments
\IfFileExists{upquote.sty}{\usepackage{upquote}}{}
\IfFileExists{microtype.sty}{% use microtype if available
  \usepackage[]{microtype}
  \UseMicrotypeSet[protrusion]{basicmath} % disable protrusion for tt fonts
}{}
\usepackage{xcolor}
\IfFileExists{xurl.sty}{\usepackage{xurl}}{} % add URL line breaks if available
\IfFileExists{bookmark.sty}{\usepackage{bookmark}}{\usepackage{hyperref}}
\hypersetup{
  pdftitle={Bootstrapping confidence intervals (BCa) of the maximum likelihood estimator of components in a series systems from masked failure data},
  pdfauthor={Alex Towell},
  hidelinks,
  pdfcreator={LaTeX via pandoc}}
\urlstyle{same} % disable monospaced font for URLs
\usepackage[margin=1in]{geometry}
\usepackage{color}
\usepackage{fancyvrb}
\newcommand{\VerbBar}{|}
\newcommand{\VERB}{\Verb[commandchars=\\\{\}]}
\DefineVerbatimEnvironment{Highlighting}{Verbatim}{commandchars=\\\{\}}
% Add ',fontsize=\small' for more characters per line
\usepackage{framed}
\definecolor{shadecolor}{RGB}{248,248,248}
\newenvironment{Shaded}{\begin{snugshade}}{\end{snugshade}}
\newcommand{\AlertTok}[1]{\textcolor[rgb]{0.94,0.16,0.16}{#1}}
\newcommand{\AnnotationTok}[1]{\textcolor[rgb]{0.56,0.35,0.01}{\textbf{\textit{#1}}}}
\newcommand{\AttributeTok}[1]{\textcolor[rgb]{0.77,0.63,0.00}{#1}}
\newcommand{\BaseNTok}[1]{\textcolor[rgb]{0.00,0.00,0.81}{#1}}
\newcommand{\BuiltInTok}[1]{#1}
\newcommand{\CharTok}[1]{\textcolor[rgb]{0.31,0.60,0.02}{#1}}
\newcommand{\CommentTok}[1]{\textcolor[rgb]{0.56,0.35,0.01}{\textit{#1}}}
\newcommand{\CommentVarTok}[1]{\textcolor[rgb]{0.56,0.35,0.01}{\textbf{\textit{#1}}}}
\newcommand{\ConstantTok}[1]{\textcolor[rgb]{0.00,0.00,0.00}{#1}}
\newcommand{\ControlFlowTok}[1]{\textcolor[rgb]{0.13,0.29,0.53}{\textbf{#1}}}
\newcommand{\DataTypeTok}[1]{\textcolor[rgb]{0.13,0.29,0.53}{#1}}
\newcommand{\DecValTok}[1]{\textcolor[rgb]{0.00,0.00,0.81}{#1}}
\newcommand{\DocumentationTok}[1]{\textcolor[rgb]{0.56,0.35,0.01}{\textbf{\textit{#1}}}}
\newcommand{\ErrorTok}[1]{\textcolor[rgb]{0.64,0.00,0.00}{\textbf{#1}}}
\newcommand{\ExtensionTok}[1]{#1}
\newcommand{\FloatTok}[1]{\textcolor[rgb]{0.00,0.00,0.81}{#1}}
\newcommand{\FunctionTok}[1]{\textcolor[rgb]{0.00,0.00,0.00}{#1}}
\newcommand{\ImportTok}[1]{#1}
\newcommand{\InformationTok}[1]{\textcolor[rgb]{0.56,0.35,0.01}{\textbf{\textit{#1}}}}
\newcommand{\KeywordTok}[1]{\textcolor[rgb]{0.13,0.29,0.53}{\textbf{#1}}}
\newcommand{\NormalTok}[1]{#1}
\newcommand{\OperatorTok}[1]{\textcolor[rgb]{0.81,0.36,0.00}{\textbf{#1}}}
\newcommand{\OtherTok}[1]{\textcolor[rgb]{0.56,0.35,0.01}{#1}}
\newcommand{\PreprocessorTok}[1]{\textcolor[rgb]{0.56,0.35,0.01}{\textit{#1}}}
\newcommand{\RegionMarkerTok}[1]{#1}
\newcommand{\SpecialCharTok}[1]{\textcolor[rgb]{0.00,0.00,0.00}{#1}}
\newcommand{\SpecialStringTok}[1]{\textcolor[rgb]{0.31,0.60,0.02}{#1}}
\newcommand{\StringTok}[1]{\textcolor[rgb]{0.31,0.60,0.02}{#1}}
\newcommand{\VariableTok}[1]{\textcolor[rgb]{0.00,0.00,0.00}{#1}}
\newcommand{\VerbatimStringTok}[1]{\textcolor[rgb]{0.31,0.60,0.02}{#1}}
\newcommand{\WarningTok}[1]{\textcolor[rgb]{0.56,0.35,0.01}{\textbf{\textit{#1}}}}
\usepackage{longtable,booktabs}
% Correct order of tables after \paragraph or \subparagraph
\usepackage{etoolbox}
\makeatletter
\patchcmd\longtable{\par}{\if@noskipsec\mbox{}\fi\par}{}{}
\makeatother
% Allow footnotes in longtable head/foot
\IfFileExists{footnotehyper.sty}{\usepackage{footnotehyper}}{\usepackage{footnote}}
\makesavenoteenv{longtable}
\usepackage{graphicx}
\makeatletter
\def\maxwidth{\ifdim\Gin@nat@width>\linewidth\linewidth\else\Gin@nat@width\fi}
\def\maxheight{\ifdim\Gin@nat@height>\textheight\textheight\else\Gin@nat@height\fi}
\makeatother
% Scale images if necessary, so that they will not overflow the page
% margins by default, and it is still possible to overwrite the defaults
% using explicit options in \includegraphics[width, height, ...]{}
\setkeys{Gin}{width=\maxwidth,height=\maxheight,keepaspectratio}
% Set default figure placement to htbp
\makeatletter
\def\fps@figure{htbp}
\makeatother
\setlength{\emergencystretch}{3em} % prevent overfull lines
\providecommand{\tightlist}{%
  \setlength{\itemsep}{0pt}\setlength{\parskip}{0pt}}
\setcounter{secnumdepth}{5}
\hypersetup{linktoc=all}
\AtBeginDocument{\renewcommand{\refname}{References}}
\usepackage{hyperref}
\usepackage{graphicx}
\usepackage{amsthm}
\usepackage{amsmath}
\usepackage{natbib}
\usepackage{tikz}
\usepackage[]{natbib}
\bibliographystyle{apalike}

\title{Bootstrapping confidence intervals (BCa) of the maximum
likelihood estimator of components in a series systems from masked
failure data}
\author{Alex Towell}
\date{}

\begin{document}
\maketitle
\begin{abstract}
We estimate the parameters of a series system with Weibull component
lifetimes from relatively small samples consisting of right-censored
system lifetimes and masked component cause of failure. Under a set of
conditions that permit us to ignore how the component cause of failures
are masked, we assess the bias and variance of the estimator. Then, we
assess the accuracy of the boostrapped variance and calibration of the
confidence intervals of the MLE under a variety of scenarios.
\end{abstract}

{
\setcounter{tocdepth}{2}
\tableofcontents
}
\newcommand{\T}{T}
\newtheorem{definition}{Definition}
\newtheorem{theorem}{Theorem}
\newtheorem{corollary}{Corollary}
\newtheorem{condition}{Condition}
\renewcommand{\v}[1]{\boldsymbol{#1}}
\numberwithin{equation}{section}

\hypertarget{introduction}{%
\section{Introduction}\label{introduction}}

Accurately estimating the reliability of individual components in
multi-component systems is an important problem in many engineering
domains. However, component lifetimes and failure causes are often not
directly observable. In a series system, only the system-level failure
time may be recorded along with limited information about which
component failed. Such \emph{masked} data poses challenges for
estimating component reliability.

In this paper, we develop a maximum likelihood approach to estimate
component reliability in series systems using right-censored lifetime
data and candidate sets that contain the failed component. The key
contributions are:

\begin{enumerate}
\def\labelenumi{\arabic{enumi}.}
\item
  Deriving a likelihood model that accounts for right-censoring and
  masked failure causes through candidate sets. This allows the
  available masked data to be used for estimation.
\item
  Validating the accuracy, precision, and robustness of the maximum
  likelihood estimator through an extensive simulation study under
  different sample sizes, masking probabilities, and censoring levels.
\item
  Demonstrating that bootstrapping provides well-calibrated confidence
  intervals for the MLEs even with small samples.
\end{enumerate}

Together, these contributions provide a statistically rigorous
methodology for learning about latent component properties from series
system data. The methods are shown to work well even when failure
information is significantly masked. This capability expands the range
of applications where component reliability can be quantified from
limited observations.

The remainder of this paper is organized as follows. First, we detail
the series system and masked data models. Next, we present the
likelihood construction and maximum likelihood theory. We then describe
the bootstrap approach for variance and confidence interval estimation.
Finally, we validate the methods through simulation studies under
various data scenarios and sample sizes.

\hypertarget{sec:statmod}{%
\section{Series System Model}\label{sec:statmod}}

Consider a system composed of \(m\) components arranged in a series
configuration. Each component and system has two possible states,
functioning or failed. We have \(n\) systems whose lifetimes are
independent and identically distributed (i.i.d.). The lifetime of the
\(i\)\textsuperscript{th} system denoted by the random variable
\(T_{i}\). The lifetime of the \(j\)\textsuperscript{th} component in
the \(i\)\textsuperscript{th} system is denoted by the random variable
\(T_{i j}\). We assume the component lifetimes in a single system are
statistically independent and non-identically distributed. Here,
lifetime is defined as the elapsed time from when the new, functioning
component (or system) is put into operation until it fails for the first
time. A series system fails when any component fails, thus the lifetime
of the \(i\)\textsuperscript{th} system is given by the component with
the shortest lifetime, \[
    T_i = \min\bigr\{T_{i 1},T_{i 2}, \ldots, T_{i m} \bigr\}.
\]

There are three particularly important distribution functions in
survival analysis: the survival function, the probability density
function, and the hazard function. The survival function,
\(R_{T_i}(t)\), is the probability that the \(i\)\textsuperscript{th}
system has a lifespan larger than a duration \(t\), \begin{equation}
R_{T_i}(t) = \Pr\{T_i > t\}\\
\end{equation} The probability density function (pdf) of \(T_i\) is
denoted by \(f_{T_i}(t)\) and may be defined as \[
    f_{T_i}(t) = -\frac{d}{dt} R_{T_i}(t).
\] Next, we introduce the hazard function. The probability that a
failure occurs between \(t\) and \(\Delta t\) given that no failure
occurs before time \(t\) is given by \[
\Pr\{T_i \leq t+\Delta t|T_i > t\} = \frac{\Pr\{t < T_i < t+\Delta t\}}{\Pr\{T_i > t\}}.
\] The failure rate is given by the dividing this equation by the length
of the time interval, \(\Delta t\): \[
\frac{\Pr\{t < T < t+\Delta t\}}{\Delta t} \frac{1}{\Pr\{T > t\}} =
    \frac{R_T(t) - R(t+\Delta t)}{R_T(t)}.
\] The hazard function \(h_{T_i}(t)\) for \(T_i\) is the instantaneous
failure rate at time \(t\), which is given by \begin{equation}
\label{eq:failure_rate}
\begin{split}
h_{T_i}(t) &= \lim_{\Delta t \to 0} \frac{\Pr\{t < T_i < t+\Delta t\}}{\Delta t} \frac{1}{\Pr\{T_i > t\}}\\
       &= \frac{f_{T_i}(t)}{R_{T_i}(t)}.
\end{split}
\end{equation} \textbackslash end\{definition\}

The lifetime of the \(j\)\textsuperscript{th} component is assumed to
follow a parametric distribution indexed by a parameter vector
\(\boldsymbol{\theta_j}\). The parameter vector of the overall system is
defined as \[
    \boldsymbol{\theta }= (\boldsymbol{\theta_1},\ldots,\boldsymbol{\theta_m}).
\]

When a random variable \(T\) is parameterized by a particular
\(\boldsymbol{\theta}\), we denote the reliability function by
\(R_T(t;\boldsymbol{\theta})\), and the same for other distribution
functions. If it is clear from the context which random variable a
distribution function is for, we drop the subscripts, e.g., \(R(t)\)
instead of \(R_T(t)\). As a special case, we denote the pdf of the
\(j\)\textsuperscript{th} component by \(f_j(t;\boldsymbol{\theta_j})\)
and its reliability function by \(R_j(t;\boldsymbol{\theta_j})\).

Two random variables \(X\) and \(Y\) have a joint pdf \(f_{X,Y}(x,y)\).
Given the joint pdf \(f(x,y)\), the marginal pdf of \(X\) is given by \[
f_X(x) = \int_{\mathcal{Y}} f_{X,Y}(x,y) dy,
\] where \(\mathcal{Y}\) is the support of \(Y\). (If \(Y\) is discrete,
replace the integration with a summation over \(\mathcal{Y}\).)

The conditional pdf of \(Y\) given \(X=x\), \(f_{Y|X}(y|x)\), is defined
as \[
f_{X|Y}(y|x) = \frac{f_{X,Y}(x,y)}{f_X(x)}.
\] We may generalize all of the above to more than two random variables,
e.g., the joint pdf of \(X_1,\ldots,X_m\) is denoted by
\(f(x_1,\ldots,x_m)\).

Next, we dive deeper into these concepts and provide mathematical
derivations for the reliability function, pdf, and hazard function of
the series system. We begin with the reliability function of the series
system, as given by the following theorem.

\begin{theorem}
\label{thm:sys_reliability_function}
The series system has a reliability function given by
\begin{equation}
\label{eq:sys_reliability_function}
  R(t;\boldsymbol{\theta}) = \prod_{j=1}^m R_j(t;\boldsymbol{\theta_j}).
\end{equation}
\end{theorem}
\begin{proof}
The reliability function is defined as
$$
  R(t;\boldsymbol{\theta}) = \Pr\{T_i > t\}
$$
which may be rewritten as
$$
  R(t;\boldsymbol{\theta}) = \Pr\{\min\{T_{i 1},\ldots,T_{i m}\} > t\}.
$$
For the minimum to be larger than $t$, every component must be larger than $t$,
$$
  R(t;\boldsymbol{\theta}) = \Pr\{T_{i 1} > t,\ldots,T_{i m} > t\}.
$$
Since the component lifetimes are independent, by the product rule the above may
be rewritten as
$$
  R(t;\boldsymbol{\theta}) = \Pr\{T_{i 1} > t\} \times \cdots \times \Pr\{T_{i m} > t\}.
$$
By definition, $R_j(t;\boldsymbol{\theta}) = \Pr\{T_{i j} > t\}$.
Performing this substitution obtains the result
$$
  R(t;\boldsymbol{\theta}) = \prod_{j=1}^m R_j(t;\boldsymbol{\theta_j}).
$$
\end{proof}

Theorem \ref{thm:sys_reliability_function} shows that the system's
overall reliability is the product of the reliabilities of its
individual components. This property is inherent to series systems and
will be used in the subsequent derivations.

Next, we turn our attention to the pdf of the system lifetime, described
in the following theorem.

\begin{theorem}
\label{thm:sys_pdf}
The series system has a pdf given by
\begin{equation}
\label{eq:sys_pdf}
f(t;\boldsymbol{\theta}) = \sum_{j=1}^m f_j(t;\boldsymbol{\theta_j})
    \prod_{\substack{k=1\\k\neq j}}^m R_k(t;\boldsymbol{\theta_j}).
\end{equation}
\end{theorem}
\begin{proof}
By definition, the pdf may be written as
$$
    f(t;\boldsymbol{\theta}) = -\frac{d}{dt} \prod_{j=1}^m R_j(t;\boldsymbol{\theta_j}).
$$
By the product rule, this may be rewritten as
\begin{align*}
  f(t;\boldsymbol{\theta})
    &= -\frac{d}{dt} R_1(t;\boldsymbol{\theta_1})\prod_{j=2}^m R_j(t;\boldsymbol{\theta_j}) -
      R_1(t;\boldsymbol{\theta_1}) \frac{d}{dt} \prod_{j=2}^m R_j(t;\boldsymbol{\theta_j})\\
    &= f_1(t;\boldsymbol{\theta}) \prod_{j=2}^m R_j(t;\boldsymbol{\theta_j}) -
      R_1(t;\boldsymbol{\theta_1}) \frac{d}{dt} \prod_{j=2}^m R_j(t;\boldsymbol{\theta_j}).
\end{align*}
Recursively applying the product rule $m-1$ times results in
$$
f(t;\boldsymbol{\theta}) = \sum_{j=1}^{m-1} f_j(t;\boldsymbol{\theta_j})
    \prod_{\substack{k=1\\k \neq j}}^m R_k(t;\boldsymbol{\theta_k}) -
    \prod_{j=1}^{m-1} R_j(t;\boldsymbol{\theta_j}) \frac{d}{dt} R_m(t;\boldsymbol{\theta_m}),
$$
which simplifies to
$$
f(t;\boldsymbol{\theta})= \sum_{j=1}^m f_j(t;\boldsymbol{\theta_j})
    \prod_{\substack{k=1\\k \neq j}}^m R_k(t;\boldsymbol{\theta_k}).
$$
\end{proof}

Theorem \ref{thm:sys_pdf} shows the pdf of the system lifetime as a
function of the pdfs and reliabilities of its components.

We continue with the hazard function of the system lifetime, defined in
the next theorem.

\begin{theorem}
\label{thm:sys_failure_rate}
The series system has a hazard function given by
\begin{equation}
\label{eq:sys_failure_rate}
  h(t;\boldsymbol{\theta}) = \sum_{j=1}^m h_j(t;\boldsymbol{\theta_j}).
\end{equation}
\end{theorem}
\begin{proof}
By Equation \eqref{eq:failure_rate}, the $i$\textsuperscript{th} series system lifetime has a hazard function defined as
$$
  h(t;\boldsymbol{\theta}) = \frac{f_{T_i}(t;\boldsymbol{\theta})}{R_{T_i}(t;\boldsymbol{\theta})}.
$$
Plugging in expressions for these functions results in
$$
  h(t;\boldsymbol{\theta}) = \frac{\sum_{j=1}^m f_j(t;\boldsymbol{\theta_j})
    \prod_{\substack{k=1\\k \neq j}}^m R_k(t;\boldsymbol{\theta_k})}
      {\prod_{j=1}^m R_j(t;\boldsymbol{\theta_j})},
$$
which can be simplified to
$$
h_{T_i}(t;\boldsymbol{\theta}) = \sum_{j=1}^m \frac{f_j(t;\boldsymbol{\theta_j})}{R_j(t;\boldsymbol{\theta_j})} = \sum_{j=1}^m h_j(t;\boldsymbol{\theta_j}).
$$
\end{proof}

Theorem \ref{thm:sys_failure_rate} reveals that the system's hazard
function is the sum of the hazard functions of its components.

By definition, the hazard function is the ratio of the pdf to the
reliability function, \[
h(t;\boldsymbol{\theta}) = \frac{f(t;\boldsymbol{\theta})}{R(t;\boldsymbol{\theta})},
\] and we can rearrange this to get \begin{equation}
\label{eq:sys_pdf_2}
\begin{split}
f(t;\boldsymbol{\theta}) &= h(t;\boldsymbol{\theta}) R(t;\boldsymbol{\theta})\\
              &= \biggl\{\sum_{j=1}^m h_j(t;\boldsymbol{\theta_j})\biggr\}
                 \biggl\{ \prod_{j=1}^m R_j(t;\boldsymbol{\theta_j}) \biggr\},
\end{split}
\end{equation} which we sometimes find to be a more convenient form than
Equation \eqref{eq:sys_pdf}.

In this section, we derived the mathematical forms for the system's
reliability function, pdf, and hazard function. Next, we build upon
these concepts to derive distributions related to the component cause of
failure.

\hypertarget{sec:comp_cause}{%
\subsection{Component Cause of Failure}\label{sec:comp_cause}}

Whenever a series system fails, precisely one of the components is the
cause. We model the component cause of the series system failure as a
random variable.

\begin{definition}
The component cause of failure of a series system is
denoted by the random variable $K_i$ whose support is given by $\{1,\ldots,m\}$.
For example, $K_i=j$ indicates that the component indexed by $j$ failed first, i.e.,
$$
    T_{i j} < T_{i j'}
$$
for every $j'$ in the support of $K_i$ except for $j$.
Since we have series systems, $K_i$ is unique.
\end{definition}

The system lifetime and the component cause of failure has a joint
distribution given by the following theorem.

\begin{theorem}
\label{thm:f_k_and_t}
The joint pdf of the component cause of failure $K_i$ and series system lifetime
$T_i$ is given by
\begin{equation}
\label{eq:f_k_and_t}
  f_{K_i,T_i}(j,t;\boldsymbol{\theta}) = h_j(t;\boldsymbol{\theta_j}) R_{T_i}(t;\boldsymbol{\theta}),
\end{equation}
where $h_j(t;\boldsymbol{\theta_j})$ is the hazard function of the $j$\textsuperscript{th}
component and $R_{T_i}(t;\boldsymbol{\theta})$ is the reliability function of the series
system.
\end{theorem}
\begin{proof}
Consider a series system with $3$ components.
By the assumption that component lifetimes are mutually independent,
the joint pdf of $T_{i 1},T_{i 2},T_{i 3}$ is given by
$$
    f(t_1,t_2,t_3;\boldsymbol{\theta}) = \prod_{j=1}^{3} f_j(t;\boldsymbol{\theta_j}).
$$
The first component is the cause of failure at time $t$ if $K_i = 1$ and
$T_i = t$, which may be rephrased as the likelihood that $T_{i 1} = t$,
$T_{i 2} > t$, and $T_{i 3} > t$. Thus,
\begin{align*}
f_{K_i,T_i}(j;\boldsymbol{\theta}) 
    &= \int_t^{\infty} \int_t^{\infty}
        f_1(t;\boldsymbol{\theta_1}) f_2(t_2;\boldsymbol{\theta_2}) f_3(t_3;\boldsymbol{\theta_3})
        dt_3 dt_2\\
     &= \int_t^{\infty} f_1(t;\boldsymbol{\theta_1}) f_2(t_2;\boldsymbol{\theta_2})
        R_3(t;\boldsymbol{\theta_3}) dt_2\\
     &= f_1(t;\boldsymbol{\theta_1}) R_2(t;\boldsymbol{\theta_2}) R_3(t_1;\boldsymbol{\theta_3}).
\end{align*}
Since $h_1(t;\boldsymbol{\theta_1}) = f_1(t;\boldsymbol{\theta_1}) / R_1(t;\boldsymbol{\theta_1})$,
$$
f_1(t;\boldsymbol{\theta_1}) = h_1(t;\boldsymbol{\theta_1}) R_1(t;\boldsymbol{\theta_1}).
$$
Making this substitution into the above expression for $f_{K_i,T_i}(j,t;\boldsymbol{\theta})$
yields
\begin{align*}
f_{K_i,T_i}(j,t;\boldsymbol{\theta})
    &= h_1(t;\boldsymbol{\theta_1}) \prod_{l=1}^m R_l(t;\boldsymbol{\theta_l})\\
    &= h_1(t;\boldsymbol{\theta_1}) R(t;\boldsymbol{\theta}).
\end{align*}
Generalizing from this completes the proof.
\end{proof}

\hypertarget{sec:like_model}{%
\section{Likelihood Model for Masked Data}\label{sec:like_model}}

The object of interest is the (unknown) parameter value
\(\boldsymbol{\theta}\). To estimate this \(\boldsymbol{\theta}\), we
need \emph{data}. In our case, we call it \emph{masked data} because we
do not necessarily observe the event of interest, say a system failure,
directly. We consider two types of masking: masking the system failure
lifetime and masking the component cause of failure.

We generally encounter three types of system failure lifetime masking:

\begin{enumerate}
\def\labelenumi{\arabic{enumi}.}
\tightlist
\item
  A system failure is observed at a particular point in time.
\item
  A system failure is observed to occur within a particular interval of
  time.
\item
  A system failure is not observed, but we know that the system survived
  at least until a particular point in time. This is known as
  \emph{right-censoring} and can occur if, for instance, an experiment
  is terminated while the system is still functioning.
\end{enumerate}

We generally encounter two types of component cause of failure masking:

\begin{enumerate}
\def\labelenumi{\arabic{enumi}.}
\tightlist
\item
  The component cause of failure is observed.
\item
  The component cause of failure is not observed, but we know that the
  failed component is in some set of components. This is known as
  \emph{masking} the component cause of failure.
\end{enumerate}

Thus, the component cause of failure masking will take the form of
candidate sets. A candidate set consists of some subset of component
labels that plausibly contains the label of the failed component. The
sample space of candidate sets are all subsets of \(\{1,\ldots,m\}\),
thus there are \(2^m\) possible outcomes in the sample space.

In this paper, we limit our focus to observing \emph{right censored}
lifetimes and exact lifetimes but with masked component cause of
failures. We consider a sample of \(n\) i.i.d. series systems, each of
which is put into operation at some time and and observed until either
it fails or is right-censored. We denote the right-censoring time of the
\(i\)\textsuperscript{th} system by \(\tau_i\). We do not directly
observe the system lifetime, \(T_i\), but rather, we observe the
right-censored lifetime, \(S_i\), which is given by \begin{equation}
    S_i = \min\{\tau_i, T_i\},
\end{equation} We also observe a right-censoring indicator,
\(\delta_i\), which is given by \begin{equation}
    \delta_i = 1_{T_i < \tau_i}
\end{equation} where \(1_{\text{condition}}\) is an indicator function
that outputs \(1\) if \emph{condition} is true and \(0\) otherwise.
Here, \(\delta_i = 1\) indicates the event of interest, a system
failure, was observed.

If a system failure lifetime is observed, then we also observe a
candidate set that contains the component cause of failure. We denote
the candidate set for the \(i\)\textsuperscript{th} system by
\(\mathcal{C}_i\), which is a subset of \(\{1,\ldots,m\}\). Since the
data generating process for candidate sets may be subject to chance
variations, it as a random set.

Consider we have an independent and identically distributed (i.i.d.)
random sample of masked data, \(D = \{D_1, \ldots, D_n\}\), where each
\(D_i\) contanis the following:

\begin{itemize}
\tightlist
\item
  \(S_i\), the system lifetime of the \(i\)\textsuperscript{th} system.
\item
  \(\delta_i\), the right-censoring indicator of the
  \(i\)\textsuperscript{th} system.
\item
  \(\mathcal{C}_i\), the set of candidate component causes of failure
  for the \(i\)\textsuperscript{th} system.
\end{itemize}

The masked data generation process is illustrated by Figure
\ref{fig:figureone}.

\begin{figure}[h]
\centering{
\resizebox{0.5\textwidth}{!}{% Graphic for TeX using PGF
% Title: /home/spinoza/gdrive/sources/math_project/image/dep_model.dia
% Creator: Dia v0.97+git
% CreationDate: Wed Jun  8 23:19:46 2022
% For: spinoza
% \usepackage{tikz}
% The following commands are not supported in PSTricks at present
% We define them conditionally, so when they are implemented,
% this pgf file will use them.
\ifx\du\undefined
  \newlength{\du}
\fi
\setlength{\du}{15\unitlength}
\begin{tikzpicture}[even odd rule]
\pgftransformxscale{1.000000}
\pgftransformyscale{-1.000000}
\definecolor{dialinecolor}{rgb}{0.000000, 0.000000, 0.000000}
\pgfsetstrokecolor{dialinecolor}
\pgfsetstrokeopacity{1.000000}
\definecolor{diafillcolor}{rgb}{1.000000, 1.000000, 1.000000}
\pgfsetfillcolor{diafillcolor}
\pgfsetfillopacity{1.000000}
\pgfsetlinewidth{0.100000\du}
\pgfsetdash{}{0pt}
\pgfsetbuttcap
\pgfsetmiterjoin
\pgfsetlinewidth{0.100000\du}
\pgfsetbuttcap
\pgfsetmiterjoin
\pgfsetdash{}{0pt}
\definecolor{dialinecolor}{rgb}{1.000000, 0.000000, 0.000000}
\pgfsetstrokecolor{dialinecolor}
\pgfsetstrokeopacity{1.000000}
\pgfpathellipse{\pgfpoint{49.017439\du}{4.479179\du}}{\pgfpoint{0.901439\du}{0\du}}{\pgfpoint{0\du}{0.901439\du}}
\pgfusepath{stroke}
\pgfsetlinewidth{0.100000\du}
\pgfsetdash{}{0pt}
\pgfsetbuttcap
\pgfsetmiterjoin
\pgfsetlinewidth{0.100000\du}
\pgfsetbuttcap
\pgfsetmiterjoin
\pgfsetdash{}{0pt}
\definecolor{dialinecolor}{rgb}{0.803922, 0.000000, 0.000000}
\pgfsetstrokecolor{dialinecolor}
\pgfsetstrokeopacity{1.000000}
\pgfpathellipse{\pgfpoint{54.950000\du}{10.000000\du}}{\pgfpoint{1.000000\du}{0\du}}{\pgfpoint{0\du}{1.000000\du}}
\pgfusepath{stroke}
\pgfsetlinewidth{0.100000\du}
\pgfsetdash{{\pgflinewidth}{0.200000\du}}{0cm}
\pgfsetmiterjoin
\pgfsetbuttcap
\definecolor{dialinecolor}{rgb}{1.000000, 0.000000, 0.000000}
\pgfsetstrokecolor{dialinecolor}
\pgfsetstrokeopacity{1.000000}
\pgfpathmoveto{\pgfpoint{41.000000\du}{10.000000\du}}
\pgfpathcurveto{\pgfpoint{41.000000\du}{10.000000\du}}{\pgfpoint{41.000000\du}{8.000000\du}}{\pgfpoint{44.000000\du}{8.000000\du}}
\pgfpathcurveto{\pgfpoint{47.000000\du}{8.000000\du}}{\pgfpoint{48.000000\du}{8.000000\du}}{\pgfpoint{49.000000\du}{8.000000\du}}
\pgfpathcurveto{\pgfpoint{50.000000\du}{8.000000\du}}{\pgfpoint{51.000000\du}{8.000000\du}}{\pgfpoint{54.000000\du}{8.000000\du}}
\pgfpathcurveto{\pgfpoint{57.000000\du}{8.000000\du}}{\pgfpoint{57.000000\du}{10.000000\du}}{\pgfpoint{57.000000\du}{10.000000\du}}
\pgfpathcurveto{\pgfpoint{57.000000\du}{10.000000\du}}{\pgfpoint{57.000000\du}{12.000000\du}}{\pgfpoint{54.000000\du}{12.000000\du}}
\pgfpathcurveto{\pgfpoint{51.000000\du}{12.000000\du}}{\pgfpoint{50.000000\du}{12.000000\du}}{\pgfpoint{49.000000\du}{12.000000\du}}
\pgfpathcurveto{\pgfpoint{48.000000\du}{12.000000\du}}{\pgfpoint{47.000000\du}{12.000000\du}}{\pgfpoint{44.000000\du}{12.000000\du}}
\pgfpathcurveto{\pgfpoint{41.000000\du}{12.000000\du}}{\pgfpoint{41.000000\du}{10.000000\du}}{\pgfpoint{41.000000\du}{10.000000\du}}
\pgfpathclose
\pgfusepath{stroke}
% setfont left to latex
\definecolor{dialinecolor}{rgb}{0.000000, 0.000000, 0.000000}
\pgfsetstrokecolor{dialinecolor}
\pgfsetstrokeopacity{1.000000}
\definecolor{diafillcolor}{rgb}{0.000000, 0.000000, 0.000000}
\pgfsetfillcolor{diafillcolor}
\pgfsetfillopacity{1.000000}
\node[anchor=base west,inner sep=0pt,outer sep=0pt,color=dialinecolor] at (43.300000\du,10.000000\du){};
% setfont left to latex
\definecolor{dialinecolor}{rgb}{0.000000, 0.000000, 0.000000}
\pgfsetstrokecolor{dialinecolor}
\pgfsetstrokeopacity{1.000000}
\definecolor{diafillcolor}{rgb}{0.000000, 0.000000, 0.000000}
\pgfsetfillcolor{diafillcolor}
\pgfsetfillopacity{1.000000}
\node[anchor=base west,inner sep=0pt,outer sep=0pt,color=dialinecolor] at (43.300000\du,10.000000\du){};
\pgfsetlinewidth{0.100000\du}
\pgfsetdash{}{0pt}
\pgfsetbuttcap
\pgfsetmiterjoin
\pgfsetlinewidth{0.100000\du}
\pgfsetbuttcap
\pgfsetmiterjoin
\pgfsetdash{}{0pt}
\definecolor{dialinecolor}{rgb}{1.000000, 0.000000, 0.000000}
\pgfsetstrokecolor{dialinecolor}
\pgfsetstrokeopacity{1.000000}
\pgfpathellipse{\pgfpoint{45.544240\du}{4.524260\du}}{\pgfpoint{0.897140\du}{0\du}}{\pgfpoint{0\du}{0.897140\du}}
\pgfusepath{stroke}
% setfont left to latex
\definecolor{dialinecolor}{rgb}{0.000000, 0.000000, 0.000000}
\pgfsetstrokecolor{dialinecolor}
\pgfsetstrokeopacity{1.000000}
\definecolor{diafillcolor}{rgb}{0.000000, 0.000000, 0.000000}
\pgfsetfillcolor{diafillcolor}
\pgfsetfillopacity{1.000000}
\node[anchor=base,inner sep=0pt, outer sep=0pt,color=dialinecolor] at (45.544200\du,4.817368\du){$T_i$};
\pgfsetlinewidth{0.100000\du}
\pgfsetdash{}{0pt}
\pgfsetbuttcap
\pgfsetmiterjoin
\pgfsetlinewidth{0.100000\du}
\pgfsetbuttcap
\pgfsetmiterjoin
\pgfsetdash{}{0pt}
\definecolor{dialinecolor}{rgb}{0.000000, 0.639216, 0.000000}
\pgfsetstrokecolor{dialinecolor}
\pgfsetstrokeopacity{1.000000}
\pgfpathellipse{\pgfpoint{52.559874\du}{4.530414\du}}{\pgfpoint{0.903074\du}{0\du}}{\pgfpoint{0\du}{0.903074\du}}
\pgfusepath{stroke}
% setfont left to latex
\definecolor{dialinecolor}{rgb}{0.000000, 0.000000, 0.000000}
\pgfsetstrokecolor{dialinecolor}
\pgfsetstrokeopacity{1.000000}
\definecolor{diafillcolor}{rgb}{0.000000, 0.000000, 0.000000}
\pgfsetfillcolor{diafillcolor}
\pgfsetfillopacity{1.000000}
\node[anchor=base west,inner sep=0pt,outer sep=0pt,color=dialinecolor] at (49.925000\du,4.725000\du){};
% setfont left to latex
\definecolor{dialinecolor}{rgb}{0.000000, 0.000000, 0.000000}
\pgfsetstrokecolor{dialinecolor}
\pgfsetstrokeopacity{1.000000}
\definecolor{diafillcolor}{rgb}{0.000000, 0.000000, 0.000000}
\pgfsetfillcolor{diafillcolor}
\pgfsetfillopacity{1.000000}
\node[anchor=base,inner sep=0pt, outer sep=0pt,color=dialinecolor] at (54.900000\du,10.293108\du){$T_{i m}$};
% setfont left to latex
\definecolor{dialinecolor}{rgb}{0.000000, 0.000000, 0.000000}
\pgfsetstrokecolor{dialinecolor}
\pgfsetstrokeopacity{1.000000}
\definecolor{diafillcolor}{rgb}{0.000000, 0.000000, 0.000000}
\pgfsetfillcolor{diafillcolor}
\pgfsetfillopacity{1.000000}
\node[anchor=base west,inner sep=0pt,outer sep=0pt,color=dialinecolor] at (54.950000\du,10.000000\du){};
\pgfsetlinewidth{0.100000\du}
\pgfsetdash{{\pgflinewidth}{0.200000\du}}{0cm}
\pgfsetmiterjoin
\pgfsetbuttcap
{
\definecolor{diafillcolor}{rgb}{0.000000, 0.000000, 0.000000}
\pgfsetfillcolor{diafillcolor}
\pgfsetfillopacity{1.000000}
% was here!!!
\pgfsetarrowsend{latex}
\definecolor{dialinecolor}{rgb}{0.000000, 0.000000, 0.000000}
\pgfsetstrokecolor{dialinecolor}
\pgfsetstrokeopacity{1.000000}
\pgfpathmoveto{\pgfpoint{49.000000\du}{8.000000\du}}
\pgfpathcurveto{\pgfpoint{49.000000\du}{5.500000\du}}{\pgfpoint{45.561270\du}{7.799996\du}}{\pgfpoint{45.545370\du}{5.571396\du}}
\pgfusepath{stroke}
}
\pgfsetlinewidth{0.100000\du}
\pgfsetdash{}{0pt}
\pgfsetbuttcap
\pgfsetmiterjoin
\pgfsetlinewidth{0.100000\du}
\pgfsetbuttcap
\pgfsetmiterjoin
\pgfsetdash{}{0pt}
\definecolor{dialinecolor}{rgb}{1.000000, 0.000000, 0.000000}
\pgfsetstrokecolor{dialinecolor}
\pgfsetstrokeopacity{1.000000}
\pgfpathellipse{\pgfpoint{46.000000\du}{10.000000\du}}{\pgfpoint{1.000000\du}{0\du}}{\pgfpoint{0\du}{1.000000\du}}
\pgfusepath{stroke}
% setfont left to latex
\definecolor{dialinecolor}{rgb}{0.000000, 0.000000, 0.000000}
\pgfsetstrokecolor{dialinecolor}
\pgfsetstrokeopacity{1.000000}
\definecolor{diafillcolor}{rgb}{0.000000, 0.000000, 0.000000}
\pgfsetfillcolor{diafillcolor}
\pgfsetfillopacity{1.000000}
\node[anchor=base,inner sep=0pt, outer sep=0pt,color=dialinecolor] at (46.000000\du,10.293108\du){$T_{i 2}$};
\pgfsetlinewidth{0.100000\du}
\pgfsetdash{}{0pt}
\pgfsetbuttcap
\pgfsetmiterjoin
\pgfsetlinewidth{0.100000\du}
\pgfsetbuttcap
\pgfsetmiterjoin
\pgfsetdash{}{0pt}
\definecolor{dialinecolor}{rgb}{1.000000, 0.000000, 0.000000}
\pgfsetstrokecolor{dialinecolor}
\pgfsetstrokeopacity{1.000000}
\pgfpathellipse{\pgfpoint{43.300000\du}{10.000000\du}}{\pgfpoint{1.000000\du}{0\du}}{\pgfpoint{0\du}{1.000000\du}}
\pgfusepath{stroke}
% setfont left to latex
\definecolor{dialinecolor}{rgb}{0.000000, 0.000000, 0.000000}
\pgfsetstrokecolor{dialinecolor}
\pgfsetstrokeopacity{1.000000}
\definecolor{diafillcolor}{rgb}{0.000000, 0.000000, 0.000000}
\pgfsetfillcolor{diafillcolor}
\pgfsetfillopacity{1.000000}
\node[anchor=base,inner sep=0pt, outer sep=0pt,color=dialinecolor] at (43.300000\du,10.293108\du){$T_{i 1}$};
\pgfsetlinewidth{0.100000\du}
\pgfsetdash{{\pgflinewidth}{0.300000\du}}{0cm}
\pgfsetbuttcap
{
\definecolor{diafillcolor}{rgb}{0.000000, 0.000000, 0.000000}
\pgfsetfillcolor{diafillcolor}
\pgfsetfillopacity{1.000000}
% was here!!!
\definecolor{dialinecolor}{rgb}{0.000000, 0.000000, 0.000000}
\pgfsetstrokecolor{dialinecolor}
\pgfsetstrokeopacity{1.000000}
\draw (47.500000\du,10.000000\du)--(53.451200\du,10.000000\du);
}
% setfont left to latex
\definecolor{dialinecolor}{rgb}{0.000000, 0.000000, 0.000000}
\pgfsetstrokecolor{dialinecolor}
\pgfsetstrokeopacity{1.000000}
\definecolor{diafillcolor}{rgb}{0.000000, 0.000000, 0.000000}
\pgfsetfillcolor{diafillcolor}
\pgfsetfillopacity{1.000000}
\node[anchor=base west,inner sep=0pt,outer sep=0pt,color=dialinecolor] at (51.470400\du,4.611210\du){};
% setfont left to latex
\definecolor{dialinecolor}{rgb}{0.000000, 0.000000, 0.000000}
\pgfsetstrokecolor{dialinecolor}
\pgfsetstrokeopacity{1.000000}
\definecolor{diafillcolor}{rgb}{0.000000, 0.000000, 0.000000}
\pgfsetfillcolor{diafillcolor}
\pgfsetfillopacity{1.000000}
\node[anchor=base,inner sep=0pt, outer sep=0pt,color=dialinecolor] at (49.017400\du,4.772288\du){$K_i$};
\pgfsetlinewidth{0.100000\du}
\pgfsetdash{{\pgflinewidth}{0.200000\du}}{0cm}
\pgfsetmiterjoin
\pgfsetbuttcap
{
\definecolor{diafillcolor}{rgb}{0.000000, 0.000000, 0.000000}
\pgfsetfillcolor{diafillcolor}
\pgfsetfillopacity{1.000000}
% was here!!!
\pgfsetarrowsend{latex}
\definecolor{dialinecolor}{rgb}{0.000000, 0.000000, 0.000000}
\pgfsetstrokecolor{dialinecolor}
\pgfsetstrokeopacity{1.000000}
\pgfpathmoveto{\pgfpoint{49.000000\du}{7.950812\du}}
\pgfpathcurveto{\pgfpoint{49.000000\du}{5.950812\du}}{\pgfpoint{49.005646\du}{8.076709\du}}{\pgfpoint{49.016746\du}{5.530619\du}}
\pgfusepath{stroke}
}
% setfont left to latex
\definecolor{dialinecolor}{rgb}{0.000000, 0.000000, 0.000000}
\pgfsetstrokecolor{dialinecolor}
\pgfsetstrokeopacity{1.000000}
\definecolor{diafillcolor}{rgb}{0.000000, 0.000000, 0.000000}
\pgfsetfillcolor{diafillcolor}
\pgfsetfillopacity{1.000000}
\node[anchor=base,inner sep=0pt, outer sep=0pt,color=dialinecolor] at (52.559900\du,4.823518\du){$\mathcal{C}_i$};
\pgfsetlinewidth{0.100000\du}
\pgfsetdash{{\pgflinewidth}{0.200000\du}}{0cm}
\pgfsetmiterjoin
\pgfsetbuttcap
{
\definecolor{diafillcolor}{rgb}{0.000000, 0.000000, 0.000000}
\pgfsetfillcolor{diafillcolor}
\pgfsetfillopacity{1.000000}
% was here!!!
\pgfsetarrowsend{latex}
\definecolor{dialinecolor}{rgb}{0.000000, 0.000000, 0.000000}
\pgfsetstrokecolor{dialinecolor}
\pgfsetstrokeopacity{1.000000}
\pgfpathmoveto{\pgfpoint{49.000000\du}{8.000000\du}}
\pgfpathcurveto{\pgfpoint{49.000000\du}{5.500000\du}}{\pgfpoint{52.576870\du}{7.812086\du}}{\pgfpoint{52.560970\du}{5.583486\du}}
\pgfusepath{stroke}
}
% setfont left to latex
\definecolor{dialinecolor}{rgb}{0.000000, 0.000000, 0.000000}
\pgfsetstrokecolor{dialinecolor}
\pgfsetstrokeopacity{1.000000}
\definecolor{diafillcolor}{rgb}{0.000000, 0.000000, 0.000000}
\pgfsetfillcolor{diafillcolor}
\pgfsetfillopacity{1.000000}
\node[anchor=base west,inner sep=0pt,outer sep=0pt,color=dialinecolor] at (57.292700\du,5.298960\du){};
\pgfsetlinewidth{0.100000\du}
\pgfsetdash{}{0pt}
\pgfsetbuttcap
\pgfsetmiterjoin
\pgfsetlinewidth{0.100000\du}
\pgfsetbuttcap
\pgfsetmiterjoin
\pgfsetdash{}{0pt}
\definecolor{diafillcolor}{rgb}{1.000000, 1.000000, 1.000000}
\pgfsetfillcolor{diafillcolor}
\pgfsetfillopacity{1.000000}
\definecolor{dialinecolor}{rgb}{1.000000, 0.000000, 0.000000}
\pgfsetstrokecolor{dialinecolor}
\pgfsetstrokeopacity{1.000000}
\pgfpathmoveto{\pgfpoint{55.594850\du}{3.641036\du}}
\pgfpathlineto{\pgfpoint{57.712975\du}{3.641036\du}}
\pgfpathcurveto{\pgfpoint{58.005427\du}{3.641036\du}}{\pgfpoint{58.242506\du}{4.039392\du}}{\pgfpoint{58.242506\du}{4.530789\du}}
\pgfpathcurveto{\pgfpoint{58.242506\du}{5.022185\du}}{\pgfpoint{58.005427\du}{5.420541\du}}{\pgfpoint{57.712975\du}{5.420541\du}}
\pgfpathlineto{\pgfpoint{55.594850\du}{5.420541\du}}
\pgfpathcurveto{\pgfpoint{55.302398\du}{5.420541\du}}{\pgfpoint{55.065319\du}{5.022185\du}}{\pgfpoint{55.065319\du}{4.530789\du}}
\pgfpathcurveto{\pgfpoint{55.065319\du}{4.039392\du}}{\pgfpoint{55.302398\du}{3.641036\du}}{\pgfpoint{55.594850\du}{3.641036\du}}
\pgfpathclose
\pgfusepath{fill,stroke}
% setfont left to latex
\definecolor{dialinecolor}{rgb}{0.000000, 0.000000, 0.000000}
\pgfsetstrokecolor{dialinecolor}
\pgfsetstrokeopacity{1.000000}
\definecolor{diafillcolor}{rgb}{0.000000, 0.000000, 0.000000}
\pgfsetfillcolor{diafillcolor}
\pgfsetfillopacity{1.000000}
\node[anchor=base,inner sep=0pt, outer sep=0pt,color=dialinecolor] at (56.653912\du,4.724851\du){covariates};
\pgfsetlinewidth{0.100000\du}
\pgfsetdash{{\pgflinewidth}{0.200000\du}}{0cm}
\pgfsetbuttcap
{
\definecolor{diafillcolor}{rgb}{0.000000, 0.000000, 0.000000}
\pgfsetfillcolor{diafillcolor}
\pgfsetfillopacity{1.000000}
% was here!!!
\pgfsetarrowsend{latex}
\definecolor{dialinecolor}{rgb}{0.000000, 0.000000, 0.000000}
\pgfsetstrokecolor{dialinecolor}
\pgfsetstrokeopacity{1.000000}
\draw (54.961385\du,4.530588\du)--(53.563000\du,4.530422\du);
}
\pgfsetlinewidth{0.100000\du}
\pgfsetdash{}{0pt}
\pgfsetbuttcap
\pgfsetmiterjoin
\pgfsetlinewidth{0.100000\du}
\pgfsetbuttcap
\pgfsetmiterjoin
\pgfsetdash{}{0pt}
\definecolor{dialinecolor}{rgb}{0.000000, 0.639216, 0.000000}
\pgfsetstrokecolor{dialinecolor}
\pgfsetstrokeopacity{1.000000}
\pgfpathellipse{\pgfpoint{42.048811\du}{5.683655\du}}{\pgfpoint{0.903074\du}{0\du}}{\pgfpoint{0\du}{0.903074\du}}
\pgfusepath{stroke}
% setfont left to latex
\definecolor{dialinecolor}{rgb}{0.000000, 0.000000, 0.000000}
\pgfsetstrokecolor{dialinecolor}
\pgfsetstrokeopacity{1.000000}
\definecolor{diafillcolor}{rgb}{0.000000, 0.000000, 0.000000}
\pgfsetfillcolor{diafillcolor}
\pgfsetfillopacity{1.000000}
\node[anchor=base west,inner sep=0pt,outer sep=0pt,color=dialinecolor] at (42.048811\du,5.683655\du){};
% setfont left to latex
\definecolor{dialinecolor}{rgb}{0.000000, 0.000000, 0.000000}
\pgfsetstrokecolor{dialinecolor}
\pgfsetstrokeopacity{1.000000}
\definecolor{diafillcolor}{rgb}{0.000000, 0.000000, 0.000000}
\pgfsetfillcolor{diafillcolor}
\pgfsetfillopacity{1.000000}
\node[anchor=base west,inner sep=0pt,outer sep=0pt,color=dialinecolor] at (42.048811\du,5.683655\du){};
% setfont left to latex
\definecolor{dialinecolor}{rgb}{0.000000, 0.000000, 0.000000}
\pgfsetstrokecolor{dialinecolor}
\pgfsetstrokeopacity{1.000000}
\definecolor{diafillcolor}{rgb}{0.000000, 0.000000, 0.000000}
\pgfsetfillcolor{diafillcolor}
\pgfsetfillopacity{1.000000}
\node[anchor=base,inner sep=0pt, outer sep=0pt,color=dialinecolor] at (42.088533\du,6.051128\du){$S_i$};
\pgfsetlinewidth{0.100000\du}
\pgfsetdash{{\pgflinewidth}{0.200000\du}}{0cm}
\pgfsetmiterjoin
\pgfsetbuttcap
{
\definecolor{diafillcolor}{rgb}{0.000000, 0.000000, 0.000000}
\pgfsetfillcolor{diafillcolor}
\pgfsetfillopacity{1.000000}
% was here!!!
\pgfsetarrowsend{latex}
{\pgfsetcornersarced{\pgfpoint{0.000000\du}{0.000000\du}}\definecolor{dialinecolor}{rgb}{0.000000, 0.000000, 0.000000}
\pgfsetstrokecolor{dialinecolor}
\pgfsetstrokeopacity{1.000000}
\draw (44.647100\du,4.524260\du)--(43.799493\du,4.524260\du)--(43.799493\du,5.683655\du)--(42.951885\du,5.683655\du);
}}
\pgfsetlinewidth{0.050000\du}
\pgfsetdash{{\pgflinewidth}{0.200000\du}}{0cm}
\pgfsetmiterjoin
\pgfsetbuttcap
{
\definecolor{diafillcolor}{rgb}{0.000000, 0.000000, 0.000000}
\pgfsetfillcolor{diafillcolor}
\pgfsetfillopacity{1.000000}
% was here!!!
\pgfsetarrowsend{latex}
\definecolor{dialinecolor}{rgb}{0.000000, 0.000000, 0.000000}
\pgfsetstrokecolor{dialinecolor}
\pgfsetstrokeopacity{1.000000}
\pgfpathmoveto{\pgfpoint{44.741289\du}{2.859181\du}}
\pgfpathcurveto{\pgfpoint{44.328640\du}{3.317680\du}}{\pgfpoint{44.324299\du}{3.421010\du}}{\pgfpoint{44.205140\du}{4.375415\du}}
\pgfusepath{stroke}
}
% setfont left to latex
\definecolor{dialinecolor}{rgb}{0.000000, 0.000000, 0.000000}
\pgfsetstrokecolor{dialinecolor}
\pgfsetstrokeopacity{1.000000}
\definecolor{diafillcolor}{rgb}{0.000000, 0.000000, 0.000000}
\pgfsetfillcolor{diafillcolor}
\pgfsetfillopacity{1.000000}
\node[anchor=base west,inner sep=0pt,outer sep=0pt,color=dialinecolor] at (44.909001\du,2.905296\du){$\tau$};
\pgfsetlinewidth{0.100000\du}
\pgfsetdash{}{0pt}
\pgfsetbuttcap
\pgfsetmiterjoin
\pgfsetlinewidth{0.100000\du}
\pgfsetbuttcap
\pgfsetmiterjoin
\pgfsetdash{}{0pt}
\definecolor{dialinecolor}{rgb}{0.000000, 0.639216, 0.000000}
\pgfsetstrokecolor{dialinecolor}
\pgfsetstrokeopacity{1.000000}
\pgfpathellipse{\pgfpoint{42.051403\du}{3.420130\du}}{\pgfpoint{0.903074\du}{0\du}}{\pgfpoint{0\du}{0.903074\du}}
\pgfusepath{stroke}
% setfont left to latex
\definecolor{dialinecolor}{rgb}{0.000000, 0.000000, 0.000000}
\pgfsetstrokecolor{dialinecolor}
\pgfsetstrokeopacity{1.000000}
\definecolor{diafillcolor}{rgb}{0.000000, 0.000000, 0.000000}
\pgfsetfillcolor{diafillcolor}
\pgfsetfillopacity{1.000000}
\node[anchor=base west,inner sep=0pt,outer sep=0pt,color=dialinecolor] at (42.051403\du,3.420130\du){};
% setfont left to latex
\definecolor{dialinecolor}{rgb}{0.000000, 0.000000, 0.000000}
\pgfsetstrokecolor{dialinecolor}
\pgfsetstrokeopacity{1.000000}
\definecolor{diafillcolor}{rgb}{0.000000, 0.000000, 0.000000}
\pgfsetfillcolor{diafillcolor}
\pgfsetfillopacity{1.000000}
\node[anchor=base west,inner sep=0pt,outer sep=0pt,color=dialinecolor] at (42.051403\du,3.420130\du){};
% setfont left to latex
\definecolor{dialinecolor}{rgb}{0.000000, 0.000000, 0.000000}
\pgfsetstrokecolor{dialinecolor}
\pgfsetstrokeopacity{1.000000}
\definecolor{diafillcolor}{rgb}{0.000000, 0.000000, 0.000000}
\pgfsetfillcolor{diafillcolor}
\pgfsetfillopacity{1.000000}
\node[anchor=base,inner sep=0pt, outer sep=0pt,color=dialinecolor] at (42.091125\du,3.736728\du){$\delta_i$};
\pgfsetlinewidth{0.100000\du}
\pgfsetdash{{\pgflinewidth}{0.200000\du}}{0cm}
\pgfsetmiterjoin
\pgfsetbuttcap
{
\definecolor{diafillcolor}{rgb}{0.000000, 0.000000, 0.000000}
\pgfsetfillcolor{diafillcolor}
\pgfsetfillopacity{1.000000}
% was here!!!
\pgfsetarrowsend{latex}
{\pgfsetcornersarced{\pgfpoint{0.000000\du}{0.000000\du}}\definecolor{dialinecolor}{rgb}{0.000000, 0.000000, 0.000000}
\pgfsetstrokecolor{dialinecolor}
\pgfsetstrokeopacity{1.000000}
\draw (44.597249\du,4.524260\du)--(43.775863\du,4.524260\du)--(43.775863\du,3.420130\du)--(42.954477\du,3.420130\du);
}}
\end{tikzpicture}
}}
\caption{This figure showcases a dependency graph of the generative model for
$D_i = (S_i,\delta_i,\mathcal{C}_i)$. The elements in green are observed in the sample,
while the elements in red are unobserved (latent). We see that $\mathcal{C}_i$ is
related to both the unobserved component lifetimes $T_{i 1},\ldots,T_{i m}$ and
other unknown and unobserved covariates, like ambient temperature or the
particular diagnostician who generated the candidate set. These two complications
for $\mathcal{C}_i$ are why seek a way to construct a reduced likelihood function
in later sections that is not a function of the distribution of $\mathcal{C}_i$.}
\label{fig:figureone}
\end{figure}

An example of masked data \(D\) for exact, right-censored system failure
times with candidate sets that mask the component cause of failure can
be seen in Table 1 for a series system with \(m=3\) components.

\begin{longtable}[]{@{}llll@{}}
\caption{Right-censored lifetime data with masked component cause of
failure.}\tabularnewline
\toprule
\begin{minipage}[b]{0.07\columnwidth}\raggedright
System\strut
\end{minipage} & \begin{minipage}[b]{0.30\columnwidth}\raggedright
Right-censoring time (\(S_i\))\strut
\end{minipage} & \begin{minipage}[b]{0.30\columnwidth}\raggedright
Right censoring indicator (\(\delta_i\))\strut
\end{minipage} & \begin{minipage}[b]{0.23\columnwidth}\raggedright
Candidate set (\(\mathcal{C}_i\))\strut
\end{minipage}\tabularnewline
\midrule
\endfirsthead
\toprule
\begin{minipage}[b]{0.07\columnwidth}\raggedright
System\strut
\end{minipage} & \begin{minipage}[b]{0.30\columnwidth}\raggedright
Right-censoring time (\(S_i\))\strut
\end{minipage} & \begin{minipage}[b]{0.30\columnwidth}\raggedright
Right censoring indicator (\(\delta_i\))\strut
\end{minipage} & \begin{minipage}[b]{0.23\columnwidth}\raggedright
Candidate set (\(\mathcal{C}_i\))\strut
\end{minipage}\tabularnewline
\midrule
\endhead
\begin{minipage}[t]{0.07\columnwidth}\raggedright
1\strut
\end{minipage} & \begin{minipage}[t]{0.30\columnwidth}\raggedright
\(4.3\)\strut
\end{minipage} & \begin{minipage}[t]{0.30\columnwidth}\raggedright
1\strut
\end{minipage} & \begin{minipage}[t]{0.23\columnwidth}\raggedright
\(\{1,2\}\)\strut
\end{minipage}\tabularnewline
\begin{minipage}[t]{0.07\columnwidth}\raggedright
2\strut
\end{minipage} & \begin{minipage}[t]{0.30\columnwidth}\raggedright
\(1.3\)\strut
\end{minipage} & \begin{minipage}[t]{0.30\columnwidth}\raggedright
1\strut
\end{minipage} & \begin{minipage}[t]{0.23\columnwidth}\raggedright
\(\{2\}\)\strut
\end{minipage}\tabularnewline
\begin{minipage}[t]{0.07\columnwidth}\raggedright
3\strut
\end{minipage} & \begin{minipage}[t]{0.30\columnwidth}\raggedright
\(5.4\)\strut
\end{minipage} & \begin{minipage}[t]{0.30\columnwidth}\raggedright
0\strut
\end{minipage} & \begin{minipage}[t]{0.23\columnwidth}\raggedright
\(\emptyset\)\strut
\end{minipage}\tabularnewline
\begin{minipage}[t]{0.07\columnwidth}\raggedright
4\strut
\end{minipage} & \begin{minipage}[t]{0.30\columnwidth}\raggedright
\(2.6\)\strut
\end{minipage} & \begin{minipage}[t]{0.30\columnwidth}\raggedright
1\strut
\end{minipage} & \begin{minipage}[t]{0.23\columnwidth}\raggedright
\(\{2,3\}\)\strut
\end{minipage}\tabularnewline
\begin{minipage}[t]{0.07\columnwidth}\raggedright
5\strut
\end{minipage} & \begin{minipage}[t]{0.30\columnwidth}\raggedright
\(3.7\)\strut
\end{minipage} & \begin{minipage}[t]{0.30\columnwidth}\raggedright
1\strut
\end{minipage} & \begin{minipage}[t]{0.23\columnwidth}\raggedright
\(\{1,2,3\}\)\strut
\end{minipage}\tabularnewline
\begin{minipage}[t]{0.07\columnwidth}\raggedright
6\strut
\end{minipage} & \begin{minipage}[t]{0.30\columnwidth}\raggedright
\(10\)\strut
\end{minipage} & \begin{minipage}[t]{0.30\columnwidth}\raggedright
0\strut
\end{minipage} & \begin{minipage}[t]{0.23\columnwidth}\raggedright
\(\emptyset\)\strut
\end{minipage}\tabularnewline
\bottomrule
\end{longtable}

In our model, we assume the data is governed by a pdf, which is
determined by a specific parameter, represented as
\(\boldsymbol{\theta}\) within the parameter space
\(\boldsymbol{\Omega}\). The joint pdf of the data \(D\) can be
represented as follows: \[
f(D ; \boldsymbol{\theta}) = \prod_{i=1}^n f(s_i,\delta_i,c_i;\boldsymbol{\theta}),
\] where \(s_i\) is the observed system lifetime of the
\(i\)\textsuperscript{th} system, \(\delta_i\) is the observed
right-censoring indicator of the \(i\)\textsuperscript{th} system, and
\(c_i\) is the observed candidate set of the \(i\)\textsuperscript{th}
system.

This joint pdf tells us how likely we are to observe the particular
data, \(D\), given the parameter \(\boldsymbol{\theta}\). When we keep
the data constant and allow the parameter \(\boldsymbol{\theta}\) to
vary, we obtain what is called the likelihood function \(L\), defined as
\[
L(\boldsymbol{\theta}) = \prod_{i=1}^n L_i(\boldsymbol{\theta})
\] where \[
L_i(\boldsymbol{\theta}) = f(s_i,\delta_i,c_i;\boldsymbol{\theta})
\] is the likelihood contribution of the \(i\)\textsuperscript{th}
system. In other words, the likelihood function quantifies how likely
different parameter values \(\boldsymbol{\theta}\) are, given the
observed data.

For each type of data, right-censored data and masked component cause of
failure data, we will derive the \emph{likelihood contribution} \(L_i\),
which refers to the part of the likelihood function that this particular
piece of data contributes to.

We present the following theorem for the likelihood contribution model.

\begin{theorem}
\label{thm:likelihood_contribution}
The likelihood contribution of the $i$-th system is given by
\begin{equation}
\label{eq:like}
L_i(\boldsymbol{\theta}) =
\begin{cases}
    R_{T_i}(s_i;\boldsymbol{\theta})                      &\text{ if } \delta_i = 0\\
    \beta_i R_{T_i}(s_i;\boldsymbol{\theta})
        \sum_{j\in c_i} h_j(s_i;\boldsymbol{\theta_j})   &\text{ if } \delta_i = 1,
\end{cases}
\end{equation}
where $\delta_i = 0$ indicates the $i$\textsuperscript{th} system is
right-censored at time $s_i$ and $\delta_i = 1$ indicates the $i$\textsuperscript{th} system
is observed to have failed at time $s_i$ and the component cause of failure
is masked by the candidate set is $c_i$.
\end{theorem}

In the follow subsections, we prove this result for each type of masked
data, right-censored system lifetime data \((\delta_i = 0)\) and masking
of the component cause of failure \((\delta_i = 1)\).

\hypertarget{sec:candmod}{%
\subsection{Masked Component Cause of Failure}\label{sec:candmod}}

Suppose a diagnostician is unable to identify the precise component
cause of the failure, e.g., due to cost considerations he or she
replaced multiple components at once, successfully repairing the system
but failing to precisely identity the failed component. In this case,
the cause of failure is said to be \emph{masked}.

The unobserved component lifetimes may have many covariates, like
ambient operating temperature, but the only covariate we observe in our
masked data model are the system's lifetime and additional masked data
in the form of a candidate set that is somehow correlated with the
unobserved component lifetimes.

The key goal of our analysis is to estimate the parameters,
\(\boldsymbol{\theta}\), which maximize the likelihood of the observed
data, and to estimate the precision and accuracy of this estimate using
the Bootstrap method.

To achieve this, we first need to assess the joint distribution of the
system's continuous lifetime, \(T_i\), and the discrete candidate set,
\(\mathcal{C}_i\), which can be written as \[
f_{T_i,\mathcal{C}_i}(t_i,c_i;\boldsymbol{\theta}) = f_{T_i}(t_i;\boldsymbol{\theta})
    \Pr{}_{\!\boldsymbol{\theta}}\{\mathcal{C}_i = c_i | T_i = t_i\},
\] where \(f_{T_i}(t_i;\boldsymbol{\theta})\) is the pdf of \(T_i\) and
\(\Pr{}_{\!\boldsymbol{\theta}}\{\mathcal{C}_i = c_i | T_i = t_i\}\) is
the conditional pmf of \(\mathcal{C}_i\) given \(T_i = t_i\).

We assume the pdf \(f_{T_i}(t_i;\boldsymbol{\theta})\) is known, but we
do not have knowledge of
\(\Pr{}_{\!\boldsymbol{\theta}}\{\mathcal{C}_i = c_i | T_i = t_i\}\),
i.e., the data generating process for candidate sets is unknown.

However, it is critical that the masked data, \(\mathcal{C}_i\), is
correlated with the \(i\)\textsuperscript{th} system. This way, the
conditional distribution of \(\mathcal{C}_i\) given \(T_i = t_i\) may
provide information about \(\boldsymbol{\theta}\), despite our
Statistical interest being primarily in the series system rather than
the candidate sets.

To make this problem tractable, we assume a set of conditions that make
it unnecessary to estimate the generative processes for candidate sets.
The most important way in which \(\mathcal{C}_i\) is correlated with the
\(i\)\textsuperscript{th} system is given by assuming the following
condition.

\begin{condition}
\label{cond:c_contains_k}
The candidate set $\mathcal{C}_i$ contains the index of the the failed component, i.e.,
$$
\Pr{}_{\!\boldsymbol{\theta}}\{K_i \in \mathcal{C}_i\} = 1
$$
where $K_i$ is the random variable for the failed component index of the
$i$\textsuperscript{th} system.
\end{condition}

Assuming Condition \ref{cond:c_contains_k}, \(\mathcal{C}_i\) must
contain the index of the failed component, but we can say little else
about what other component indices may appear in \(\mathcal{C}_i\).

In order to derive the joint distribution of \(\mathcal{C}_i\) and
\(T_i\) assuming Condition \ref{cond:c_contains_k}, we take the
following approach. We notice that \(\mathcal{C}_i\) and \(K_i\) are
statistically dependent. We denote the conditional pmf of
\(\mathcal{C}_i\) given \(T_i = t_i\) and \(K_i = j\) as \[
\Pr{}_{\!\boldsymbol{\theta}}\{\mathcal{C}_i = c_i | T_i = t_i, K_i = j\}.
\]

Even though \(K_i\) is not observable in our masked data model, we can
still consider the joint distribution of \(T_i\), \(K_i\), and
\(\mathcal{C}_i\). By Theorem \ref{thm:f_k_and_t}, the joint pdf of
\(T_i\) and \(K_i\) is given by \[
f_{T_i,K_i}(t_i,j;\boldsymbol{\theta}) = h_j(t_i;\boldsymbol{\theta_j}) R_{T_i}(t_i;\boldsymbol{\theta}),
\] where \(h_j(t_i;\boldsymbol{\theta_j})\) is the hazard function for
the \(j\)\textsuperscript{th} component and
\(R_{T_i}(t_i;\boldsymbol{\theta})\) is the reliability function of the
system. Thus, the joint pdf of \(T_i\), \(K_i\), and \(\mathcal{C}_i\)
may be written as \begin{equation}
\label{eq:joint_pdf_t_k_c}
\begin{split}
f_{T_i,K_i,\mathcal{C}_i}(t_i,j,c_i;\boldsymbol{\theta})
    &= f_{T_i,K_i}(t_i,k;\boldsymbol{\theta}) \Pr{}_{\!\boldsymbol{\theta}}\{\mathcal{C}_i=c_i|T_i=t_i,K_i=j\}\\
    &= h_j(t_i;\boldsymbol{\theta_j}) R_{T_i}(t_i;\boldsymbol{\theta})
    \Pr{}_{\!\boldsymbol{\theta}}\{\mathcal{C}_i=c_i|T_i=t_i,K_i=j\}.
\end{split}
\end{equation} We are going to need the joint pdf of \(T_i\) and
\(\mathcal{C}_i\), which may be obtained by summing over the support
\(\{1,\ldots,m\}\) of \(K_i\) in Equation \eqref{eq:joint_pdf_t_k_c}, \[
f_{T_i,\mathcal{C}_i}(t_i,c_i;\boldsymbol{\theta}) = R_{T_i}(t_i;\boldsymbol{\theta})
    \sum_{j=1}^m \biggl\{
        h_j(t_i;\boldsymbol{\theta_j}) \Pr{}_{\!\boldsymbol{\theta}}\{\mathcal{C}_i=c_i|T_i=t_i,K_i=j\}
    \biggr\}.
\] By Condition \ref{cond:c_contains_k},
\(\Pr{}_{\!\boldsymbol{\theta}}\{\mathcal{C}_i=c_i|T_i=t_i,K_i=j\} = 0\)
when \(K_i = j\) and \(j \notin c_i\), and so we may rewrite the joint
pdf of \(T_i\) and \(\mathcal{C}_i\) as \begin{equation}
\label{eq:part1}
f_{T_i,\mathcal{C}_i}(t_i,c_i;\boldsymbol{\theta}) = R_{T_i}(t_i;\boldsymbol{\theta})
    \sum_{j \in c_i} \biggl\{
        h_j(t_i;\boldsymbol{\theta_j}) \Pr{}_{\!\boldsymbol{\theta}}\{\mathcal{C}_i=c_i|T_i=t_i,K_i=j\}
    \biggr\}.
\end{equation}

When we try to find an MLE of \(\boldsymbol{\theta}\) (see Section
\ref{sec:mle}), we solve the simultaneous equations of the MLE and
choose a solution \(\hat{\boldsymbol{\theta}}\) that is a maximum for
the likelihood function. When we do this, we find that
\(\hat{\boldsymbol{\theta}}\) depends on the unknown conditional pmf
\(\Pr{}_{\!\boldsymbol{\theta}}\{\mathcal{C}_i=c_i|T_i=t_i,K_i=j\}\).
So, we are motivated to seek out more conditions (that approximately
hold in realistic situations) whose MLEs are independent of the pmf
\(\Pr{}_{\!\boldsymbol{\theta}}\{\mathcal{C}_i=c_i|T_i=t_i,K_i=j\}\).

\begin{condition}
\label{cond:equal_prob_failure_cause}
Any of the components in the candidate set has an equal probability of being the
cause of failure.
That is, for a fixed $j \in c_i$,
$$
\Pr{}_{\!\boldsymbol{\theta}}\{\mathcal{C}_i=c_i|T_i=t_i,K_i=j'\} =
    \Pr{}_{\!\boldsymbol{\theta}}\{\mathcal{C}_i=c_i|T_i=t_i,K_i=j\}
$$
for all $j' \in c_i$.
\end{condition}

According to \citep{Fran-1991}, in many industrial problems, masking
generally occurred due to time constraints and the expense of failure
analysis. In this setting, Condition \ref{cond:equal_prob_failure_cause}
generally holds.

Assuming Conditions \ref{cond:c_contains_k} and
\ref{cond:equal_prob_failure_cause},
\(\Pr{}_{\!\boldsymbol{\theta}}\{\mathcal{C}_i=c_i|T_i=t_i,K_i=j\}\) may
be factored out of the summation in Equation \eqref{eq:part1}, and thus
the joint pdf of \(T_i\) and \(\mathcal{C}_i\) may be rewritten as \[
f_{T_i,\mathcal{C}_i}(t_i,c_i;\boldsymbol{\theta}) =
    \Pr{}_{\!\boldsymbol{\theta}}\{\mathcal{C}_i=c_i|T_i=t_i,K_i=j'\} R_{T_i}(t_i;\boldsymbol{\theta})
    \sum_{j \in c_i} h_j(t_i;\boldsymbol{\theta_j})
\] where \(j' \in c_i\).

If \(\Pr{}_{\!\boldsymbol{\theta}}\{\mathcal{C}_i=c_i|T_i=t_i,K_i=j'\}\)
is a function of \(\boldsymbol{\theta}\), the MLEs are still dependent
on the unknown
\(\Pr{}_{\!\boldsymbol{\theta}}\{\mathcal{C}_i=c_i|T_i=t_i,K_i=j'\}\).
This is a more tractable problem, but we are primarily interested in the
situation where we do not need to know (nor estimate)
\(\Pr{}_{\!\boldsymbol{\theta}}\{\mathcal{C}_i=c_i|T_i=t_i,K_i=j'\}\) to
find an MLE of \(\boldsymbol{\theta}\). The last condition we assume
achieves this result.

\begin{condition}
\label{cond:masked_indept_theta}
The masking probabilities conditioned on failure time $T_i$ and component cause
of failure $K_i$ are not functions of $\boldsymbol{\theta}$. In this case, the conditional
probability of $\mathcal{C}_i$ given $T_i=t_i$ and $K_i=j'$ is denoted by
$$
\beta_i = \Pr\{\mathcal{C}_i=c_i | T_i=t_i, K_i=j'\}
$$
where $\beta_i$ is not a function of $\boldsymbol{\theta}$.
\end{condition}

When Conditions \ref{cond:c_contains_k},
\ref{cond:equal_prob_failure_cause}, and \ref{cond:masked_indept_theta}
are satisfied, the joint pdf of \(T_i\) and \(\mathcal{C}_i\) is given
by \[
f_{T_i,\mathcal{C}_i}(t_i,c_i;\boldsymbol{\theta}) =
    \beta_i R_{T_i}(t_i;\boldsymbol{\theta})
    \sum_{j \in c_i} h_j(t_i;\boldsymbol{\theta_j}).
\] When we fix the sample and allow \(\boldsymbol{\theta}\) to vary, we
obtain the contribution to the likelihood \(L\) from the
\(i\)\textsuperscript{th} observation when the system lifetime is
exactly known (i.e., \(\delta_i = 1\)) but the component cause of
failure is masked by a candidate set \(c_i\): \begin{equation}
\label{eq:likelihood_contribution_masked}
L_i(\boldsymbol{\theta}) = R_{T_i}(t_i;\boldsymbol{\theta}) \sum_{j \in c_i} h_j(t_i;\boldsymbol{\theta_j}).
\end{equation}

To summarize this result, assuming Conditions \ref{cond:c_contains_k},
\ref{cond:equal_prob_failure_cause}, and \ref{cond:masked_indept_theta},
if we observe an exact system failure time for the \(i\)-th system
(\(\delta_i = 1\)), but the component that failed is masked by a
candidate set \(c_i\), then its likelihood contribution is given by
Equation \eqref{eq:likelihood_contribution_masked}.

\hypertarget{right-censored-data}{%
\subsection{Right-Censored Data}\label{right-censored-data}}

As described in Section \ref{sec:like_model}, we observe realizations of
\((S_i,\delta_i,\mathcal{C}_i)\) where \(S_i = \min\{T_i,\tau_i\}\) is
the right-censored system lifetime, \(\delta_i = 1_{\{T_i < \tau_i\}}\)
is the right-censoring indicator, and \(\mathcal{C}_i\) is the candidate
set.

In the previous section, we discussed the likelihood contribution from
an observation of a masked component cause of failure, i.e.,
\(\delta_i = 1\). We now derive the likelihood contribution of a
\emph{right-censored} observation \((\delta_i = 0\)) in our masked data
model.

\begin{theorem}
\label{thm:joint_s_d_c}
The likelihood contribution of a right-censored observation $(\delta_i = 0)$
is given by
\begin{equation}
L_i(\boldsymbol{\theta}) = R_{T_i}(s_i;\boldsymbol{\theta}).
\end{equation}
\end{theorem}
\begin{proof}
When right-censoring occurs, then $S_i = \tau_i$, and we only know that
$T_i > \tau_i$, and so we integrate over all possible values that it may have
obtained,
$$
L_i(\boldsymbol{\theta}) = \Pr\!{}_{\boldsymbol{\theta}}\{T_i > s_i\}.
$$
By definition, this is just the survival or reliability function of the series system
evaluated at $s_i$,
$$
L_i(\boldsymbol{\theta}) = R_{T_i}(s_i;\boldsymbol{\theta}).
$$
\end{proof}

When we combine the two likelihood contributions, we obtain the
likelihood contribution for the \(i\)\textsuperscript{th} system shown
in Theorem \ref{thm:likelihood_contribution}, \[
L_i(\boldsymbol{\theta}) =
\begin{cases}
    R_{T_i}(s_i;\boldsymbol{\theta})                      &\text{ if } \delta_i = 0\\
    \beta_i R_{T_i}(s_i;\boldsymbol{\theta})
        \sum_{j\in c_i} h_j(s_i;\boldsymbol{\theta_j})   &\text{ if } \delta_i = 1.
\end{cases}
\]

We use this result in the next section to derive the maximum likelihood
estimator of \(\boldsymbol{\theta}\).

\hypertarget{sec:mle}{%
\section{Maximum Likelihood Estimation}\label{sec:mle}}

In our analysis, we use maximum likelihood estimation (MLE) to estimate
the series system parameter \(\boldsymbol{\theta}\) from the masked data
\citep{bain, casella2002statistical}. The MLE finds parameter values
that maximize the likelihood of the observed data under the assumed
model. The maximum likelihood estimate, \(\hat{\boldsymbol{\theta}}\),
is the solution of:

\begin{equation}
\label{eq:mle}
L(\hat{\boldsymbol{\theta}}) = \max_{\boldsymbol{\theta }\in \boldsymbol{\Omega}} L(\boldsymbol{\theta}),
\end{equation}

where \(L(\boldsymbol{\theta})\) is the likelihood function of the
observed data. For computational and analytical simplicity, we work with
the log-likelihood function, denoted as \(\ell(\boldsymbol{\theta})\),
instead of the likelihood function \citep{casella2002statistical}.

\begin{theorem}
\label{thm:loglike_total}
The log-likelihood function, $\ell(\boldsymbol{\theta})$, for our masked data model is the sum of the log-likelihoods for each observation,

\begin{equation}
\label{eq:loglike}
\ell(\boldsymbol{\theta}) = \sum_{i=1}^n \ell_i(\boldsymbol{\theta}),
\end{equation}
where $\ell_i(\boldsymbol{\theta})$ is the log-likelihood contribution for the $i$\textsuperscript{th} observation:
\begin{equation}
\ell_i(\boldsymbol{\theta}) = \sum_{j=1}^m \log R_j(s_i;\boldsymbol{\theta_j}) +
    \delta_i \log \bigl(\sum_{j\in c_i} h_j(s_i;\boldsymbol{\theta_j}) \bigr).
\end{equation}
\end{theorem}
\begin{proof}
The log-likelihood function is the logarithm of the likelihood function,
$$
\ell(\boldsymbol{\theta}) = \log L(\boldsymbol{\theta}) = \log \prod_{i=1}^n L_i(\boldsymbol{\theta}) = \sum_{i=1}^n \log L_i(\boldsymbol{\theta}).
$$
Substituting $L_i(\boldsymbol{\theta})$ from Equation \eqref{eq:like} and separating the two cases of $\delta_i$, we get

\textbf{Case 1}: If the $i$-th system is right-censored ($\delta_i = 0$),
$$
\ell_i(\boldsymbol{\theta}) = \log R_{T_i}(s_i;\boldsymbol{\theta}) = \sum_{j=1}^m \log R_j(s_i;\boldsymbol{\theta_j}).
$$

\textbf{Case 2}: If the $i$-th system's component cause of failure is masked but the failure time is known ($\delta_i = 1$),
\begin{align*}
\ell_i(\boldsymbol{\theta})
    &= \log R_{T_i}(s_i;\boldsymbol{\theta}) + \log \beta_i + \log \bigl(\sum_{j\in c_i} h_j(s_i;\boldsymbol{\theta_j})\bigr) \\
    &= \sum_{j=1}^m \log R_j(s_i;\boldsymbol{\theta_j}) + \log \bigl(\sum_{j\in c_i} h_j(s_i;\boldsymbol{\theta_j}) \biggr).
\end{align*}
By Condition \ref{cond:masked_indept_theta}, we may ignore the term $\log \beta_i$ in the MLE since it does not
depend on $\boldsymbol{\theta}$. This gives us the result in Theorem \ref{thm:loglike_total}.
\end{proof}

The MLE, \(\hat{\boldsymbol{\theta}}\), is often found by solving a
system of equations derived from setting the derivative of the
log-likelihood function to zero, i.e., \begin{equation}
\label{eq:mle_eq}
\frac{\partial}{\partial \theta_j} \ell(\boldsymbol{\theta}) = 0,
\end{equation} for each component \(\theta_j\) of the parameter
\(\boldsymbol{\theta}\) \citep{bain}. When there's no closed-form
solution, we resort to numerical methods like the Newton-Raphson method.

MLE has desirable asymptotic properties that underpin statistical
inference, namely that it is asymptotically unbiased, unique, and
normally distributed, with a variance given by the inverse of the Fisher
Information Matrix (FIM) \citep{casella2002statistical}. However, for
smaller samples or complex models, these asymptotic properties may not
yield accurate approximations. Hence, we propose to use the bootstrap
method to offer an empirical approach for estimating the sampling
distribution of the MLE.

\hypertarget{sec:boot}{%
\section{Bias-Corrected and Accelerated Bootstrap Confidence
Intervals}\label{sec:boot}}

We utilize the non-parametric bootstrap to approximate the sampling
distribution of the MLE. In the non-parametric bootstrap, we resample
from the observed data with replacement to generate a bootstrap sample.
The MLE is then computed for the bootstrap sample. This process is
repeated \(B\) times, giving us \(B\) bootstrap replicates of the MLE.
The sampling distribution of the MLE is then approximated by the
empirical distribution of the bootstrap replicates of the MLE.

The method we use to generate confidence intervals is known as
Bias-Corrected and Accelerated Bootstrap Confidence Intervals (BCa),
which applies two corrections to the standard bootstrap method:

\begin{itemize}
\item
  Bias correction: This adjusts for bias in the bootstrap distribution
  itself. This bias is measured as the difference between the mean of
  the bootstrap distribution and the observed statistic. It works by
  transforming the percentiles of the bootstrap distribution to correct
  for these issues.

  This may be a useful transformation in our case since we are dealing
  with small samples and we have two potential sources of bias:
  right-censoring and masking component cause of failure. They seem to
  have opposing effects on the MLE, but the relationship is difficult to
  quantify.
\item
  Acceleration: This adjusts for the rate of change of the statistic as
  a function of the true, unknown parameter. This correction is
  important when the shape of the statistic's distribution changes with
  the true parameter.

  Since we have a number of different shape parameters,
  \(k_1,\ldots,k_m\), we may expect the shape of the distribution of the
  MLE to change as a function of the true parameter, making this
  correction potentially useful.
\end{itemize}

Since we are primarly interested in generating confidence intervals for
small samples (otherwise the inverse FIM would be a good approximation)
for a potentially biased MLE for the parameters of Weibull components in
a series configuration (see Sections \ref{sec:effect-censoring} and
\ref{sec:effect-masking}), we think the BCa method is a good choice for
our analysis. For more details on BCa, see \cite{efron1987better}.

In our simulation study, we will assess the performance of the
bootstrapped confidence intervals by computing the coverage probability
of the confidence intervals. A 95\% confidence interval should contain
the true value 95\% of the time. If the confidence interval is too
narrow, it will have a coverage probability less than 95\%, which
conveys a sort of false confidence in the precision of the MLE. If the
confidence interval is too wide, it will have a coverage probability
greater than 95\%, which conveys a lack of confidence in the precision
of the MLE. Thus, we want the confidence interval to be as narrow as
possible while still having a coverage probability close to the nominal
level, 95\%.

\hypertarget{issues-with-resampling-from-the-observed-data}{%
\subsubsection*{Issues with Resampling from the Observed
Data}\label{issues-with-resampling-from-the-observed-data}}
\addcontentsline{toc}{subsubsection}{Issues with Resampling from the
Observed Data}

While the bootstrap method provides a robust and flexible tool for
statistical estimation, its effectiveness can be influenced by several
factors \citep{efron1994introduction}.

Firstly, instances of non-convergence in our bootstrap samples were
observed. Such cases can occur when the estimation method, like the MLE
used in our analysis, fails to converge due to the specifics of the
resampled data \citep{casella2002statistical}. This issue can
potentially introduce bias or reduce the effective sample size of our
bootstrap distribution.

Secondly, the bootstrap's accuracy can be compromised with small sample
sizes, as the method relies on the law of large numbers to approximate
the true sampling distribution. For small datasets, the bootstrap
samples might not adequately represent the true variability in the data,
leading to inaccurate results \citep{efron1994introduction}.

Thirdly, our data involves right censoring and a masking of the
component cause of failure when a system failure is observed. These
aspects can cause certain data points or trends to be underrepresented
or not represented at all in our data, introducing bias in the bootstrap
distribution \citep{klein2005survival}.

Despite these challenges, we found the bootstrap method useful in
approximating the sampling distribution of the MLE, taking care in
interpreting the results, particularly as it relates to coverage
probabilities.

\hypertarget{sec:weibull}{%
\section{Series System with Weibull Components}\label{sec:weibull}}

In the real world, systems are quite complex:

\begin{enumerate}
\def\labelenumi{\arabic{enumi}.}
\item
  They are not perfect series systems.
\item
  The components in a system are not independent.
\item
  The lifetimes of the components are not precisely modeled by any named
  probability distributions.
\item
  The components may depend on many other unobserved factors.
\end{enumerate}

With these caveats in mind, we model the data as coming from a Weibull
series system of \(m = 5\) components, and other factors, like ambient
temperature, are either negligible (on the distribution of component
lifetimes) or are more or less constant.

The \(j\)\textsuperscript{th} component of the \(i\)\textsuperscript{th}
has a lifetime distribution given by \[
    T_{i j} \sim \operatorname{WEI}(\boldsymbol{\theta_j})
\] where \(\boldsymbol{\theta_j} = (k_j, \lambda_j)\) for
\(j=1,\ldots,m\). Thus,
\(\boldsymbol{\theta }= (\boldsymbol{\theta_1},\ldots,\boldsymbol{\theta_m})' = \bigl(k_1,\lambda_1,\ldots,k_m,\lambda_m\bigr)\).
The random variable \(T_{i j}\) has a reliability function, pdf, and
hazard function given respectively by \begin{align}
    R_j(t;\lambda_j,k_j)
        &= \exp\biggl\{-\biggl(\frac{t}{\lambda_j}\biggr)^{k_j}\biggr\},\\
    f_j(t;\lambda_j,k_j)
        &= \frac{k_j}{\lambda_j}\biggl(\frac{t}{\lambda_j}\biggr)^{k_j-1}
        \exp\biggl\{-\left(\frac{t}{\lambda_j}\right)^{k_j} \biggr\},\\
    h_j(t;\lambda_j,k_j) \label{eq:weibull_haz}
        &= \frac{k_j}{\lambda_j}\biggl(\frac{t}{\lambda_j}\biggr)^{k_j-1}
\end{align} where \(t > 0\) is the lifetime, \(\lambda_j > 0\) is the
scale parameter and \(k_j > 0\) is the shape parameter. The shape
parameters \(k_1, \ldots, k_m\) have the following interpretations:

\begin{enumerate}
\item[$k_j < 1$] The hazard function decreases with respect to time. For instance,
  this may occur as a result of defective components being weeded out early. This
  is known as the *infant mortality* phase.
\item[$k_j = 1$] The hazard function is constant with respect to time. This is an
  idealized case that is rarely observed in practice, but may be useful for modeling
  purposes.
\item[$k_j > 1$] The hazard function increases with respect to time. For instance,
  this may occur as a result of components wearing out. This is known as the
  *aging* phase.
\end{enumerate}

The lifetime of the series system composed of \(m\) Weibull components
has a reliability function given by \begin{equation}
\label{eq:sys_weibull_reliability_function}
R(t;\boldsymbol{\theta}) = \exp\biggl\{-\sum_{j=1}^{m}\biggl(\frac{t}{\lambda_j}\biggr)^{k_j}\biggr\}.
\end{equation}

\begin{proof}
By Theorem \ref{thm:sys_reliability_function},
$$
R(t;\boldsymbol{\theta}) = \prod_{j=1}^{m} R_j(t;\lambda_j,k_j).
$$
Plugging in the Weibull component reliability functions obtains the result
\begin{align*}
R(t;\boldsymbol{\theta})
    &= \prod_{j=1}^{m} \exp\biggl\{-\biggl(\frac{t}{\lambda_j}\biggr)^{k_j}\biggr\}\\
    &= \exp\biggl\{-\sum_{j=1}^{m}\biggl(\frac{t}{\lambda_j}\biggr)^{k_j}\biggr\}.
\end{align*}
\end{proof}

The Weibull series system's hazard function is given by \begin{equation}
\label{eq:sys_weibull_failure_rate_function}
h(t;\boldsymbol{\theta}) =
    \sum_{j=1}^{m} \frac{k_j}{\lambda_j}\biggl(\frac{t}{\lambda_j}\biggr)^{k_j-1},
\end{equation} whose proof follows from Theorem
\ref{thm:sys_failure_rate}.

The pdf of the series system is given by \begin{equation}
\label{eq:sys_weibull_pdf}
f(t;\boldsymbol{\theta}) =
\biggl\{
    \sum_{j=1}^m \frac{k_j}{\lambda_j}\left(\frac{t}{\lambda_j}\right)^{k_j-1}
\biggr\}
\exp
\biggl\{
    -\sum_{j=1}^m \bigl(\frac{t}{\lambda_j}\bigr)^{k_j}
\biggr\}.
\end{equation}

\begin{proof}
By definition,
$$
f(t;\boldsymbol{\theta}) = h(t;\boldsymbol{\theta}) R(t;\boldsymbol{\theta}).
$$
Plugging in the failure rate and reliability functions given respectively by
Equations \eqref{eq:sys_weibull_reliability_function} and
\eqref{eq:sys_weibull_failure_rate_function} completes the proof.
\end{proof}

\hypertarget{weibull-likelihood-model-for-masked-data}{%
\subsection{Weibull Likelihood Model for Masked
Data}\label{weibull-likelihood-model-for-masked-data}}

In Section \ref{sec:like_model}, we discussed two separate kinds of
likelihood contributions, masked component cause of failure data (with
exact system failure times) and right-censored data. The likelihood
contribution of the \(i\)\textsuperscript{th} system is given by the
following theorem.

\begin{theorem}
Let $\delta_i$ be an indicator variable that is 1 if the
$i$\textsuperscript{th} system fails and 0 (right-censored) otherwise.
Then the likelihood contribution of the $i$\textsuperscript{th} system is given by
\begin{equation}
\label{eq:weibull_likelihood_contribution}
L_i(\boldsymbol{\theta}) =
\begin{cases}
    \exp\biggl\{-\sum_{j=1}^{m}\bigl(\frac{t_i}{\lambda_j}\bigr)^{k_j}\biggr\}
        \beta_i \sum_{j \in c_i} \frac{k_j}{\lambda_j}\bigl(\frac{t_i}{\lambda_j}\bigr)^{k_j-1}
    & \text{if } \delta_i = 1,\\
    \exp\bigl\{-\sum_{j=1}^{m}\bigl(\frac{t_i}{\lambda_j}\bigr)^{k_j}\biggr\} & \text{if } \delta_i = 0.
\end{cases}
\end{equation}
\end{theorem}
\begin{proof}
By Theorem \ref{thm:likelihood_contribution}, the likelihood contribution of the
$i$-th system is given by
$$
L_i(\boldsymbol{\theta}) =
\begin{cases}
    R_{T_i}(s_i;\boldsymbol{\theta})                      &\text{ if } \delta_i = 0\\
    \beta_i R_{T_i}(s_i;\boldsymbol{\theta})
        \sum_{j\in c_i} h_j(s_i;\boldsymbol{\theta_j})   &\text{ if } \delta_i = 1.
\end{cases}
$$
By Equation \eqref{eq:sys_weibull_reliability_function}, the system reliability
function $R_{T_i}$ is given by
$$
R_{T_i}(t_i;\boldsymbol{\theta}) = \exp\biggl\{-\sum_{j=1}^{m}\biggl(\frac{t_i}{\lambda_j}\biggr)^{k_j}\biggr\}.
$$
and by Equation \eqref{eq:weibull_haz}, the Weibull component hazard function $h_j$ is
given by
$$
h_j(t_i;\boldsymbol{\theta_j}) = \frac{k_j}{\lambda_j}\biggl(\frac{t_i}{\lambda_j}\biggr)^{k_j-1}.
$$
Plugging these into the likelihood contribution function obtains the result.
\end{proof}

Taking the log of the likelihood contribution function obtains the
following result.

\begin{corollary}
The log-likelihood contribution of the $i$-th system is given by
\begin{equation}
\label{eq:weibull_log_likelihood_contribution}
\ell_i(\boldsymbol{\theta}) =
-\sum_{j=1}^{m}\biggl(\frac{t_i}{\lambda_j}\biggr)^{k_j} +
    \delta_i \log \!\Biggl(    
        \sum_{j \in c_i} \frac{k_j}{\lambda_j}\biggl(\frac{t_i}{\lambda_j}\biggr)^{k_j-1}
    \Biggr)
\end{equation}
where we drop any terms that do not depend on $\boldsymbol{\theta}$ since they do not
affect the MLE.
\end{corollary}

We find an MLE by solving \eqref{eq:mle_eq}, i.e., a point
\(\boldsymbol{\hat\theta} = (\hat{k}_1,\hat{\lambda}_1,\ldots,\hat{k}_m,\hat{\lambda}_m)\)
satisfying
\(\nabla_{\theta} \ell(\boldsymbol{\hat\theta}) = \boldsymbol{0}\),
where \(\nabla_{\boldsymbol{\theta}}\) is the gradient of the
log-likelihood function (score) with respect to \(\boldsymbol{\theta}\).

To solve this system of equations, we use the Newton-Raphson method,
which requires the score and the Hessian of the log-likelihood function.
We analytically derive the score since it is useful to have for the
Newton-Raphson method, but we do not do the same for the Hessian of the
log-likelihood for the following reasons:

\begin{enumerate}
\def\labelenumi{\arabic{enumi}.}
\item
  The gradient is relatively easy to derive, and it is useful to have
  for computing gradients efficiently and accurately, which will be
  useful for numerically approximating the Hessian.
\item
  The Hessian is tedious and error prone to derive, and Newton-like
  methods often do not require the Hessian to be explicitly computed.
\end{enumerate}

The following theorem derives the score function.

\begin{theorem}
\label{thm:weibull_score}
The score function of the log-likelihood contribution of the $i$-th Weibull series
system is given by
\begin{equation}
\label{eq:weibull_score}
\nabla \ell_i(\boldsymbol{\theta}) = \biggl(
    \frac{\partial \ell_i(\boldsymbol{\theta})}{\partial k_1},
    \frac{\partial \ell_i(\boldsymbol{\theta})}{\partial \lambda_1},
    \cdots, 
    \frac{\partial \ell_i(\boldsymbol{\theta})}{\partial k_m},
    \frac{\partial \ell_i(\boldsymbol{\theta})}{\partial \lambda_m} \biggr)',
\end{equation}
where
\begin{equation}
\frac{\partial \ell_i(\boldsymbol{\theta})}{\partial k_r} = 
    -\biggl(\frac{t_i}{\lambda_r}\biggr)^{k_r}    
        \!\!\log\biggl(\frac{t_i}{\lambda_r}\biggr) +
        \frac{\frac{1}{t_i} \bigl(\frac{t_i}{\lambda_r}\bigr)^{k_r}
            \bigl(1+ k_r \log\bigl(\frac{t_i}{\lambda_r}\bigr)\bigr)}
            {\sum_{j \in c_i} \frac{k_j}{\lambda_j}\bigl(\frac{t_i}{\lambda_j}\bigr)^{k_j-1}}
        1_{\delta_i = 1 \land r \in c_i}
\end{equation}
and 
\begin{equation}
\frac{\partial \ell_i(\boldsymbol{\theta})}{\partial \lambda_r} = 
    \frac{k_r}{\lambda_r} \biggl(\frac{t_i}{\lambda_r}\biggr)^{k_r} -
    \frac{
        \bigl(\frac{k_r}{\lambda_r}\bigr)^2 \bigl(\frac{t_i}{\lambda_r}\bigr)^{k_r - 1}
    }
    {
        \sum_{j \in c_i} \frac{k_j}{\lambda_j}\bigl(\frac{t_i}{\lambda_j}\bigr)^{k_j-1}
    }
    1_{\delta_i = 1 \land r \in c_i}
\end{equation}
\end{theorem}

The result follows from taking the partial derivatives of the
log-likelihood contribution of the \(i\)-th system given by Equation
\eqref{eq:weibull_likelihood_contribution}. It is a tedious calculation
so the proof has been omitted, but the result has been verified by using
a very precise numerical approximation of the gradient.

By the linearity of differentiation, the gradient of a sum of functions
is the sum of their gradients, and so the score function conditioned on
the entire sample is given by \begin{equation}
\label{eq:weibull_series_score}
\nabla \ell(\boldsymbol{\theta}) = \sum_{i=1}^n \nabla \ell_i(\boldsymbol{\theta}).
\end{equation}

\hypertarget{simstudy}{%
\section{Simulation Study}\label{simstudy}}

We derived the likelihood model for masked data for the Weibull series
system in Section \ref{sec:weibull}. In this section, we describe the
design of our simulation study, and we assess the performance of the
bootstrap method for estimating the sampling distribution of the MLE for
a Weibull series system with \(m=5\) components under our proposed
likelihood model.

\hypertarget{sec:reliability}{%
\subsection{Realistic System Designs}\label{sec:reliability}}

A series system is only as reliable as its least reliable component. In
order to make the simulation study representative of real-world
scenarios, at least for systems designed to be reliable, we choose
parameter values that are representative of real-world systems where
there is no single component that is much less reliable than the others.

We consider two definitions of reliability, the mean time to failure
(MTTF) and the probability of component cause of failure. The MTTF for
the \(j\)\textsuperscript{th} component is given by its expected value,
\[
\text{MTTF}_j = E(T_{i j})
\] which for the Weibull distribution has the closed-form expression \[
\text{MTTF}_j = \lambda_j \, \Gamma(1 + 1/k_j),
\] where \(\Gamma\) is the gamma function. The probability that the
\(j\)\textsuperscript{th} component is the cause of failure for the
\(i\)\textsuperscript{th} series system is the probability that a
component is the cause of failure of the series system, which is given
by \[
\Pr\{K_i = j\} = \int_{0}^\infty f_{T_i, K_i}(t, j ; \boldsymbol{\theta}) dt.
\]

We consider the data from \citep{Huairu-2013}, which includes a study of
the reliability of a series system with three Weibull components with
shape and scale parameters given by \begin{equation}
\begin{aligned}
    k_1 = 1.2576 &\quad \lambda_1 = 994.3661\\
    k_2 = 1.1635 &\quad \lambda_2 = 908.9458\\
    k_3 = 1.1308 &\quad \lambda_3 = 840.1141.
\end{aligned}
\end{equation}

Our approach is to extend this system to a five component system by
adding two more components with shape and scale parameters given by
\begin{equation}
\begin{aligned}
    k_4 = 1.1802 &\quad \lambda_4 = 940.1342\\
    k_5 = 1.2034 &\quad \lambda_5 = 923.1631.
\end{aligned}
\end{equation}

\begin{longtable}[]{@{}lrr@{}}
\caption{Mean Time To Failure (MTTF) and Probability of Component
Failure of Weibull Components in Series Configuration}\tabularnewline
\toprule
& MTTF & Failure Probability\tabularnewline
\midrule
\endfirsthead
\toprule
& MTTF & Failure Probability\tabularnewline
\midrule
\endhead
Component 1 & 924.8693 & 0.1685628\tabularnewline
Component 2 & 862.1568 & 0.2069114\tabularnewline
Component 3 & 803.5639 & 0.2337547\tabularnewline
Component 4 & 888.2370 & 0.1955561\tabularnewline
Component 5 & 867.7484 & 0.1952150\tabularnewline
Series System & 222.8836 & NA\tabularnewline
\bottomrule
\end{longtable}

As shown by Table 2, there are no components that are significantly less
reliable than any of the others. Note that a series system in which,
say, one of the components does have a significantly shorter MTTF would
also pose significant challenges to estimating the parameters of the
system from our masked failure data, since the failure time of the
series system would be dominated by the failure time of the least
reliable component. See Section \ref{sec:opt_rescale} for further
discussion.

\hypertarget{verification}{%
\subsubsection*{Verification}\label{verification}}
\addcontentsline{toc}{subsubsection}{Verification}

To verify that our likelihood model is correct, we load the Table 2 data
from \citep{Huairu-2013} and fit the Weibull series model to the data to
see if we can recover the MLE they reported. When we fit the Weibull
series model to this data by maximizing the likelihood function, we
obtain the following fit for the shape and scale parameters given
respectively by \[
    \hat{k}_1 = 1.2576,
    \hat{k}_2 = 1.1635,
    \hat{k}_3 = 1.1308,
\] and \[
    \hat{\lambda}_1 = 994.3661,
    \hat{\lambda}_2 = 908.9458,
    \hat{\lambda}_3 = 840.1141,
\] which is in agreement with the MLE they reported. Satisfied that our
likelihood model is correct, we proceed with the simulation study.

\hypertarget{data-generating-process}{%
\subsection{Data Generating Process}\label{data-generating-process}}

In this section, we describe the data generating process for our
simulation study. It consists of three parts: the series system, the
candidate set model, and the right-censoring model.

\hypertarget{weibull-series-system-lifetime}{%
\subsubsection*{Weibull Series System
Lifetime}\label{weibull-series-system-lifetime}}
\addcontentsline{toc}{subsubsection}{Weibull Series System Lifetime}

We generate data from a Weibull series system with \(m=5\) components.
As described in Section \ref{sec:weibull}, the \(j\)\textsuperscript{th}
component of the \(i\)\textsuperscript{th} system has a lifetime
distribution given by \[
    T_{i j} \sim \operatorname{WEI}(k_j, \lambda_j)
\] and the lifetime of the series system composed of \(m\) Weibull
components is defined as \[
    T_i = \min\{T_{i 1}, \ldots, T_{i m}\}.
\]

To generate a data set, we first generate the \(m\) component failure
times, by efficiently sampling from their respective distributions, and
we then set the failure time \(t_i\) of the system to the minimum of the
component failure times.

\hypertarget{right-censoring-model}{%
\subsubsection*{Right-Censoring Model}\label{right-censoring-model}}
\addcontentsline{toc}{subsubsection}{Right-Censoring Model}

We employ a very simple right-censoring model, where the right-censoring
time \(\tau\) is fixed at some known value, e.g., an experiment is run
for a fixed amount of time \(\tau\), and all systems that have not
failed by the end of the experiment are right-censored. The censoring
time \(S_i\) of the \(i\)\textsuperscript{th} system is thus given by \[
    S_i = \min\{T_i, \tau\}.
\] So, after we generate the system failure time \(T_i\), we generate
the censoring time \(S_i\) by taking the minimum of \(T_i\) and
\(\tau\).

\hypertarget{masking-model-for-component-cause-of-failure}{%
\subsubsection*{Masking Model for Component Cause of
Failure}\label{masking-model-for-component-cause-of-failure}}
\addcontentsline{toc}{subsubsection}{Masking Model for Component Cause
of Failure}

We must generate data that satisfies the masking conditions described in
Section \ref{sec:candmod}. There are many ways to satisfying the masking
conditions. We choose the simplest method, which we call the
\emph{Bernoulli candidate set model}. In this model, each non-failed
component is included in the candidate set with a fixed probability
\(p\), independently of all other components and independently of
\(\boldsymbol{\theta}\), and the failed component is always included in
the candidate set.

\hypertarget{sec:opt_rescale}{%
\subsection{Issues with Convergence to the MLE}\label{sec:opt_rescale}}

The surface of the log-likelihood function can be quite complex with
many local maxima and ridges. This makes it difficult to find the MLE
using local search methods like Newton-Raphson. In this section, we
discuss some of the issues we encountered when estimating the MLE.

Since this is a simulation study, we knew the true parameter value
\(\boldsymbol{\theta}\) and so started the optimization routine at the
true parameter value. This is not the case in real-world scenarios,
where the true parameter value is unknown, and so the optimization
routine must start at some initial guess. We would have encountered even
more issues if we had started the optimization routine at a poor initial
guess.

\hypertarget{identifiability}{%
\subsubsection*{Identifiability}\label{identifiability}}
\addcontentsline{toc}{subsubsection}{Identifiability}

When estimating the parameters, it may sometimes be the case that the
likelihood function is not maximized at a unique point. This is known as
the \emph{identifiability} problem. If the likelihood function is not
maximized at a unique point, then a lot of the theory we have developed
so far breaks down \citep{mclachlan2007algorithm}.

In our case, since we are estimating the parameters of latent
components, identifiability is not guaranteed. We consider two examples
where this might occur, but there are many more.

\begin{enumerate}
\def\labelenumi{\arabic{enumi}.}
\item
  The candidate sets could have been constructed in a way that prevents
  us from distinguishing between some of the components. For example, if
  the candidate sets in a sample have the characteristic that component
  \(1\) is in a candidate set if and only if component \(2\) is in a
  candidate set, then we do not have enough information to estimate the
  parameters of component \(1\) and component \(2\) separately. This
  could happen, for instance, if the failure analysis is done by a
  human, and he or she is only able to identity that a larger component
  failed, but not which of the smaller components inside it had failed.
  In this case, we may want to combine the two components into one
  component, and estimate the parameters of the combined component.

  In our Bernoulli candidate set model, this is something that can arise
  only by chance, and is unlikely to occur in practice, particularly for
  reasonably large samples.
\item
  The series system has a component that is the least reliable by a
  significant margin and is most likely the component cause of failure
  and is in every candidate set. In this case, our data may not be
  informative enough to estimate the parameters of the other components.
\end{enumerate}

\begin{figure}

{\centering \includegraphics{image/test-flat-likelihood} 

}

\caption{Log-likelihood Profile vs Shape Parameter for Component 1 ($k_1$): Non-unique MLE vs Unique MLE.
Blue Dashed Line is the True Shape $k_1$.}\label{fig:flat-loglike-prof}
\end{figure}

We constructed a quick experiment to demonstrate (2) above. In this
experiment, we performed the following steps:

\begin{enumerate}
\def\labelenumi{\arabic{enumi}.}
\item
  We use the the series system from \citep{Huairu-2013} as a base, but
  tweak it slightly to design the MTTF of the last component (component
  3) be be two orders of magnitude smaller than the others. We did this
  by changing its scale parameter to \(\lambda_3 = 4.1141\).
\item
  Generated a data set of size \(n = 30\) from this system with a
  right-censoring time of \(\tau = 6.706782\), corresponding to the
  \(82.5\%\) quantile of the system's lifetime, and with a masking
  probability \(p = 0.215\) using the Bernoulli candidate set model.
\item
  We found an MLE by maximizing the log-likelihood function with the
  data set generated in step 2. As shown in the left plot in Figure
  \ref{fig:flat-loglike-prof}, the log-likelihood function is flat, and
  therefore there there is no unique MLE. It appears any value of
  \(\hat k_1\) larger than \(3\) will maximize the log-likelihood
  function.

  For a possible reason, we see that \emph{every} candidate set in the
  data set contains component 3 because it was the component cause of
  failure in every system failure due to its significantly shorter MTTF.
\item
  We tweaked the data set in step 2 by removing component 3 from the
  candidate set in the first observation and adding component 1. This is
  permissible because the observation was not right-censored, and so
  must have a non-empty candidate set. This is a very small change to
  the data set, but as shown in the right plot in Figure
  \ref{fig:flat-loglike-prof}, the log-likelihood function is no longer
  flat, and there is a unique MLE. We also see that it is a much better
  estimate of the true parameter value for \(k_1\), although we did not
  do the analysis to assess whether this generally holds.
\end{enumerate}

See \hyperref[app:flat-like-code]{Appendix D} for the R code used to
generate the data set for this experiment.

According to this experiment, one could potentially justify either
excluding these data sets from the analysis, or tweak them slightly as
we had done, since otherwise they do not provide a unique MLE. In our
simulation study, we mitgated these issues by choosing parameter values
that are representative of real-world systems where there is no single
component that is much less reliable than the others. We also use the
Bernoulli candidate set model, which is unlikely to produce candidate
sets that are not informative enough to estimate the parameters of the
components, unless of course the masking probability \(p\) is very
large.

After taking these precautions, we largely ignored identifiability
issues in our simulation study, with the exception that we discarded any
data sets that did not converge to a solution after 150
iterations.\footnote{The choice of $150$ iterations was driven by the computational demands of the
simulation study combined with the subsequent bootstrapping of the confidence intervals.
} A log-likelihood function that is flat can cause our convergence
criteria to take a long time to reach a solution. Therefore, a failure
to converge within 150 iterations could be seen as evidence of potential
identifiability issues.

Nonetheless, such scenarios occurred infrequently. During the
bootstrapping of confidence intervals, we included all MLEs, even those
that did not converge. This worst-case analysis approach was adopted
because our main objective was to assess the performance of the BCa
confidence intervals. We were concerned that if we took any additional
steps, we may unintentionally bias the results in favor of producing
narrow BCa confidence intervals with good coverage probabilities.

\hypertarget{parameter-rescaling}{%
\subsubsection*{Parameter rescaling}\label{parameter-rescaling}}
\addcontentsline{toc}{subsubsection}{Parameter rescaling}

When the parameters under investigation span different orders of
magnitude, parameter rescaling can significantly improve the performance
and reliability of optimization algorithms. Parameter rescaling gives an
optimizer a sense of the typical size of each parameter, enabling it to
adjust its steps accordingly. This is crucial in scenarios like ours,
where shape and scale parametes are a few orders of magnitude apart.
Without rescaling, the optimization routine may struggle, taking
numerous small steps for larger parameters and overshooting for smaller
ones. For more information, see \citep{nocedal2006numerical}.

Speed of convergence was particularly important in our case, since in
our simulation study, we employ the bootstrap method to estimate the
sampling distribution of the MLE, which requires us to estimate the MLE
for many data sets. We found that parameter rescaling significantly
improved the speed of convergence, which allowed us to run our
simulation study in a tractable amount of time.

\hypertarget{sec:effect-censoring}{%
\subsection{Effect of Right-Censoring on the
MLE}\label{sec:effect-censoring}}

In all of our simulation studies, we use a fixed right-censoring time
\(\tau = 377.71\), which is the \(82.5\%\) quantile of the series
system. This means that \(82.5\%\) of the series systems are expected to
fail before time \(\tau\) and \(17.5\%\) of the series are expected to
be right-censored. This represents a situation in which an experiment is
run for a fixed amount of time \(\tau\), and all systems that have not
failed by the end of the experiment are right-censored.

Right-censoring introduces a source of bias in the MLE. Right-censoring
has the effect of pushing the MLE to estimate a larger MTTF for each of
the components, so that the series system has a larger MTTF. This is
because when we observe a right-censoring event, we know that the system
failed after the censoring time, but we do not know precisely when it
will fail. This uncertainty has the effect of pushing the MLE to
estimate a larger MTTF for the system so that it is more likely to fail
after the censoring time. See \cite{klein2005survival} for more
information on this phenomenon.

To increase the MTTF of a series system, the MTTF of each component is
increased. The mean time to failure (MTTF) for the
\(j\)\textsuperscript{th} component in a Weibull distribution is given
by \begin{equation}
\label{eq:mttf-wei}
\text{MTTF} = \lambda_j \Gamma(1 + 1/k_j),
\end{equation} therefore, in order to increase the MTTF of the
components, lower values for the shape parameters are chosen and higher
values for the scale parameters are chosen.

\hypertarget{sec:effect-masking}{%
\subsection{Effect of Masking the Component Cause of Failure on the
MLE}\label{sec:effect-masking}}

When we observe a system failure, we know that one of the components in
the candidate set caused the system to fail, but we do not know which
one. This uncertainty has the effect of pushing the MLE to estimate a
smaller MTTF for each of the components in the candidate set. For
components that are frequently in candidate sets but proportionally not
more likely to be a component cause of failure, the effect is more
pronounced, which may introduce a source of bias in the MLE for such
components.

In our Bernoulli candidate set model, the masking probability \(p\)
determines how commonly each non-failed component is in the candidate
set, and so we expect that as \(p\) increases, this will become a more
pronoucned source of
bias.\footnote{In a more complicated candidate set model, it is possible
that masking could introduce a significant source of bias for some components, and none at all
for others.} However, note that the effect of masking, which pushes the
MLE to estimate a smaller MTTF, has opposite effect to that of
right-censoring, which pushes the MLE to estimate a larger MTTF. As
these two sources of bias compete with each other, it is not clear which
one will dominate

In what follows, we explain how the bias induced by masking the
component cause of failure effects the MLE for the shape and scale
parameters of a Weibull component. Assessing Equation
\eqref{eq:mttf-wei}, we see that the MTTF of a Weibull component is
proportional to its scale parameter \(\lambda_j\), which means when we
decrease the scale parameter \(\lambda_j\) (keeping the shape parameter
\(k_j\) constant), the MTTF decreases. Therefore, if the
\(j\)\textsuperscript{th} component is in the candidate set, to make it
more likely to appear in the candidate set, its scale parameter should
be decreased, potentially biasing the MLE for the scale parameter
downwards.

Conversely, we see that the MTTF decreases as we increase the shape
parameter \(k_j\). Therefore, if the \(j\)\textsuperscript{th} component
is in the candidate set, to make it more likely to appear in in the
candidate set, its shape parameter should be increased, potentially
biasing the MLE for the shape parameter upwards.

\hypertarget{assessing-the-bootstrapped-confidence-intervals}{%
\subsection{Assessing the Bootstrapped Confidence
Intervals}\label{assessing-the-bootstrapped-confidence-intervals}}

Our primary interest is in assessing the performance of the BCa
confidence intervals for the MLE. We will assess the performance of the
BCa confidence intervals by computing the coverage probability of the
confidence intervals. Under a variety of scenarios, we will bootstrap a
\(95\%\)-confidence interval for \(\boldsymbol{\theta}\) using the BCa
method, and we will evaluate its calibration by computing the coverage
probability and its precision by assessing the width of the confidence
interval.

The coverage probability is defined as the proportion of times that the
true value of \(\boldsymbol{\theta}\) falls within the confidence
interval. We will compute the coverage probability by generating \(R\)
datasets from the Data Generating Process (DGP) and computing the
coverage probability for each dataset. We will then aggregate this
information across all \(R\) datasets to estimate the coverage
probability.

\hypertarget{simulation-scenarios}{%
\subsection{Simulation Scenarios}\label{simulation-scenarios}}

We parameterize \(\tau\) by quantiles of the series system, e.g., if
\(q = 0.8\), then \(\tau(0.8)\) is the \(80\%\) quantile of the series
system such that \(80\%\) of the sytems are expected to fail before time
\(\tau(0.8)\) and \(20\%\) of the series systems are expected to be
right-censored.

We define a simulation scenario to be some comination of \(n\) and
\(p\). We are interested in choosing a small number of scenarios that
are representative of real-world scenarios and that are interesting to
analyze.

Here is an outline of the simulation study analysis:

\begin{enumerate}
\def\labelenumi{\arabic{enumi}.}
\item
  Choose a scenario (sample size \(n\), masking probability \(p\)
  (\(\tau\) fixed).
\item
  Generate R datasets from the Data Generating Process (DGP). The DGP
  should be compatible with the assumptions in our likelihood model. In
  our case, we use:

  \begin{itemize}
  \item
    Right-censored series lifetimes with \(m = 5\) Weibull components.
  \item
    Mmasking component cause of failure using Bernoulli candidate set
    model.
  \end{itemize}
\item
  For each of these \(R\) datasets, calculate the Maximum Likelihood
  Estimator (MLE).
\item
  For each of these \(R\) datasets, perform bootstrap resampling \(B\)
  times to create a set of bootstrap samples.
\item
  Calculate the MLE for each of these bootstrap samples. This generates
  an empirical distribution of the MLE, which is used to construct a
  confidence interval for the MLE.
\item
  Repeat steps 4 and 5 for each of the \(R\) datasets.
\item
  For each dataset, determine whether the true parameter value falls
  within the computed CI. Aggregate this information across all \(R\)
  datasets to estimate the coverage probability of the CI.
\item
  Interpret the results and discuss the performance of the MLE estimator
  under various scenarios.
\end{enumerate}

For how we generate a scenario, see Appendix A.

\hypertarget{coverage-probability-vs-sample-size}{%
\subsection{Coverage Probability vs Sample
Size}\label{coverage-probability-vs-sample-size}}

In the simulation study, we have generated many different synthetic
samples of different sizes (\(n\)) from a data generating process (DGP)
that is compatible with the assumptions our likelihood model makes about
the data. In particular, right-censored series system lifetimes with a
fixed right-censoring time for the system and five components with
Weibull lifetimes, each with a different shape and scale parameter. For
each observation, we then mask the component cause of failure with
candidate sets that satisfy the three primary conditions of the
likelihood model, e.g., the failed component is always in the candidate
set. For each synthetic data set, we then compute the MLEs of the shape
and scale parameters of the Weibull distribution. We then use the MLEs
to compute the BCa bootstrapped 95\% CIs.

In what follows, we analyze the performance of the BCa bootstrapped CIs
for the shape and scale parameters under different masking conditions
(\(p\)) for the component cause of failure. We will focus on the
following statistics:

\begin{itemize}
\item
  \emph{Coverage Probability (CP)}: The CP is the proportion of the
  bootstrapped CIs that contain the true value of the parameter. The CP
  is a good indicator of the reliability of the estimates as previously
  discussed.
\item
  \emph{Dispersion of MLEs}: The shaded regions representing the 95\%
  probability range of the MLEs get narrower as the sample size
  increases. This is an indicator of the increased precision in the
  estimates as more data is available. We call it a \emph{Confidence
  Band}, but it is actually an estimate of the quantile range of the
  MLEs. The shaded region provides insight into the distribution of the
  MLEs.
\item
  \emph{IQR of Bootstrapped CIs}: The vertical blue bars represent the
  Interquartile Range (IQR) of the actual bootstrapped Confidence
  Intervals (CIs). Since in practice we only have one sample and
  consequently one MLE, we use bootstrapping to resample and compute
  multiple CIs. The IQR then represents the middle 50\% range of these
  bootstrapped CIs.
\item
  \emph{Mean of the MLEs}: The mean of the MLEs is a good indicator of
  the bias in the estimates. If the mean of the MLEs is close to the
  true value, then the MLEs are, on average, unbiased.
\end{itemize}

The distinction between the shaded region (95\% range of MLEs) and the
blue vertical bars (IQR of bootstrapped CIs) is important. The shaded
region provides insight into the distribution of the MLEs, whereas the
blue vertical bars provide information about the variation in the
bootstrapped CIs. Both are relevant for understanding the behavior of
the estimations.

\hypertarget{scale-parameters}{%
\subsubsection*{Scale Parameters}\label{scale-parameters}}
\addcontentsline{toc}{subsubsection}{Scale Parameters}

\begin{figure}

{\centering \includegraphics{image/plot-n-vs-stats-p215-scale} 

}

\caption{Sample Size vs Bootstrapped Scale CI Statistics (p = 0.215)}\label{fig:samp-size-n-vs-stats-p215-scale}
\end{figure}

Figure \ref{fig:samp-size-n-vs-stats-p215-scale} shows the distribution
of the MLEs for the shape parameters of the first three components and
the bootstrapped CIs for different sample sizes with a component cause
of failrue masking probaility of \(p = 0.215\) (each non-failed
component is in the candidate set with a \(21.5\%\) probabiltiy).

The distinction between the shaded region (95\% range of MLEs) and the
blue vertical bars (IQR of bootstrapped CIs) is important. The shaded
region provides insight into the distribution of the MLEs, whereas the
blue vertical bars provide information about the variation in the
bootstrapped CIs. Both are relevant for understanding the behavior of
the estimations. Here are several key observations:

\begin{itemize}
\item
  \emph{Coverage Probability (CP)}: The CP is well-calibrated, obtaining
  a value near the nominal 95\% level across different sample sizes.
  This suggests that the bootstrapped CIs will contain the true value of
  the shape parameter with the specified confidence level. The CIs are
  neither too wide nor too narrow.
\item
  \emph{Dispersion of MLEs}: The shaded regions representing the 95\%
  probability range of the MLEs get narrower as the sample size
  increases. This is an indicator of the increased precision in the
  estimates as more data is available.
\item
  \emph{IQR of Bootstrapped CIs}: The IQR (vertical blue bars) reduces
  with an increase in sample size. This suggests that the bootstrapped
  CIs are getting more consistent and focused around a narrower range
  with larger samples while maintaining a good coverage probability. As
  we get more data, the bootstrapped CIs are more likely to be closer to
  each other and the true value of the scale parameter. For small sample
  sizes, they are quite large, but to maintain well-calibrated CIs, this
  was necessary. The estimator is quite sensitive to the data, and so
  the bootstrapped CIs are quite wide to account for this sensitivity
  when the sample size is small and not necessarily representative of
  the true distribution.
\item
  \emph{Mean of MLEs}: The red dashed line indicating the mean of MLEs
  remains stable across different sample sizes and close to the true
  value, suggesting that the MLEs are, on average, reasonably unbiased.
\end{itemize}

\hypertarget{sec:p-vs-mttf}{%
\subsection{Masking Probability for Component Cause of
Failure}\label{sec:p-vs-mttf}}

In this scenario, we fix the sample size at \(n = 90\) and we fix the
right-censoring quantile fixed \(q = .825\), and we vary the masking
probability \(p\) from \(p = 0\) (no masking the component cause of
failure) to \(p = 0.45\) (significant masking of the component cause of
failure). simualted

\hypertarget{scale-parameter}{%
\subsubsection*{Scale Parameter}\label{scale-parameter}}
\addcontentsline{toc}{subsubsection}{Scale Parameter}

In Figures \ref{fig:masking-prob-vs-stats-scale}, we show the effect of
the masking probability \(p\) on the MLE and the bootstrapped BCa
confidence intervals for the scale parameters. A this relatively sample
size, we see that the MLE is relatively unbiased for small \(p\), but as
\(p\) increases, the MLE becomes increasingly biased. We also see the
confidence interval width seems relatively stable until the masking
becomes significant, at \(p = 0.45\). The confidence intervals appear to
be well-calibrated at all masking probabilities, even for \(p = 0.45\),
although it exhibits the worst coverage at around \(90\%\) for 95\%
confidence intervals.

\hypertarget{shape-parameter}{%
\subsubsection*{Shape Parameter}\label{shape-parameter}}
\addcontentsline{toc}{subsubsection}{Shape Parameter}

In Figures \ref{fig:masking-prob-vs-stats-shape}, we show the effect of
the masking probability \(p\) on the shape parameters. Here, we see that
as we increase the masking probability, the confidence interval widths
increase fairly significantly, as does the bias. In other words, the
shape parameters appear to be more sensitive to masking.

Overall, for both parameter types, we see that as the masking
probability increases, the IQR of the bootstrapped CIs, the dispersion
of the MLEs,a the bias increases, which indicates that the masking
probability effects the precision and accuracy of the estimates. As the
masking probability increases, we have less certainty about the
component cause of failure, and thus less certainty about the estimates
for the component parameters.

\begin{figure}

{\centering \includegraphics{image/plot-p-vs-stats-scale-n90} 

}

\caption{Component Cause of Failure Masking (p) vs Scale CI Statistics}\label{fig:masking-prob-vs-stats-scale}
\end{figure}

\begin{figure}

{\centering \includegraphics{image/plot-p-vs-stats-shapes-n90} 

}

\caption{Component Cause of Failure Masking (p) vs Shape CI Statistics}\label{fig:masking-prob-vs-stats-shape}
\end{figure}

\hypertarget{sec:mttf-c3}{%
\subsection{Manipulating the MTTF of Component 3}\label{sec:mttf-c3}}

In Section \ref{?}, we discussed that if one of the components had a
significantly lower MTTF, this could pose challenges to accurately
estimating the parameters of the other components. In this section, we
will explore this phenomenon in more detail by manipulating the MTTF of
component 3 and observing the effect it has on the MLE and the
bootstrapped confidence intervals for component 3 and component
1.\footnote{Since the other components had a similiar MTTF, we
will arbitrarily choose component 1 to represent the other components.}

\begin{figure}

{\centering \includegraphics{image/plot-scale3-vs-stats} 

}

\caption{MTTF vs Parameter Statistics}\label{fig:mttf-vs-ci}
\end{figure}

In Figure \ref{fig:mttf-vs-ci}, we show the effect of the MTTF of
component 3 on the MLE and the bootstrapped confidence intervals for the
shape and scale parameters for components 1 and 3 (the component we are
varying). We simulate samples with a sample size of \(n = 100\), a
right-censoring quantile of \(q = 0.825\), and a masking probability of
\(p = 0.215\). (Note that while \(q\) is fixed, \(\tau\) varies as we
change the MTTF of component 3.) The MTTF of component 3 varies from
around \(300\) to \(1500\) and MTTF of the other components, including
component 2, is around \(900\). There are several interesting
observations that we can make about Figure \ref{fig:mttf-vs-ci}:

\begin{enumerate}
\def\labelenumi{\arabic{enumi}.}
\item
  When the MTTF of component 3 is much smaller than the other
  components, the estimate of parameters of component 3 is precise
  (narrow CIs with high Probability coverage) and accurate (the MLE is
  close to the true value). This is because component 3 is the component
  cause of failure in nearly every system failure, and so the data is
  very informative about the parameters of component 3. Conversely, the
  estimates of the parameters of the other components is quite poor,
  with wide CIs and large positive bias. Nonetheless, the coverage
  probability of the CIs for the other components is still
  well-calibrated, which means that the CIs will contain the true value
  of the parameter with a probability around the specified confidence
  level. So, while we may not have a good point estimates for the
  parameters, we can still be confident that CIs contain them. That is
  to say, we have properly quantified our uncertainty about the
  parameters of the other components.
\item
  When the MTTF of component 3 is much larger than the MTTF of the other
  components, then component 3 is much less likely to be the component
  cause of failure, and with a masking probability of \(p = 0.215\), it
  will be in the candidate set with approximately \(21.5\%\)
  probability, but it will generally be a false candidate. The end
  result is that the estimates of the parameters of component 3 are
  quite poor, with wide CIs and large positive bias. However, the
  estimates of the parameters of the other components are quite good,
  with narrow CIs and small positive bias. The coverage probability of
  the CIs for the other components are, in comparison, quite good. As
  the MTTF of component 3 increases and it becomes less likely to be the
  component cause of failure, the estimates of the parameters of the
  other components become more precise and accurate.

  We also see that the bias is positive for both parameters of component
  3. We had not necessarily expected this, but we knew there would be a
  complex relationship given the presence of right-censoring and
  masking. When a system is right-censored, or the exact time of failure
  is observed but the component cause of failure is masked and component
  3 is not in the candidate set, then to make component 3 more likely to
  not be the component cause of failure, its failure rate at that
  observed time is pushed down and its MTTF is pushed to the right by
  the MLE. Thus, \(\hat\lambda_3\) being positively biased is expected.
  However, \(k_3\) being positively biased is not necessarily expected,
  but the fact is, decreasing \(k_3\) only has a small impact on the
  MTTF compared to the scale parameter \(\lambda_3\), and the shape
  parameter may be more particular about when the failures occur. For
  example, if the shape parameter is large, then the failures may be
  more likely to occur at the beginning of the lifetime, which would
  cause the MTTF to be pushed to the right. This is

  anticipated this: from our preliminary analysis, we had expected that
  the bias would be positive for the scale parameter and negative for
  the shape parameter. We believed this because if component 3 is not
  the component cause of failure, then the system is more likely to fail
  due to the failure of one of the other components, which would cause
  the system to fail sooner. This would cause
\end{enumerate}

\hypertarget{adding-fake-data-implcit-prior-beliefs}{%
\subsection{Adding Fake Data (Implcit Prior
Beliefs)}\label{adding-fake-data-implcit-prior-beliefs}}

In the identifiability section, 7.3.1, we discussed how the likelihood
function may not be maximized at a unique point. In this section, we
will explore this phenomenon in more detail.

We are going to add fake data as a sort of implicit prior to the
likelihood function.

Notice that the likelihood function is flat and any value greater than
1.5 or so could have been chosen as the MLE. If we add fake data, as we
had done previously, then we can ``bias'' the MLE and also make it
unique.

How will we add data?

\begin{enumerate}
\def\labelenumi{\arabic{enumi}.}
\item
  We could have an initial estimate of a component's MTTF, and we could
  add data that is consistent with that estimate. For example, if we
  have an initial estimate of the MTTF of component 1, then we could add
  \(t_1 = \text{MTTF}\), \(delta_0 = 1\), and \(\mathcal{C}_0 = \{1\}\).
  We could do likewise for other components if we have prior data.
\item
  We could have the prior assumption that
  \(\mathcal{C}_i = c_i | T_i = t_i, K_i = j\) is equally probable for
  every component \(j\) in \(c_i\). So, another way to construct a prior
  consistent with this assumption is to look at all the candidate sets
  in the sample that contain component \(j\), and then add data where we
  assume it is the component cause of failure, estimate the component
  parameters given this assumption, and then sample data from it. The
  larger the sample, the stronger this prior.
\item
  We could do something very simple, like take the average of the system
  lifetimes where it is in the candidate set, and then add that to the
  sample, e.g., suppose the average failure time of the system when
  component \(j\) is in the candidate set is \(\bar{t}_j\), then we
  could add \(t_0 = \bar{t}_j\), \(delta_0 = 1\), and
  \(\mathcal{C}_0 = \{j\}\). This has the nice feature that it mostly
  lets the data speak for itself, and only adds a bit of prior
  information, and this prior is related (but not equivalent) to
  Condition 2.
\end{enumerate}

\hypertarget{future-work}{%
\section{Future Work}\label{future-work}}

\hypertarget{parametric-bootstrap}{%
\subsection{Parametric Bootstrap}\label{parametric-bootstrap}}

There is an alternative form of the bootstrap called the parametric
bootstap, where the bootstrap samples are generated from a parametric
distribution. However, this parametric bootstrap is not appropriate for
our analysis because we do not assume a parametric form for the
distribution of the candidate sets \(\mathcal{C}_i\).

However, it would be possible to employ a semi-parametric bootstrap,
where the bootstrap samples are generated from a parametric distribution
for the system lifetimes \(T_i\) and a non-parametric distribution for
the candidate sets \(\mathcal{C}_i | T_i, K_i\), e.g., the empirical
distribution with some discretization of \(T_i\) used.

\hypertarget{conclusion}{%
\section{Conclusion}\label{conclusion}}

We have developed a likelihood model for series systems with latent
components and right-censoring. We have provided evidence that, as long
as certain regularity conditions are met, the MLE is asymptotically
unbiased and consistent.

\begin{quote}
Repeat earlier results about how the masking probability effects the MLE
and its bootstrapped CIs and the explanation why. Do the same for the
sample size.
\end{quote}

\hypertarget{appendix-a-r-code-for-log-likelihood-function}{%
\section*{Appendix A: R Code For Log-likelihood
Function}\label{appendix-a-r-code-for-log-likelihood-function}}
\addcontentsline{toc}{section}{Appendix A: R Code For Log-likelihood
Function}

\label{app:loglike-code}

The following code is the log-likelihood function for the Weibull series
system with a likelihood model that includes masked component cause of
failure and right-censoring. It is implemented in the R library
\texttt{wei.series.md.c1.c2.c3} and is available on
\href{https://github.com/queelius/wei.series.md.c1.c2.c3}{GitHub}.

For clarity and brevity, we removed some of the functionality that is
not relevant to the analysis in this paper.

\begin{Shaded}
\begin{Highlighting}[]
\CommentTok{\#\textquotesingle{} Generates a log{-}likelihood function for a Weibull series system with respect}
\CommentTok{\#\textquotesingle{} to parameter \textasciigrave{}theta\textasciigrave{} (shape, scale) for masked data with candidate sets}
\CommentTok{\#\textquotesingle{} that satisfy conditions C1, C2, and C3 and right{-}censored data.}
\CommentTok{\#\textquotesingle{}}
\CommentTok{\#\textquotesingle{} @param df (masked) data frame}
\CommentTok{\#\textquotesingle{} @param theta parameter vector (shape1, scale1, ..., shapem, scalem)}
\CommentTok{\#\textquotesingle{} @returns Log{-}likelihood with respect to \textasciigrave{}theta\textasciigrave{} given \textasciigrave{}df\textasciigrave{}}
\NormalTok{loglik\_wei\_series\_md\_c1\_c2\_c3 \textless{}{-}}\StringTok{ }\ControlFlowTok{function}\NormalTok{(df, theta) \{}
\NormalTok{    n \textless{}{-}}\StringTok{ }\KeywordTok{nrow}\NormalTok{(df)}
\NormalTok{    C \textless{}{-}}\StringTok{ }\KeywordTok{md\_decode\_matrix}\NormalTok{(df, candset)}
\NormalTok{    m \textless{}{-}}\StringTok{ }\KeywordTok{ncol}\NormalTok{(C)}
\NormalTok{    delta \textless{}{-}}\StringTok{ }\NormalTok{df[[right\_censoring\_indicator]]}
\NormalTok{    t \textless{}{-}}\StringTok{ }\NormalTok{df[[lifetime]]}
\NormalTok{    k \textless{}{-}}\StringTok{ }\KeywordTok{length}\NormalTok{(theta)}
\NormalTok{    shapes \textless{}{-}}\StringTok{ }\NormalTok{theta[}\KeywordTok{seq}\NormalTok{(}\DecValTok{1}\NormalTok{, k, }\DecValTok{2}\NormalTok{)]}
\NormalTok{    scales \textless{}{-}}\StringTok{ }\NormalTok{theta[}\KeywordTok{seq}\NormalTok{(}\DecValTok{2}\NormalTok{, k, }\DecValTok{2}\NormalTok{)]}

\NormalTok{    s \textless{}{-}}\StringTok{ }\DecValTok{0}
    \ControlFlowTok{for}\NormalTok{ (i }\ControlFlowTok{in} \DecValTok{1}\OperatorTok{:}\NormalTok{n) \{}
\NormalTok{        s \textless{}{-}}\StringTok{ }\NormalTok{s }\OperatorTok{{-}}\StringTok{ }\KeywordTok{sum}\NormalTok{((t[i]}\OperatorTok{/}\NormalTok{scales)}\OperatorTok{\^{}}\NormalTok{shapes)}
        \ControlFlowTok{if}\NormalTok{ (delta[i]) \{}
\NormalTok{            s \textless{}{-}}\StringTok{ }\NormalTok{s }\OperatorTok{+}\StringTok{ }\KeywordTok{log}\NormalTok{(}\KeywordTok{sum}\NormalTok{(shapes[C[i, ]]}\OperatorTok{/}\NormalTok{scales[C[i, ]] }\OperatorTok{*}\StringTok{ }\NormalTok{(t[i]}\OperatorTok{/}\NormalTok{scales[C[i,}
\NormalTok{                ]])}\OperatorTok{\^{}}\NormalTok{(shapes[C[i, ]] }\OperatorTok{{-}}\StringTok{ }\DecValTok{1}\NormalTok{)))}
\NormalTok{        \}}
\NormalTok{    \}}
\NormalTok{    s}
\NormalTok{\}}
\end{Highlighting}
\end{Shaded}

\hypertarget{appendix-b-score-function}{%
\section*{Appendix B: Score Function}\label{appendix-b-score-function}}
\addcontentsline{toc}{section}{Appendix B: Score Function}

\label{app:score-code}

The following code is the score function (gradient of the log-likelihood
function with respect to \(\boldsymbol{\theta}\)) for the Weibull series
system with a likelihood model that includes masked component cause of
failure and right-censoring. It is implemented in the R library
\texttt{wei.series.md.c1.c2.c3} and is available on
\href{https://github.com/queelius/wei.series.md.c1.c2.c3}{GitHub}.

For clarity and brevity, we removed some of the functionality that is
not relevant to the analysis in this paper.

\begin{Shaded}
\begin{Highlighting}[]
\CommentTok{\#\textquotesingle{} Computes the score function (gradient of the log{-}likelihood function) for a}
\CommentTok{\#\textquotesingle{} Weibull series system with respect to parameter \textasciigrave{}theta\textasciigrave{} (shape, scale) for masked}
\CommentTok{\#\textquotesingle{} data with candidate sets that satisfy conditions C1, C2, and C3 and right{-}censored}
\CommentTok{\#\textquotesingle{} data.}
\CommentTok{\#\textquotesingle{}}
\CommentTok{\#\textquotesingle{} @param df (masked) data frame}
\CommentTok{\#\textquotesingle{} @param theta parameter vector (shape1, scale1, ..., shapem, scalem)}
\CommentTok{\#\textquotesingle{} @returns Score with respect to \textasciigrave{}theta\textasciigrave{} given \textasciigrave{}df\textasciigrave{}}
\NormalTok{score\_wei\_series\_md\_c1\_c2\_c3 \textless{}{-}}\StringTok{ }\ControlFlowTok{function}\NormalTok{(df, theta) \{}
\NormalTok{    n \textless{}{-}}\StringTok{ }\KeywordTok{nrow}\NormalTok{(df)}
\NormalTok{    C \textless{}{-}}\StringTok{ }\KeywordTok{md\_decode\_matrix}\NormalTok{(df, candset)}
\NormalTok{    m \textless{}{-}}\StringTok{ }\KeywordTok{ncol}\NormalTok{(C)}
\NormalTok{    delta \textless{}{-}}\StringTok{ }\NormalTok{df[[right\_censoring\_indicator]]}
\NormalTok{    t \textless{}{-}}\StringTok{ }\NormalTok{df[[lifetime]]}
\NormalTok{    shapes \textless{}{-}}\StringTok{ }\NormalTok{theta[}\KeywordTok{seq}\NormalTok{(}\DecValTok{1}\NormalTok{, }\KeywordTok{length}\NormalTok{(theta), }\DecValTok{2}\NormalTok{)]}
\NormalTok{    scales \textless{}{-}}\StringTok{ }\NormalTok{theta[}\KeywordTok{seq}\NormalTok{(}\DecValTok{2}\NormalTok{, }\KeywordTok{length}\NormalTok{(theta), }\DecValTok{2}\NormalTok{)]}
\NormalTok{    shape\_scores \textless{}{-}}\StringTok{ }\KeywordTok{rep}\NormalTok{(}\DecValTok{0}\NormalTok{, m)}
\NormalTok{    scale\_scores \textless{}{-}}\StringTok{ }\KeywordTok{rep}\NormalTok{(}\DecValTok{0}\NormalTok{, m)}

    \ControlFlowTok{for}\NormalTok{ (i }\ControlFlowTok{in} \DecValTok{1}\OperatorTok{:}\NormalTok{n) \{}
\NormalTok{        rt.term.shapes \textless{}{-}}\StringTok{ }\OperatorTok{{-}}\NormalTok{(t[i]}\OperatorTok{/}\NormalTok{scales)}\OperatorTok{\^{}}\NormalTok{shapes }\OperatorTok{*}\StringTok{ }\KeywordTok{log}\NormalTok{(t[i]}\OperatorTok{/}\NormalTok{scales)}
\NormalTok{        rt.term.scales \textless{}{-}}\StringTok{ }\NormalTok{(shapes}\OperatorTok{/}\NormalTok{scales) }\OperatorTok{*}\StringTok{ }\NormalTok{(t[i]}\OperatorTok{/}\NormalTok{scales)}\OperatorTok{\^{}}\NormalTok{shapes}

        \CommentTok{\# Initialize mask terms to 0}
\NormalTok{        mask.term.shapes \textless{}{-}}\StringTok{ }\KeywordTok{rep}\NormalTok{(}\DecValTok{0}\NormalTok{, m)}
\NormalTok{        mask.term.scales \textless{}{-}}\StringTok{ }\KeywordTok{rep}\NormalTok{(}\DecValTok{0}\NormalTok{, m)}

        \ControlFlowTok{if}\NormalTok{ (delta[i]) \{}
\NormalTok{            cindex \textless{}{-}}\StringTok{ }\NormalTok{C[i, ]}
\NormalTok{            denom \textless{}{-}}\StringTok{ }\KeywordTok{sum}\NormalTok{(shapes[cindex]}\OperatorTok{/}\NormalTok{scales[cindex] }\OperatorTok{*}\StringTok{ }\NormalTok{(t[i]}\OperatorTok{/}\NormalTok{scales[cindex])}\OperatorTok{\^{}}\NormalTok{(shapes[cindex] }\OperatorTok{{-}}
\StringTok{                }\DecValTok{1}\NormalTok{))}

\NormalTok{            numer.shapes \textless{}{-}}\StringTok{ }\DecValTok{1}\OperatorTok{/}\NormalTok{t[i] }\OperatorTok{*}\StringTok{ }\NormalTok{(t[i]}\OperatorTok{/}\NormalTok{scales[cindex])}\OperatorTok{\^{}}\NormalTok{shapes[cindex] }\OperatorTok{*}
\StringTok{                }\NormalTok{(}\DecValTok{1} \OperatorTok{+}\StringTok{ }\NormalTok{shapes[cindex] }\OperatorTok{*}\StringTok{ }\KeywordTok{log}\NormalTok{(t[i]}\OperatorTok{/}\NormalTok{scales[cindex]))}
\NormalTok{            mask.term.shapes[cindex] \textless{}{-}}\StringTok{ }\NormalTok{numer.shapes}\OperatorTok{/}\NormalTok{denom}

\NormalTok{            numer.scales \textless{}{-}}\StringTok{ }\NormalTok{(shapes[cindex]}\OperatorTok{/}\NormalTok{scales[cindex])}\OperatorTok{\^{}}\DecValTok{2} \OperatorTok{*}\StringTok{ }\NormalTok{(t[i]}\OperatorTok{/}\NormalTok{scales[cindex])}\OperatorTok{\^{}}\NormalTok{(shapes[cindex] }\OperatorTok{{-}}
\StringTok{                }\DecValTok{1}\NormalTok{)}
\NormalTok{            mask.term.scales[cindex] \textless{}{-}}\StringTok{ }\NormalTok{numer.scales}\OperatorTok{/}\NormalTok{denom}
\NormalTok{        \}}

\NormalTok{        shape\_scores \textless{}{-}}\StringTok{ }\NormalTok{shape\_scores }\OperatorTok{+}\StringTok{ }\NormalTok{rt.term.shapes }\OperatorTok{+}\StringTok{ }\NormalTok{mask.term.shapes}
\NormalTok{        scale\_scores \textless{}{-}}\StringTok{ }\NormalTok{scale\_scores }\OperatorTok{+}\StringTok{ }\NormalTok{rt.term.scales }\OperatorTok{{-}}\StringTok{ }\NormalTok{mask.term.scales}
\NormalTok{    \}}

\NormalTok{    scr \textless{}{-}}\StringTok{ }\KeywordTok{rep}\NormalTok{(}\DecValTok{0}\NormalTok{, }\KeywordTok{length}\NormalTok{(theta))}
\NormalTok{    scr[}\KeywordTok{seq}\NormalTok{(}\DecValTok{1}\NormalTok{, }\KeywordTok{length}\NormalTok{(theta), }\DecValTok{2}\NormalTok{)] \textless{}{-}}\StringTok{ }\NormalTok{shape\_scores}
\NormalTok{    scr[}\KeywordTok{seq}\NormalTok{(}\DecValTok{2}\NormalTok{, }\KeywordTok{length}\NormalTok{(theta), }\DecValTok{2}\NormalTok{)] \textless{}{-}}\StringTok{ }\NormalTok{scale\_scores}
\NormalTok{    scr}
\NormalTok{\}}
\end{Highlighting}
\end{Shaded}

\hypertarget{appendix-c-simulation-of-scenarios-for-assessing-bootstrapped-bca-confidence-intervals}{%
\section*{Appendix C: Simulation of Scenarios For Assessing Bootstrapped
(BCa) Confidence
Intervals}\label{appendix-c-simulation-of-scenarios-for-assessing-bootstrapped-bca-confidence-intervals}}
\addcontentsline{toc}{section}{Appendix C: Simulation of Scenarios For
Assessing Bootstrapped (BCa) Confidence Intervals}

The following code is the Monte-carlo simulation code for estimating the
confidence intervals of the MLE using the bootstrap method.

\begin{Shaded}
\begin{Highlighting}[]
\CommentTok{\#\#\#\# Setup simulation parameters here \#\#\#\#}

\CommentTok{\# Set the theta parameter. It contains shape and scale}
\CommentTok{\# parameters. Add as many components as you desire. We show the}
\CommentTok{\# default values for the 5{-}component in our paper.}
\NormalTok{theta \textless{}{-}}\StringTok{ }\KeywordTok{c}\NormalTok{(}\DataTypeTok{shape1 =} \FloatTok{1.2576}\NormalTok{, }\DataTypeTok{scale1 =} \FloatTok{994.3661}\NormalTok{, }\DataTypeTok{shape2 =} \FloatTok{1.1635}\NormalTok{, }\DataTypeTok{scale2 =} \FloatTok{908.9458}\NormalTok{,}
    \DataTypeTok{shape3 =} \FloatTok{1.1308}\NormalTok{, }\DataTypeTok{scale3 =} \FloatTok{840.1141}\NormalTok{, }\DataTypeTok{shape4 =} \FloatTok{1.1802}\NormalTok{, }\DataTypeTok{scale4 =} \FloatTok{940.1342}\NormalTok{,}
    \DataTypeTok{shape5 =} \FloatTok{1.2034}\NormalTok{, }\DataTypeTok{scale5 =} \FloatTok{923.1631}\NormalTok{)}

\NormalTok{N \textless{}{-}}\StringTok{ }\KeywordTok{c}\NormalTok{(}\DecValTok{30}\NormalTok{, }\DecValTok{60}\NormalTok{, }\DecValTok{100}\NormalTok{)  }\CommentTok{\# Sample sizes to simulate}
\NormalTok{P \textless{}{-}}\StringTok{ }\KeywordTok{c}\NormalTok{(}\FloatTok{0.215}\NormalTok{, }\FloatTok{0.333}\NormalTok{)  }\CommentTok{\# masking probabilities to simulate}
\NormalTok{Q \textless{}{-}}\StringTok{ }\KeywordTok{c}\NormalTok{(}\FloatTok{0.825}\NormalTok{)  }\CommentTok{\# right censoring probabilities to simulate}
\NormalTok{R \textless{}{-}}\StringTok{ }\DecValTok{100}  \CommentTok{\# number of simulations per scenario}
\NormalTok{B \textless{}{-}}\StringTok{ }\NormalTok{750L  }\CommentTok{\# number of bootstrap samples}
\NormalTok{max\_iter \textless{}{-}}\StringTok{ }\NormalTok{150L  }\CommentTok{\# max iterations for MLE}
\NormalTok{max\_boot\_iter \textless{}{-}}\StringTok{ }\NormalTok{150L  }\CommentTok{\# max iterations for bootstrap MLE}
\NormalTok{total\_retries \textless{}{-}}\StringTok{ }\NormalTok{10000L  }\CommentTok{\# total number of retries for each scenario}
\NormalTok{n\_cores \textless{}{-}}\StringTok{ }\KeywordTok{detectCores}\NormalTok{() }\OperatorTok{{-}}\StringTok{ }\DecValTok{1}  \CommentTok{\# number of cores to use for parallel processing}
\NormalTok{filename \textless{}{-}}\StringTok{ "data{-}boot{-}tau{-}fixed{-}bca{-}p{-}vs{-}ci"}  \CommentTok{\# filename prefix for output files}
\NormalTok{ci\_method \textless{}{-}}\StringTok{ "bca"}  \CommentTok{\# bootstrap CI method. See ?boot::boot.ci for details.}
\NormalTok{ci\_level \textless{}{-}}\StringTok{ }\FloatTok{0.95}  \CommentTok{\# confidence interval level}

\CommentTok{\#\#\#\# Simulation code below here \#\#\#\#}
\KeywordTok{library}\NormalTok{(tidyverse)}
\KeywordTok{library}\NormalTok{(parallel)}
\KeywordTok{library}\NormalTok{(boot)}
\KeywordTok{library}\NormalTok{(algebraic.mle)  }\CommentTok{\# for \textasciigrave{}mle\_boot\textasciigrave{}}
\KeywordTok{library}\NormalTok{(wei.series.md.c1.c2.c3)  }\CommentTok{\# for \textasciigrave{}mle\_lbfgsb\_wei\_series\_md\_c1\_c2\_c3\textasciigrave{} etc}

\NormalTok{file.meta \textless{}{-}}\StringTok{ }\KeywordTok{paste0}\NormalTok{(filename, }\StringTok{".txt"}\NormalTok{)}
\NormalTok{file.csv \textless{}{-}}\StringTok{ }\KeywordTok{paste0}\NormalTok{(filename, }\StringTok{".csv"}\NormalTok{)}
\ControlFlowTok{if}\NormalTok{ (}\KeywordTok{file.exists}\NormalTok{(file.meta)) \{}
    \KeywordTok{stop}\NormalTok{(}\StringTok{"File already exists: "}\NormalTok{, file.meta)}
\NormalTok{\}}
\ControlFlowTok{if}\NormalTok{ (}\KeywordTok{file.exists}\NormalTok{(file.csv)) \{}
    \KeywordTok{stop}\NormalTok{(}\StringTok{"File already exists: "}\NormalTok{, file.csv)}
\NormalTok{\}}

\NormalTok{shapes \textless{}{-}}\StringTok{ }\NormalTok{theta[}\KeywordTok{seq}\NormalTok{(}\DecValTok{1}\NormalTok{, }\KeywordTok{length}\NormalTok{(theta), }\DecValTok{2}\NormalTok{)]}
\NormalTok{scales \textless{}{-}}\StringTok{ }\NormalTok{theta[}\KeywordTok{seq}\NormalTok{(}\DecValTok{2}\NormalTok{, }\KeywordTok{length}\NormalTok{(theta), }\DecValTok{2}\NormalTok{)]}
\NormalTok{m \textless{}{-}}\StringTok{ }\KeywordTok{length}\NormalTok{(shapes)}

\KeywordTok{sink}\NormalTok{(file.meta)}
\KeywordTok{cat}\NormalTok{(}\StringTok{"boostrap of confidence intervals:}\CharTok{\textbackslash{}n}\StringTok{"}\NormalTok{)}
\KeywordTok{cat}\NormalTok{(}\StringTok{"   simulated on: "}\NormalTok{, }\KeywordTok{Sys.time}\NormalTok{(), }\StringTok{"}\CharTok{\textbackslash{}n}\StringTok{"}\NormalTok{)}
\KeywordTok{cat}\NormalTok{(}\StringTok{"   type: "}\NormalTok{, ci\_method, }\StringTok{"}\CharTok{\textbackslash{}n}\StringTok{"}\NormalTok{)}
\KeywordTok{cat}\NormalTok{(}\StringTok{"weibull series system:}\CharTok{\textbackslash{}n}\StringTok{"}\NormalTok{)}
\KeywordTok{cat}\NormalTok{(}\StringTok{"   number of components: "}\NormalTok{, m, }\StringTok{"}\CharTok{\textbackslash{}n}\StringTok{"}\NormalTok{)}
\KeywordTok{cat}\NormalTok{(}\StringTok{"   scale parameters: "}\NormalTok{, scales, }\StringTok{"}\CharTok{\textbackslash{}n}\StringTok{"}\NormalTok{)}
\KeywordTok{cat}\NormalTok{(}\StringTok{"   shape parameters: "}\NormalTok{, shapes, }\StringTok{"}\CharTok{\textbackslash{}n}\StringTok{"}\NormalTok{)}
\KeywordTok{cat}\NormalTok{(}\StringTok{"simulation parameters:}\CharTok{\textbackslash{}n}\StringTok{"}\NormalTok{)}
\KeywordTok{cat}\NormalTok{(}\StringTok{"   N: "}\NormalTok{, N, }\StringTok{"}\CharTok{\textbackslash{}n}\StringTok{"}\NormalTok{)}
\KeywordTok{cat}\NormalTok{(}\StringTok{"   P: "}\NormalTok{, P, }\StringTok{"}\CharTok{\textbackslash{}n}\StringTok{"}\NormalTok{)}
\KeywordTok{cat}\NormalTok{(}\StringTok{"   Q: "}\NormalTok{, Q, }\StringTok{"}\CharTok{\textbackslash{}n}\StringTok{"}\NormalTok{)}
\KeywordTok{cat}\NormalTok{(}\StringTok{"   R: "}\NormalTok{, R, }\StringTok{"}\CharTok{\textbackslash{}n}\StringTok{"}\NormalTok{)}
\KeywordTok{cat}\NormalTok{(}\StringTok{"   B: "}\NormalTok{, B, }\StringTok{"}\CharTok{\textbackslash{}n}\StringTok{"}\NormalTok{)}
\KeywordTok{cat}\NormalTok{(}\StringTok{"   max\_iter: "}\NormalTok{, max\_iter, }\StringTok{"}\CharTok{\textbackslash{}n}\StringTok{"}\NormalTok{)}
\KeywordTok{cat}\NormalTok{(}\StringTok{"   max\_boot\_iter: "}\NormalTok{, max\_boot\_iter, }\StringTok{"}\CharTok{\textbackslash{}n}\StringTok{"}\NormalTok{)}
\KeywordTok{cat}\NormalTok{(}\StringTok{"   n\_cores: "}\NormalTok{, n\_cores, }\StringTok{"}\CharTok{\textbackslash{}n}\StringTok{"}\NormalTok{)}
\KeywordTok{cat}\NormalTok{(}\StringTok{"   total\_retries: "}\NormalTok{, total\_retries, }\StringTok{"}\CharTok{\textbackslash{}n}\StringTok{"}\NormalTok{)}
\KeywordTok{sink}\NormalTok{()}

\ControlFlowTok{for}\NormalTok{ (n }\ControlFlowTok{in}\NormalTok{ N) \{}
    \ControlFlowTok{for}\NormalTok{ (p }\ControlFlowTok{in}\NormalTok{ P) \{}
        \ControlFlowTok{for}\NormalTok{ (q }\ControlFlowTok{in}\NormalTok{ Q) \{}
            \KeywordTok{cat}\NormalTok{(}\StringTok{"[starting scenario: n = "}\NormalTok{, n, }\StringTok{", p = "}\NormalTok{, p, }\StringTok{", q = "}\NormalTok{,}
\NormalTok{                q, }\StringTok{"]}\CharTok{\textbackslash{}n}\StringTok{"}\NormalTok{)}
\NormalTok{            tau \textless{}{-}}\StringTok{ }\KeywordTok{qwei\_series}\NormalTok{(}\DataTypeTok{p =}\NormalTok{ q, }\DataTypeTok{scales =}\NormalTok{ scales, }\DataTypeTok{shapes =}\NormalTok{ shapes)}

            \CommentTok{\# we compute R MLEs for each scenario}
\NormalTok{            shapes.mle \textless{}{-}}\StringTok{ }\KeywordTok{matrix}\NormalTok{(}\OtherTok{NA}\NormalTok{, }\DataTypeTok{nrow =}\NormalTok{ R, }\DataTypeTok{ncol =}\NormalTok{ m)}
\NormalTok{            scales.mle \textless{}{-}}\StringTok{ }\KeywordTok{matrix}\NormalTok{(}\OtherTok{NA}\NormalTok{, }\DataTypeTok{nrow =}\NormalTok{ R, }\DataTypeTok{ncol =}\NormalTok{ m)}
\NormalTok{            shapes.lower \textless{}{-}}\StringTok{ }\KeywordTok{matrix}\NormalTok{(}\OtherTok{NA}\NormalTok{, }\DataTypeTok{nrow =}\NormalTok{ R, }\DataTypeTok{ncol =}\NormalTok{ m)}
\NormalTok{            shapes.upper \textless{}{-}}\StringTok{ }\KeywordTok{matrix}\NormalTok{(}\OtherTok{NA}\NormalTok{, }\DataTypeTok{nrow =}\NormalTok{ R, }\DataTypeTok{ncol =}\NormalTok{ m)}
\NormalTok{            scales.lower \textless{}{-}}\StringTok{ }\KeywordTok{matrix}\NormalTok{(}\OtherTok{NA}\NormalTok{, }\DataTypeTok{nrow =}\NormalTok{ R, }\DataTypeTok{ncol =}\NormalTok{ m)}
\NormalTok{            scales.upper \textless{}{-}}\StringTok{ }\KeywordTok{matrix}\NormalTok{(}\OtherTok{NA}\NormalTok{, }\DataTypeTok{nrow =}\NormalTok{ R, }\DataTypeTok{ncol =}\NormalTok{ m)}
\NormalTok{            logliks \textless{}{-}}\StringTok{ }\KeywordTok{rep}\NormalTok{(}\DecValTok{0}\NormalTok{, R)}

\NormalTok{            iter \textless{}{-}}\StringTok{ }\NormalTok{0L}
\NormalTok{            retries \textless{}{-}}\StringTok{ }\NormalTok{0L}
            \ControlFlowTok{repeat}\NormalTok{ \{}
\NormalTok{                retry \textless{}{-}}\StringTok{ }\OtherTok{FALSE}
                \KeywordTok{tryCatch}\NormalTok{(\{}
                  \ControlFlowTok{repeat}\NormalTok{ \{}
\NormalTok{                    df \textless{}{-}}\StringTok{ }\KeywordTok{generate\_guo\_weibull\_table\_2\_data}\NormalTok{(}\DataTypeTok{shapes =}\NormalTok{ shapes,}
                      \DataTypeTok{scales =}\NormalTok{ scales, }\DataTypeTok{n =}\NormalTok{ n, }\DataTypeTok{p =}\NormalTok{ p, }\DataTypeTok{tau =}\NormalTok{ tau)}

\NormalTok{                    sol \textless{}{-}}\StringTok{ }\KeywordTok{mle\_lbfgsb\_wei\_series\_md\_c1\_c2\_c3}\NormalTok{(}\DataTypeTok{theta0 =}\NormalTok{ theta,}
                      \DataTypeTok{df =}\NormalTok{ df, }\DataTypeTok{hessian =} \OtherTok{FALSE}\NormalTok{, }\DataTypeTok{control =} \KeywordTok{list}\NormalTok{(}\DataTypeTok{maxit =}\NormalTok{ max\_iter,}
                        \DataTypeTok{parscale =}\NormalTok{ theta))}
                    \ControlFlowTok{if}\NormalTok{ (sol}\OperatorTok{$}\NormalTok{convergence }\OperatorTok{==}\StringTok{ }\DecValTok{0}\NormalTok{) \{}
                      \ControlFlowTok{break}
\NormalTok{                    \}}
                    \KeywordTok{cat}\NormalTok{(}\StringTok{"["}\NormalTok{, iter, }\StringTok{"] MLE did not converge, retrying.}\CharTok{\textbackslash{}n}\StringTok{"}\NormalTok{)}
\NormalTok{                  \}}

\NormalTok{                  mle\_solver \textless{}{-}}\StringTok{ }\ControlFlowTok{function}\NormalTok{(df, i) \{}
                    \KeywordTok{mle\_lbfgsb\_wei\_series\_md\_c1\_c2\_c3}\NormalTok{(}\DataTypeTok{theta0 =}\NormalTok{ sol}\OperatorTok{$}\NormalTok{par,}
                      \DataTypeTok{df =}\NormalTok{ df[i, ], }\DataTypeTok{hessian =} \OtherTok{FALSE}\NormalTok{, }\DataTypeTok{control =} \KeywordTok{list}\NormalTok{(}\DataTypeTok{maxit =}\NormalTok{ max\_boot\_iter,}
                        \DataTypeTok{parscale =}\NormalTok{ sol}\OperatorTok{$}\NormalTok{par))}\OperatorTok{$}\NormalTok{par}
\NormalTok{                  \}}

                  \CommentTok{\# do the non{-}parametric bootstrap}
\NormalTok{                  sol.boot \textless{}{-}}\StringTok{ }\KeywordTok{boot}\NormalTok{(df, mle\_solver, }\DataTypeTok{R =}\NormalTok{ B, }\DataTypeTok{parallel =} \StringTok{"multicore"}\NormalTok{,}
                    \DataTypeTok{ncpus =}\NormalTok{ n\_cores)}
\NormalTok{                \}, }\DataTypeTok{error =} \ControlFlowTok{function}\NormalTok{(e) \{}
                  \KeywordTok{cat}\NormalTok{(}\StringTok{"Error: "}\NormalTok{, }\KeywordTok{conditionMessage}\NormalTok{(e), }\StringTok{"}\CharTok{\textbackslash{}n}\StringTok{"}\NormalTok{)}
\NormalTok{                  retries \textless{}\textless{}{-}}\StringTok{ }\NormalTok{retries }\OperatorTok{+}\StringTok{ }\NormalTok{1L}
                  \ControlFlowTok{if}\NormalTok{ (retries }\OperatorTok{\textless{}}\StringTok{ }\NormalTok{total\_retries) \{}
                    \KeywordTok{cat}\NormalTok{(}\StringTok{"[scenario: n = "}\NormalTok{, n, }\StringTok{", p = "}\NormalTok{, p, }\StringTok{", q = "}\NormalTok{,}
\NormalTok{                      q, }\StringTok{"]: "}\NormalTok{, retries, }\StringTok{"/"}\NormalTok{, total\_retries, }\StringTok{" retries}\CharTok{\textbackslash{}n}\StringTok{"}\NormalTok{)}
\NormalTok{                    retry \textless{}\textless{}{-}}\StringTok{ }\OtherTok{TRUE}
\NormalTok{                  \}}
\NormalTok{                \})}
                \ControlFlowTok{if}\NormalTok{ (retries }\OperatorTok{\textgreater{}}\StringTok{ }\NormalTok{total\_retries) \{}
                  \ControlFlowTok{break}
\NormalTok{                \}}

                \ControlFlowTok{if}\NormalTok{ (retry) \{}
                  \ControlFlowTok{next}
\NormalTok{                \}}
\NormalTok{                iter \textless{}{-}}\StringTok{ }\NormalTok{iter }\OperatorTok{+}\StringTok{ }\NormalTok{1L}
\NormalTok{                shapes.mle[iter, ] \textless{}{-}}\StringTok{ }\NormalTok{sol}\OperatorTok{$}\NormalTok{par[}\KeywordTok{seq}\NormalTok{(}\DecValTok{1}\NormalTok{, }\KeywordTok{length}\NormalTok{(theta),}
                  \DecValTok{2}\NormalTok{)]}
\NormalTok{                scales.mle[iter, ] \textless{}{-}}\StringTok{ }\NormalTok{sol}\OperatorTok{$}\NormalTok{par[}\KeywordTok{seq}\NormalTok{(}\DecValTok{2}\NormalTok{, }\KeywordTok{length}\NormalTok{(theta),}
                  \DecValTok{2}\NormalTok{)]}
\NormalTok{                logliks[iter] \textless{}{-}}\StringTok{ }\NormalTok{sol}\OperatorTok{$}\NormalTok{value}

                \KeywordTok{tryCatch}\NormalTok{(\{}
\NormalTok{                  ci \textless{}{-}}\StringTok{ }\KeywordTok{confint}\NormalTok{(}\KeywordTok{mle\_boot}\NormalTok{(sol.boot), }\DataTypeTok{type =}\NormalTok{ ci\_method,}
                    \DataTypeTok{level =}\NormalTok{ ci\_level)}
\NormalTok{                  shapes.ci \textless{}{-}}\StringTok{ }\NormalTok{ci[}\KeywordTok{seq}\NormalTok{(}\DecValTok{1}\NormalTok{, }\KeywordTok{length}\NormalTok{(theta), }\DecValTok{2}\NormalTok{), ]}
\NormalTok{                  scales.ci \textless{}{-}}\StringTok{ }\NormalTok{ci[}\KeywordTok{seq}\NormalTok{(}\DecValTok{2}\NormalTok{, }\KeywordTok{length}\NormalTok{(theta), }\DecValTok{2}\NormalTok{), ]}
\NormalTok{                  shapes.lower[iter, ] \textless{}{-}}\StringTok{ }\NormalTok{shapes.ci[, }\DecValTok{1}\NormalTok{]}
\NormalTok{                  shapes.upper[iter, ] \textless{}{-}}\StringTok{ }\NormalTok{shapes.ci[, }\DecValTok{2}\NormalTok{]}
\NormalTok{                  scales.lower[iter, ] \textless{}{-}}\StringTok{ }\NormalTok{scales.ci[, }\DecValTok{1}\NormalTok{]}
\NormalTok{                  scales.upper[iter, ] \textless{}{-}}\StringTok{ }\NormalTok{scales.ci[, }\DecValTok{2}\NormalTok{]}
\NormalTok{                \}, }\DataTypeTok{error =} \ControlFlowTok{function}\NormalTok{(e) \{}
                  \KeywordTok{cat}\NormalTok{(}\StringTok{"[error] "}\NormalTok{, }\KeywordTok{conditionMessage}\NormalTok{(e), }\StringTok{"}\CharTok{\textbackslash{}n}\StringTok{"}\NormalTok{)}
\NormalTok{                \})}
                \ControlFlowTok{if}\NormalTok{ (iter}\OperatorTok{\%\%}\DecValTok{10} \OperatorTok{==}\StringTok{ }\DecValTok{0}\NormalTok{) \{}
                  \KeywordTok{cat}\NormalTok{(}\StringTok{"[iteration "}\NormalTok{, iter, }\StringTok{"] shapes = "}\NormalTok{, shapes.mle[iter,}
\NormalTok{                    ], }\StringTok{"scales = "}\NormalTok{, scales.mle[iter, ], }\StringTok{"}\CharTok{\textbackslash{}n}\StringTok{"}\NormalTok{)}
\NormalTok{                \}}

                \ControlFlowTok{if}\NormalTok{ (iter }\OperatorTok{==}\StringTok{ }\NormalTok{R) \{}
                  \ControlFlowTok{break}
\NormalTok{                \}}
\NormalTok{            \}}

\NormalTok{            df \textless{}{-}}\StringTok{ }\KeywordTok{data.frame}\NormalTok{(}\DataTypeTok{n =} \KeywordTok{rep}\NormalTok{(n, R), }\DataTypeTok{p =} \KeywordTok{rep}\NormalTok{(p, R), }\DataTypeTok{q =} \KeywordTok{rep}\NormalTok{(q,}
\NormalTok{                R), }\DataTypeTok{tau =} \KeywordTok{rep}\NormalTok{(tau, R), }\DataTypeTok{B =} \KeywordTok{rep}\NormalTok{(B, R), }\DataTypeTok{shapes =}\NormalTok{ shapes.mle,}
                \DataTypeTok{scales =}\NormalTok{ scales.mle, }\DataTypeTok{shapes.lower =}\NormalTok{ shapes.lower,}
                \DataTypeTok{shapes.upper =}\NormalTok{ shapes.upper, }\DataTypeTok{scales.lower =}\NormalTok{ scales.lower,}
                \DataTypeTok{scales.upper =}\NormalTok{ scales.upper, }\DataTypeTok{logliks =}\NormalTok{ logliks)}

            \KeywordTok{write.table}\NormalTok{(df, }\DataTypeTok{file =}\NormalTok{ file.csv, }\DataTypeTok{sep =} \StringTok{","}\NormalTok{, }\DataTypeTok{row.names =} \OtherTok{FALSE}\NormalTok{,}
                \DataTypeTok{col.names =} \OperatorTok{!}\KeywordTok{file.exists}\NormalTok{(file.csv), }\DataTypeTok{append =} \OtherTok{TRUE}\NormalTok{)}
\NormalTok{        \}}
\NormalTok{    \}}
\NormalTok{\}}
\end{Highlighting}
\end{Shaded}

\hypertarget{appendix-d-flat-log-likelihood-experiment}{%
\section*{Appendix D: Flat Log-likelihood
Experiment}\label{appendix-d-flat-log-likelihood-experiment}}
\addcontentsline{toc}{section}{Appendix D: Flat Log-likelihood
Experiment}

\label{app:flat-like_code} Here is the code to generate the data set for
Figure \ref{fig:flat-loglike-prof}.

\begin{Shaded}
\begin{Highlighting}[]
\KeywordTok{library}\NormalTok{(wei.series.md.c1.c2.c3)}

\CommentTok{\# changed scale3 to make MTTF of component 3 much smaller than}
\CommentTok{\# the others}
\NormalTok{theta \textless{}{-}}\StringTok{ }\KeywordTok{c}\NormalTok{(}\DataTypeTok{shape1 =} \FloatTok{1.2576}\NormalTok{, }\DataTypeTok{scale1 =} \FloatTok{994.3661}\NormalTok{, }\DataTypeTok{shape2 =} \FloatTok{1.1635}\NormalTok{, }\DataTypeTok{scale2 =} \FloatTok{908.9458}\NormalTok{,}
    \DataTypeTok{shape3 =} \FloatTok{1.1308}\NormalTok{, }\DataTypeTok{scale3 =} \FloatTok{4.1141}\NormalTok{)}
\NormalTok{shapes \textless{}{-}}\StringTok{ }\NormalTok{theta[}\KeywordTok{seq}\NormalTok{(}\DecValTok{1}\NormalTok{, }\KeywordTok{length}\NormalTok{(theta), }\DecValTok{2}\NormalTok{)]}
\NormalTok{scales \textless{}{-}}\StringTok{ }\NormalTok{theta[}\KeywordTok{seq}\NormalTok{(}\DecValTok{2}\NormalTok{, }\KeywordTok{length}\NormalTok{(theta), }\DecValTok{2}\NormalTok{)]}

\NormalTok{n \textless{}{-}}\StringTok{ }\DecValTok{30}
\NormalTok{p \textless{}{-}}\StringTok{ }\FloatTok{0.215}
\NormalTok{q \textless{}{-}}\StringTok{ }\FloatTok{0.825}
\KeywordTok{set.seed}\NormalTok{(}\DecValTok{151234}\NormalTok{)}
\NormalTok{tau \textless{}{-}}\StringTok{ }\KeywordTok{qwei\_series}\NormalTok{(}\DataTypeTok{p =}\NormalTok{ q, }\DataTypeTok{scales =}\NormalTok{ scales, }\DataTypeTok{shapes =}\NormalTok{ shapes)}
\NormalTok{df \textless{}{-}}\StringTok{ }\KeywordTok{generate\_guo\_weibull\_table\_2\_data}\NormalTok{(}\DataTypeTok{shapes =}\NormalTok{ shapes, }\DataTypeTok{scales =}\NormalTok{ scales,}
    \DataTypeTok{n =}\NormalTok{ n, }\DataTypeTok{p =}\NormalTok{ p, }\DataTypeTok{tau =}\NormalTok{ tau)}
\NormalTok{df.tweaked \textless{}{-}}\StringTok{ }\NormalTok{df}
\NormalTok{df.tweaked[}\DecValTok{1}\NormalTok{, }\StringTok{"x3"}\NormalTok{] \textless{}{-}}\StringTok{ }\OtherTok{FALSE}  \CommentTok{\# remove component 3 from first observation}
\NormalTok{df.tweaked[}\DecValTok{1}\NormalTok{, }\StringTok{"x1"}\NormalTok{] \textless{}{-}}\StringTok{ }\OtherTok{TRUE}  \CommentTok{\# add component 1 to first observation}

\NormalTok{sol \textless{}{-}}\StringTok{ }\KeywordTok{mle\_lbfgsb\_wei\_series\_md\_c1\_c2\_c3}\NormalTok{(}\DataTypeTok{theta0 =}\NormalTok{ theta, }\DataTypeTok{df =}\NormalTok{ df,}
    \DataTypeTok{hessian =} \OtherTok{FALSE}\NormalTok{, }\DataTypeTok{control =} \KeywordTok{list}\NormalTok{(}\DataTypeTok{maxit =} \DecValTok{1000}\NormalTok{))}
\NormalTok{sol.tweaked \textless{}{-}}\StringTok{ }\KeywordTok{mle\_lbfgsb\_wei\_series\_md\_c1\_c2\_c3}\NormalTok{(}\DataTypeTok{theta0 =}\NormalTok{ theta, }\DataTypeTok{df =}\NormalTok{ df.tweaked,}
    \DataTypeTok{hessian =} \OtherTok{FALSE}\NormalTok{, }\DataTypeTok{control =} \KeywordTok{list}\NormalTok{(}\DataTypeTok{maxit =} \DecValTok{1000}\NormalTok{))}
\NormalTok{l \textless{}{-}}\StringTok{ }\KeywordTok{Vectorize}\NormalTok{(}\ControlFlowTok{function}\NormalTok{(shape1) \{}
\NormalTok{    theta \textless{}{-}}\StringTok{ }\NormalTok{sol}\OperatorTok{$}\NormalTok{par}
\NormalTok{    theta[}\DecValTok{1}\NormalTok{] \textless{}{-}}\StringTok{ }\NormalTok{shape1}
    \KeywordTok{loglik\_wei\_series\_md\_c1\_c2\_c3}\NormalTok{(df, theta)}
\NormalTok{\}, }\DataTypeTok{vectorize.args =} \StringTok{"shape1"}\NormalTok{)}
\NormalTok{l.tweaked \textless{}{-}}\StringTok{ }\KeywordTok{Vectorize}\NormalTok{(}\ControlFlowTok{function}\NormalTok{(shape1) \{}
\NormalTok{    theta \textless{}{-}}\StringTok{ }\NormalTok{sol.tweaked}\OperatorTok{$}\NormalTok{par}
\NormalTok{    theta[}\DecValTok{1}\NormalTok{] \textless{}{-}}\StringTok{ }\NormalTok{shape1}
    \KeywordTok{loglik\_wei\_series\_md\_c1\_c2\_c3}\NormalTok{(df.tweaked, theta)}
\NormalTok{\}, }\DataTypeTok{vectorize.args =} \StringTok{"shape1"}\NormalTok{)}
\end{Highlighting}
\end{Shaded}


  \bibliography{refs.bib}

\end{document}
