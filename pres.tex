\documentclass[]{tufte-book}

% ams
\usepackage{amssymb,amsmath}

\usepackage{ifxetex,ifluatex}
\usepackage{fixltx2e} % provides \textsubscript
\ifnum 0\ifxetex 1\fi\ifluatex 1\fi=0 % if pdftex
  \usepackage[T1]{fontenc}
  \usepackage[utf8]{inputenc}
\else % if luatex or xelatex
  \makeatletter
  \@ifpackageloaded{fontspec}{}{\usepackage{fontspec}}
  \makeatother
  \defaultfontfeatures{Ligatures=TeX,Scale=MatchLowercase}
  \makeatletter
  \@ifpackageloaded{soul}{
     \renewcommand\allcapsspacing[1]{{\addfontfeature{LetterSpace=15}#1}}
     \renewcommand\smallcapsspacing[1]{{\addfontfeature{LetterSpace=10}#1}}
   }{}
  \makeatother

\fi

% graphix
\usepackage{graphicx}
\setkeys{Gin}{width=\linewidth,totalheight=\textheight,keepaspectratio}

% booktabs
\usepackage{booktabs}

% url
\usepackage{url}

% hyperref
\usepackage{hyperref}

% units.
\usepackage{units}


\setcounter{secnumdepth}{2}

% citations
\usepackage{natbib}
\bibliographystyle{plainnat}


% pandoc syntax highlighting

% table with pandoc
\usepackage{longtable,booktabs,array}
\usepackage{calc} % for calculating minipage widths
% Correct order of tables after \paragraph or \subparagraph
\usepackage{etoolbox}
\makeatletter
\patchcmd\longtable{\par}{\if@noskipsec\mbox{}\fi\par}{}{}
\makeatother
% Allow footnotes in longtable head/foot
\IfFileExists{footnotehyper.sty}{\usepackage{footnotehyper}}{\usepackage{footnote}}
\makesavenoteenv{longtable}

% multiplecol
\usepackage{multicol}

% strikeout
\usepackage[normalem]{ulem}

% morefloats
\usepackage{morefloats}


% tightlist macro required by pandoc >= 1.14
\providecommand{\tightlist}{%
  \setlength{\itemsep}{0pt}\setlength{\parskip}{0pt}}

% title / author / date
\title{Reliability Estimation in Series Systems}
\author{Alex Towell}
\date{}

\usepackage{tikz}
\usepackage{caption}
\usepackage{amsthm}
\renewcommand{\v}[1]{\boldsymbol{#1}}
\theoremstyle{definition}
\newtheorem{condition}{Condition}
\theoremstyle{plain}

\begin{document}

\maketitle



{
\setcounter{tocdepth}{1}
\tableofcontents
}

\hypertarget{introduction}{%
\chapter{Introduction}\label{introduction}}

\hypertarget{presentation-overview}{%
\section{Presentation Overview}\label{presentation-overview}}

\begin{itemize}
\tightlist
\item
  Reliability estimation in series systems
\item
  Challenges of masked and right-censored failure data
\item
  New maximum likelihood techniques
\item
  Modeling framework and results from simulation studies
\end{itemize}

\hypertarget{context-motivation}{%
\section{Context \& Motivation}\label{context-motivation}}

\begin{itemize}
\tightlist
\item
  Quantifying reliability in series systems is essential.
\item
  Real-world systems often only provide system-level failure data.
\item
  Masked and right-censored data obscure true reliability metrics.
\item
  Need robust techniques to decipher this data and make accurate estimations.
\end{itemize}

\hypertarget{core-contributions}{%
\section{Core Contributions}\label{core-contributions}}

\begin{enumerate}
\def\labelenumi{\arabic{enumi}.}
\tightlist
\item
  New likelihood model that accounts for right-censoring and masking.
\item
  Extensive simulation studies with Weibull-distributed lifetimes.
\item
  Evaluations of BCa confidence intervals.
\item
  Insights into the performance of the maximum likelihood estimator.
\end{enumerate}

\hypertarget{aim}{%
\section{Aim}\label{aim}}

\begin{itemize}
\tightlist
\item
  Offer a comprehensive understanding of reliability estimation techniques.
\item
  Validate the use of masked reliability data in such analyses.
\end{itemize}

\hypertarget{series-system-derivations}{%
\chapter{Series System Derivations}\label{series-system-derivations}}

\hypertarget{system-reliability-function}{%
\section{System Reliability Function}\label{system-reliability-function}}

\begin{itemize}
\tightlist
\item
  Describes the probability a system functions at a specific time.
  \(R_{T_i}(t';\boldsymbol{\theta})\) represents the probability the \(i^{th}\) system functions at time \(t'\).
\item
  Defined as the product of the reliabilities of its individual components.
  \[ R_{T_i}(t;\boldsymbol{\theta}) = \prod_{j=1}^m R_j(t;\boldsymbol{\theta_j}) \]
\end{itemize}

\hypertarget{system-hazard-function}{%
\section{System Hazard Function}\label{system-hazard-function}}

\begin{itemize}
\tightlist
\item
  Sum of the hazard functions of its components.
  \[ h_{T_i}(t;\boldsymbol{\theta}) = \sum_{j=1}^m h_j(t;\boldsymbol{\theta_j}) \]
\item
  Relation to the system's reliability and pdf:
  \[ f_{T_i}(t;\boldsymbol{\theta}) = \biggl\{\sum_{j=1}^m h_j(t;\boldsymbol{\theta_j})\biggr\} \biggl\{ \prod_{j=1}^m R_j(t;\boldsymbol{\theta_j}) \biggr\} \]
\end{itemize}

\hypertarget{system-pdf}{%
\section{System PDF}\label{system-pdf}}

\begin{itemize}
\tightlist
\item
  Represents how the likelihood of system failure varies over time.
  \[ f_{T_i}(t;\boldsymbol{\theta}) = h_{T_i}(t;\boldsymbol{\theta}) R_{T_i}(t;\boldsymbol{\theta}) \]
\end{itemize}

\hypertarget{summary}{%
\section{Summary}\label{summary}}

\begin{itemize}
\tightlist
\item
  Series system models derive mathematical relationships between component and system lifetimes.
\item
  These derivations provide a foundation for understanding system reliability and predicting failures.
\end{itemize}

\hypertarget{system-and-component-reliabilities}{%
\chapter{System and Component Reliabilities}\label{system-and-component-reliabilities}}

\hypertarget{mean-time-to-failure-mttf}{%
\section{Mean Time to Failure (MTTF)}\label{mean-time-to-failure-mttf}}

\begin{itemize}
\tightlist
\item
  A summary measure of the system's reliability.
  \[ \text{MTTF} = E_{\boldsymbol{\theta}}[T_i] \]
\item
  Integration of the reliability function over its support given certain assumptions\footnote{Assumptions: \(T_i\) is non-negative and continuous, \(R_{T_i}(t;\boldsymbol{\theta})\) is continuous and differentiable for \(t > 0\), and \(\int_0^\infty R_{T_i}(t;\boldsymbol{\theta}) dt\) converges.}.
\item
  MTTF can be misleading, especially for systems with fat-tailed distributions\footnote{Fat-tailed distributions have tails that decay slower than the exponential family. They can affect MTTF with higher likelihoods of extreme values.}.
\end{itemize}

\hypertarget{component-reliabilities}{%
\section{Component Reliabilities}\label{component-reliabilities}}

\begin{itemize}
\tightlist
\item
  System's reliability is determined by its components.
\item
  MTTF for the \(j^{th}\) component: \(\text{MTTF}_j\).
\item
  Probability \(j^{th}\) component causes failure: \(\Pr\{K_i = j\}\).
\item
  In a \textbf{well-designed} series system:

  \begin{itemize}
  \tightlist
  \item
    Components have similar MTTFs.
  \item
    Equal probabilities of being the failure cause.
  \end{itemize}
\end{itemize}

\hypertarget{upcoming}{%
\subsection{Upcoming}\label{upcoming}}

\begin{itemize}
\tightlist
\item
  Use the joint PDF of \(T\) and \(K\) in the likelihood model derivation.
\item
  Incorporate the reliability function in the likelihood model for right censoring.
\end{itemize}

\bibliography{refs.bib}



\end{document}
