\chapter{Optimal node inspection order}
In \Cref{sec:distribution_inspection_counts}, we considered the optimal inspection order to minimize the expected number of inspections required to identify a failed component. Again, consider order policy $\orderfunc$, a one-to-one function that maps each inspection order $n \in \{1, \ldots, m\}$ to a unique component index $j \in \{1, \ldots, m\}$. In what follows, we demonstrate how to compute the utility of a particular order policy, where the utility is characterized by a utility function $\operatorname{u}$.
\begin{definition}
The expected utility of order policy $\orderfunc$ given a probability mass function $\pmf_{\tparam{\vec\theta}}(k \given t)$ and a utility function $\operatorname{u}$ is
\begin{equation}
\label{eq:expected_utility}
    \exputility[\orderfunc] = \sum_{n=1}^{m} \operatorname{u}(\orderfunc(n), n)
        \cdot \pmf_{\tparam{\vec\theta}}(\orderfunc(n) \given t)
\end{equation}
where $\operatorname{u}$ is a utility function that takes as its first argument a component index and as its second argument an inspection order and returns the value of assigning the given inspection order to the given component index.
\end{definition}
The utility function $\operatorname{u}$ can take on many different forms. For instance, components may have different inspection costs, and thus given two components, all things else being equal the component that is less costly to inspect should be inspected before the component that is more costly to inspect in order to minimize the expected utility cost.

Let the optimal order policy $\orderfunc^{*}$ be defined as the order policy that maximizes the expected utility.
\begin{definition}
The optimal order policy $\orderfunc^{*}$ given utility function $\operatorname{u}$ is
\begin{equation}
    \orderfunc^{*} = \argmax_{\orderfunc} \exputility[\orderfunc]\,.
\end{equation}
\end{definition}

