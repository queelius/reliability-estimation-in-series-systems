\documentclass[../main.tex]{subfiles}
\begin{document}
\chapter{Introduction}
\label{sec:intro}
Empirical modeling involves a blending of substantive subject matter and statistical information.
In our case, the substantive information includes assumptions like a series system and the $\alpha$-masked candidate set models described later.
Such information suggests the relevant variables and simplifies learning from data (in our case, maximum likelihood estimation).

In contrast, statistical information stems from the chance regularities in data, which are statistical model attempts to ``adequately'' capture.


Discuss latent or hidden variables.

Draw a graph in prob. model section. What do we know, how we know it, etc.


We consider series systems consisting of some fixed number of components with 
uncertain lifetimes. We desire a mathematical model of the system that may be 
used to predict the failure time of the system and its component cause. 
However, since the system lifetime and the component cause of failure is 
uncertain, we are interested in probabilistically modeling the system lifetime, 
e.g., there is a $75\%$ chance that component $3$ will cause a system failure 
in the next $3$ years.

We assume the system belongs to some parametric family but whose true parameter 
index is unknown. Furthermore, we also assume there is a sample of 
\emph{masked system failure times} that carry information about the true 
parameter index.

Using the information in the sample, we employ the frequentist approach of 
repeated experiments which induces a sampling distribution on the maximum 
likelihood point estimator of the fixed but unknown true parameter index where 
\emph{a priori} any value in the parameter space is equally likely. The point 
estimator converges to a multivariate normal sampling distribution with a mean 
given by the true parameter index and a variance that is inversely proportional 
to sample size.
\end{document}