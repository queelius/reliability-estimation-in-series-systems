\documentclass[../main.tex]{subfiles}
\begin{document}
\chapter{Probabilistic model}
\label{sec:general_series_system}
%A probabilistic model is specified by equations involving random variables which make assumptions about how observable data about the system is generated. In %what follows, we describe our model and observable data.

A probabilistic model describes how observable data about the system is generated.
In what follows, we describe our model and observable data.

A \emph{set} is an unordered collection of distinct elements.
The set of all subsets of a set is called the \emph{powerset}.
The powerset of $\Set{X}$ is denoted by $\PS{\Set{X}}$.
An interval in $\PS{\RealSet}$ is a set of all the real numbers lying between two indicated real numbers, e.g.,
\begin{equation}
	(a,b] \coloneqq \SetBuilder{x \in \RealSet}{x \geq a \land x \leq b} \in \PS{\RealSet}\,.
\end{equation}

\begin{definition}
A probability space $(\Omega, \FancySet{F}, \Fun{P})$, where $\Fun{P} \colon \PS{\Omega} \mapsto [0,1]$ is the probability set function, $\Omega$ is the \emph{sample space}, and $\FancySet{F}$ is the set of measurable events (subsets of $\Omega$).
\end{definition}


A random variable is a measurable function defined on a probability space.
\begin{definition}
Given a probability space $(\Omega, \FancySet{F}, \Fun{P})$, a random variable $\RV{X}$ is a function of type $\Omega \mapsto \Codom(\RV{X})$.
\end{definition}
The range (or image) of $\RV{X}$ is also called the space of $\RV{X}$.

For a random variable $\RV{X}$, we use $\RV{X} = x$ to denote the subset containing all elements $e \in \Omega$ that $\RV{X}$ maps to the value of $x$.

\section{Series system}
The system consists of $m$ components. We make the following assumption about 
the state of the components.
\begin{assumption}
\label{asm:binary_state}
The \jth component is initially in a non-failed state and permanently enters a failed state at some uncertain time $t_j > 0$ for $j=1\,,\ldots\,,m$.
\end{assumption}
Components have uncertain lifetimes as given by the following definition.
\begin{definition}
The lifetime of the \jth component is given by the continuous random variable 
$\cv_j \in \left(0\,,\infty\right)$ for $j=1,\ldots,m$.
\end{definition}
We make the following assumption about the joint distribution of the component lifetimes.
\begin{assumption}
\label{asm:indep}
The random variables $\cv_1,\ldots,\cv_m$ are mutually independent.
\end{assumption}

The system is a series system.
\begin{assumption}
    The system is in a non-failed state if all $m$ components are in a non-failed state, otherwise the system is in a failed state.
\end{assumption}
A system in which at least $k$ of the components must be in a non-failed state for the system to be in a non-failed system is denoted a $k$-out-of-$m$ system, 
thus a series system is an $m$-out-of-$m$ system.

The uncertain lifetime of the system is given by the following definition.
\begin{theorem}
    \label{thm:series_system_lifetime}
    The lifetime of the system is given by the random variable
    \begin{equation}
    \label{eq:series_system_lifetime}
    \sv = \min{\!\left(\cv_1,\ldots,\cv_m\right)}
    \end{equation}
    with a sample space $\left(0, \infty\right)$, where $\cv_1,\ldots,\cv_m$ are the component lifetimes.
\end{theorem}
\begin{proof}
    A series system fails whenever any component fails, therefore its lifetime is equal to the lifetime of the component with the minimum lifetime.
\end{proof}

\section{Probability distribution functions and parametric families}
As random variables, it is not certain which values $\sv,\cv_1,\ldots,\cv_m$ 
will realize. However, their respective probability distribution functions 
quantify the uncertainty. The cumulative distribution function is given by the 
following definition.
\begin{definition}
    \label{def:cdf_function}
    Let $\RV{X}$ be some random variable. The probability that $\RV{X} \leq x$ 
    is given by its cumulative distribution function, denoted by
    \begin{equation}
    \cdf_{\RV{X}}\!\left(x\right) = \Prob{\RV{X} \leq x}\,.
    \end{equation}
\end{definition}
The probability density function is given by the following definition.
\begin{definition}
    \label{def:pdf}
    Let $\RV{X}$ be some continuous random variable. The \emph{relative 
        likelihood} that $\RV{X} = x$ is given by its probability density 
    function, denoted by
    \begin{equation}
    \pdf_{\RV{X}}(x) = \derivative{t}\cdf_{\RV{X}}(t)\,.
    \end{equation}
\end{definition}
By the Second Fundamental Theorem of Calculus, the probability that a 
continuous random variable $\RV{X} \in \left(a,b\right]$, $a \leq b$, is given 
by
\begin{equation}
\label{eq:prob_F_R_p}
\Prob{a < \RV{X} \leq b} = \cdf_{\RV{X}}(b) - \cdf_{\RV{X}}(a) = \int_{a}^{b} 
\pdf_{\RV{X}}(s) \mathrm{d}s\,.
\end{equation}

By these definitions, the relative likelihood that the system will fail at time 
$t$ is $\spdf(t)$ and probability the \jth component will before time $t$ is 
$\cdf_{\cv_j}(t)$.
\begin{notation}
We simplify the notation and let $\cdf_j$ denote $\cdf_{\cv_j}$ and $\pdf_j$ denote $\pdf_{\cv_j}$.
Column vectors are denoted by boldface lower case letters, e.g., $\vec{x}$, and row vectors are denoted by the transpose of column vectors, e.g., $\vec{x}^\Transpose$.
Matrices are denoted by boldface upper case greek letters, e.g., $\matrx{A}$.
The \jth column of $\matrx{A}$ is denoted by $\matrx{A}_j$ (or $\left[\matrx{A}\right]_j$) and the $(j,k)$-th element of $\matrx{A}$ is denoted by $\matrx{A}_{j\,k}$ (or $\left[\matrx{A}\right]_{j\,k}$).
\end{notation}

We restrict our attention to parametrized families of probability distribution functions. The system lifetime has a probability density function that is a member of the following parametric family.
\begin{definition}
    \label{def:sysparametric}
    The system lifetime has a probability density function that is a member of 
    the parametric family of probability density functions denoted by
    \begin{equation}
    \label{eq:paramsys}
    \pfam = \left \{ \spdf\left(\; \cdot \given \sysparam\right) \colon 
    \sysparam \in \RealSet^{q \times m} \right \}\,,
    \end{equation}
    where $\sysparam_j \in \boldsymbol\Omega$ corresponds to the parameter 
    index of the \jth component lifetime.
\end{definition}
Particular values of $\sysparam$ index particular probability density functions 
within the $\pfam$-family. The true probability density function of the system 
lifetime is given by the following definition.
\begin{definition}
    \label{def:true_param}
    The distribution of the system lifetime has a true parameter index denoted 
    by $\tparam\sysparam$, and we say that
    \begin{equation}
        \sv \sim \spdf\!\left(\;\cdot \given \tparam\sysparam\right)\,.
    \end{equation}
\end{definition}
By these definitions, it follows that
\begin{equation}
    \cv_j \sim \pdf_j\!\left(\; \cdot \given \tparam\sysparam_j\right)\,.
\end{equation}

\section{$\alpha$-masked component failure}
By \cref{thm:series_system_lifetime}, if a system failure occurs at time $\ti$, then one of the components failed at time $\ti$.\footnote{Only a single component can cause a system failure since two or more continuous random variables cannot realize the same value.}
\begin{definition}
\label{def:comp_cause}
The discrete random variable indicating the component responsible for a system failure is denoted by $\RV{K}$ with a sample space $\Set{K} = \{1,\ldots,m\}$. That is, $\RV{K} = j$ indicates that $\cv_j < \cv_k$ for $k \in \SetDiff[\Set{K}][\{ j \}]$.
\end{definition}

We may not be certain about which component caused a system failure, but we may be able to narrow the component cause to some \emph{subset}.
We denote these a \emph{masked} component failure as given by the following definition.
\begin{definition}
\label{def:random_candidate_set}
Let $\rvcand$ denote the random masked component failure with a sample space $\PS{\Set{K}}$.
\end{definition}

We suppose there is some process that, informed by some degree of knowledge about the components, generates a masked component failure which \emph{plausibly}\footnote{The random masked component failures carry information about the true parameter index $\tparam\sysparam$ by being dependent in some way on the component lifetimes.} contains the failed component as depicted in \cref{fig:gen_model}.
\begin{figure}
	\caption{Graphical model of random variables where $\sv = \min\{\cv_1,\cv_2\ldots,\cv_m\}$ and masked component failure $\rvcand$ is generated by some process that is a function of $\tparam\sysparam$.}
	\label{fig:gen_model}
	\begin{center}
		% Graphic for TeX using PGF
% Title: /home/spinoza/Documents/repos/papers/series_system_estimation_from_masked_system_failure_time_samples/research/graph_model.dia
% Creator: Dia v0.97+git
% CreationDate: Mon Jul  8 16:26:58 2019
% For: spinoza
% \usepackage{tikz}
% The following commands are not supported in PSTricks at present
% We define them conditionally, so when they are implemented,
% this pgf file will use them.
\ifx\du\undefined
  \newlength{\du}
\fi
\setlength{\du}{15\unitlength}
\begin{tikzpicture}[even odd rule]
\pgftransformxscale{1.000000}
\pgftransformyscale{-1.000000}
\definecolor{dialinecolor}{rgb}{0.000000, 0.000000, 0.000000}
\pgfsetstrokecolor{dialinecolor}
\pgfsetstrokeopacity{1.000000}
\definecolor{diafillcolor}{rgb}{1.000000, 1.000000, 1.000000}
\pgfsetfillcolor{diafillcolor}
\pgfsetfillopacity{1.000000}
\pgfsetlinewidth{0.100000\du}
\pgfsetdash{{\pgflinewidth}{0.200000\du}}{0cm}
\pgfsetmiterjoin
\pgfsetbuttcap
\definecolor{dialinecolor}{rgb}{0.498039, 0.498039, 0.498039}
\pgfsetstrokecolor{dialinecolor}
\pgfsetstrokeopacity{1.000000}
\pgfpathmoveto{\pgfpoint{41.000000\du}{10.000000\du}}
\pgfpathcurveto{\pgfpoint{41.000000\du}{10.000000\du}}{\pgfpoint{41.000000\du}{8.000000\du}}{\pgfpoint{44.000000\du}{8.000000\du}}
\pgfpathcurveto{\pgfpoint{47.000000\du}{8.000000\du}}{\pgfpoint{48.000000\du}{8.000000\du}}{\pgfpoint{49.000000\du}{8.000000\du}}
\pgfpathcurveto{\pgfpoint{50.000000\du}{8.000000\du}}{\pgfpoint{51.000000\du}{8.000000\du}}{\pgfpoint{54.000000\du}{8.000000\du}}
\pgfpathcurveto{\pgfpoint{57.000000\du}{8.000000\du}}{\pgfpoint{57.000000\du}{10.000000\du}}{\pgfpoint{57.000000\du}{10.000000\du}}
\pgfpathcurveto{\pgfpoint{57.000000\du}{10.000000\du}}{\pgfpoint{57.000000\du}{12.000000\du}}{\pgfpoint{54.000000\du}{12.000000\du}}
\pgfpathcurveto{\pgfpoint{51.000000\du}{12.000000\du}}{\pgfpoint{50.000000\du}{12.000000\du}}{\pgfpoint{49.000000\du}{12.000000\du}}
\pgfpathcurveto{\pgfpoint{48.000000\du}{12.000000\du}}{\pgfpoint{47.000000\du}{12.000000\du}}{\pgfpoint{44.000000\du}{12.000000\du}}
\pgfpathcurveto{\pgfpoint{41.000000\du}{12.000000\du}}{\pgfpoint{41.000000\du}{10.000000\du}}{\pgfpoint{41.000000\du}{10.000000\du}}
\pgfpathclose
\pgfusepath{stroke}
% setfont left to latex
\definecolor{dialinecolor}{rgb}{0.000000, 0.000000, 0.000000}
\pgfsetstrokecolor{dialinecolor}
\pgfsetstrokeopacity{1.000000}
\definecolor{diafillcolor}{rgb}{0.000000, 0.000000, 0.000000}
\pgfsetfillcolor{diafillcolor}
\pgfsetfillopacity{1.000000}
\node[anchor=base west,inner sep=0pt,outer sep=0pt,color=dialinecolor] at (43.000000\du,10.000000\du){};
% setfont left to latex
\definecolor{dialinecolor}{rgb}{0.000000, 0.000000, 0.000000}
\pgfsetstrokecolor{dialinecolor}
\pgfsetstrokeopacity{1.000000}
\definecolor{diafillcolor}{rgb}{0.000000, 0.000000, 0.000000}
\pgfsetfillcolor{diafillcolor}
\pgfsetfillopacity{1.000000}
\node[anchor=base west,inner sep=0pt,outer sep=0pt,color=dialinecolor] at (43.000000\du,10.000000\du){};
% setfont left to latex
\definecolor{dialinecolor}{rgb}{0.000000, 0.000000, 0.000000}
\pgfsetstrokecolor{dialinecolor}
\pgfsetstrokeopacity{1.000000}
\definecolor{diafillcolor}{rgb}{0.000000, 0.000000, 0.000000}
\pgfsetfillcolor{diafillcolor}
\pgfsetfillopacity{1.000000}
\node[anchor=base west,inner sep=0pt,outer sep=0pt,color=dialinecolor] at (45.013600\du,4.000000\du){};
\pgfsetlinewidth{0.100000\du}
\pgfsetdash{}{0pt}
\pgfsetbuttcap
\pgfsetmiterjoin
\pgfsetlinewidth{0.100000\du}
\pgfsetbuttcap
\pgfsetmiterjoin
\pgfsetdash{}{0pt}
\definecolor{dialinecolor}{rgb}{0.000000, 0.639216, 0.000000}
\pgfsetstrokecolor{dialinecolor}
\pgfsetstrokeopacity{1.000000}
\pgfpathmoveto{\pgfpoint{45.013600\du}{3.000000\du}}
\pgfpathcurveto{\pgfpoint{45.563600\du}{3.000000\du}}{\pgfpoint{46.013600\du}{3.450000\du}}{\pgfpoint{46.013600\du}{4.000000\du}}
\pgfpathcurveto{\pgfpoint{46.013600\du}{4.550000\du}}{\pgfpoint{45.563600\du}{5.000000\du}}{\pgfpoint{45.013600\du}{5.000000\du}}
\pgfpathcurveto{\pgfpoint{44.463600\du}{5.000000\du}}{\pgfpoint{44.013600\du}{4.550000\du}}{\pgfpoint{44.013600\du}{4.000000\du}}
\pgfpathcurveto{\pgfpoint{44.013600\du}{3.450000\du}}{\pgfpoint{44.463600\du}{3.000000\du}}{\pgfpoint{45.013600\du}{3.000000\du}}
\pgfusepath{stroke}
% setfont left to latex
\definecolor{dialinecolor}{rgb}{0.000000, 0.000000, 0.000000}
\pgfsetstrokecolor{dialinecolor}
\pgfsetstrokeopacity{1.000000}
\definecolor{diafillcolor}{rgb}{0.000000, 0.000000, 0.000000}
\pgfsetfillcolor{diafillcolor}
\pgfsetfillopacity{1.000000}
\node[anchor=base west,inner sep=0pt,outer sep=0pt,color=dialinecolor] at (44.77\du,4.35\du){$\sv$};
% setfont left to latex
\definecolor{dialinecolor}{rgb}{0.000000, 0.000000, 0.000000}
\pgfsetstrokecolor{dialinecolor}
\pgfsetstrokeopacity{1.000000}
\definecolor{diafillcolor}{rgb}{0.000000, 0.000000, 0.000000}
\pgfsetfillcolor{diafillcolor}
\pgfsetfillopacity{1.000000}
\node[anchor=base west,inner sep=0pt,outer sep=0pt,color=dialinecolor] at (49.000000\du,4.000000\du){};
% setfont left to latex
\definecolor{dialinecolor}{rgb}{0.000000, 0.000000, 0.000000}
\pgfsetstrokecolor{dialinecolor}
\pgfsetstrokeopacity{1.000000}
\definecolor{diafillcolor}{rgb}{0.000000, 0.000000, 0.000000}
\pgfsetfillcolor{diafillcolor}
\pgfsetfillopacity{1.000000}
\node[anchor=base west,inner sep=0pt,outer sep=0pt,color=dialinecolor] at (49.000000\du,4.000000\du){};
% setfont left to latex
\definecolor{dialinecolor}{rgb}{0.000000, 0.000000, 0.000000}
\pgfsetstrokecolor{dialinecolor}
\pgfsetstrokeopacity{1.000000}
\definecolor{diafillcolor}{rgb}{0.000000, 0.000000, 0.000000}
\pgfsetfillcolor{diafillcolor}
\pgfsetfillopacity{1.000000}
\node[anchor=base west,inner sep=0pt,outer sep=0pt,color=dialinecolor] at (49.000000\du,4.000000\du){};
\pgfsetlinewidth{0.100000\du}
\pgfsetdash{}{0pt}
\pgfsetbuttcap
\pgfsetmiterjoin
\pgfsetlinewidth{0.100000\du}
\pgfsetbuttcap
\pgfsetmiterjoin
\pgfsetdash{}{0pt}
\definecolor{dialinecolor}{rgb}{0.000000, 0.639216, 0.000000}
\pgfsetstrokecolor{dialinecolor}
\pgfsetstrokeopacity{1.000000}
\pgfpathmoveto{\pgfpoint{53.013631\du}{3.000000\du}}
\pgfpathcurveto{\pgfpoint{53.563631\du}{3.000000\du}}{\pgfpoint{54.013631\du}{3.450000\du}}{\pgfpoint{54.013631\du}{4.000000\du}}
\pgfpathcurveto{\pgfpoint{54.013631\du}{4.550000\du}}{\pgfpoint{53.563631\du}{5.000000\du}}{\pgfpoint{53.013631\du}{5.000000\du}}
\pgfpathcurveto{\pgfpoint{52.463631\du}{5.000000\du}}{\pgfpoint{52.013631\du}{4.550000\du}}{\pgfpoint{52.013631\du}{4.000000\du}}
\pgfpathcurveto{\pgfpoint{52.013631\du}{3.450000\du}}{\pgfpoint{52.463631\du}{3.000000\du}}{\pgfpoint{53.013631\du}{3.000000\du}}
\pgfusepath{stroke}
% setfont left to latex
\definecolor{dialinecolor}{rgb}{0.000000, 0.000000, 0.000000}
\pgfsetstrokecolor{dialinecolor}
\pgfsetstrokeopacity{1.000000}
\definecolor{diafillcolor}{rgb}{0.000000, 0.000000, 0.000000}
\pgfsetfillcolor{diafillcolor}
\pgfsetfillopacity{1.000000}
%44.76\du,4.36\du
\node[anchor=base west,inner sep=0pt,outer sep=0pt,color=dialinecolor] at (52.75\du,4.35\du){$\rvcand$};
% setfont left to latex
\definecolor{dialinecolor}{rgb}{0.000000, 0.000000, 0.000000}
\pgfsetstrokecolor{dialinecolor}
\pgfsetstrokeopacity{1.000000}
\definecolor{diafillcolor}{rgb}{0.000000, 0.000000, 0.000000}
\pgfsetfillcolor{diafillcolor}
\pgfsetfillopacity{1.000000}
\node[anchor=base west,inner sep=0pt,outer sep=0pt,color=dialinecolor] at (50.000000\du,4.000000\du){};
% setfont left to latex
\definecolor{dialinecolor}{rgb}{0.000000, 0.000000, 0.000000}
\pgfsetstrokecolor{dialinecolor}
\pgfsetstrokeopacity{1.000000}
\definecolor{diafillcolor}{rgb}{0.000000, 0.000000, 0.000000}
\pgfsetfillcolor{diafillcolor}
\pgfsetfillopacity{1.000000}
\node[anchor=base west,inner sep=0pt,outer sep=0pt,color=dialinecolor] at (53.000000\du,4.000000\du){};
% setfont left to latex
\definecolor{dialinecolor}{rgb}{0.000000, 0.000000, 0.000000}
\pgfsetstrokecolor{dialinecolor}
\pgfsetstrokeopacity{1.000000}
\definecolor{diafillcolor}{rgb}{0.000000, 0.000000, 0.000000}
\pgfsetfillcolor{diafillcolor}
\pgfsetfillopacity{1.000000}
\node[anchor=base west,inner sep=0pt,outer sep=0pt,color=dialinecolor] at (54.550000\du,10.300000\du){$\cv_m$};
% setfont left to latex
\definecolor{dialinecolor}{rgb}{0.000000, 0.000000, 0.000000}
\pgfsetstrokecolor{dialinecolor}
\pgfsetstrokeopacity{1.000000}
\definecolor{diafillcolor}{rgb}{0.000000, 0.000000, 0.000000}
\pgfsetfillcolor{diafillcolor}
\pgfsetfillopacity{1.000000}
\node[anchor=base west,inner sep=0pt,outer sep=0pt,color=dialinecolor] at (55.000000\du,10.000000\du){};
\pgfsetlinewidth{0.100000\du}
\pgfsetdash{{\pgflinewidth}{0.200000\du}}{0cm}
\pgfsetmiterjoin
\pgfsetbuttcap
{
\definecolor{diafillcolor}{rgb}{0.000000, 0.000000, 0.000000}
\pgfsetfillcolor{diafillcolor}
\pgfsetfillopacity{1.000000}
% was here!!!
\pgfsetarrowsend{latex}
\definecolor{dialinecolor}{rgb}{0.000000, 0.000000, 0.000000}
\pgfsetstrokecolor{dialinecolor}
\pgfsetstrokeopacity{1.000000}
\pgfpathmoveto{\pgfpoint{49.000000\du}{8.000000\du}}
\pgfpathcurveto{\pgfpoint{49.000000\du}{6.000000\du}}{\pgfpoint{45.013600\du}{7.150000\du}}{\pgfpoint{45.013600\du}{5.150000\du}}
\pgfusepath{stroke}
}
\pgfsetlinewidth{0.100000\du}
\pgfsetdash{}{0pt}
\pgfsetbuttcap
\pgfsetmiterjoin
\pgfsetlinewidth{0.100000\du}
\pgfsetbuttcap
\pgfsetmiterjoin
\pgfsetdash{}{0pt}
\definecolor{dialinecolor}{rgb}{0.803922, 0.000000, 0.000000}
\pgfsetstrokecolor{dialinecolor}
\pgfsetstrokeopacity{1.000000}
\pgfpathellipse{\pgfpoint{46.000000\du}{10.000000\du}}{\pgfpoint{1.000000\du}{0\du}}{\pgfpoint{0\du}{1.000000\du}}
\pgfusepath{stroke}
% setfont left to latex
\definecolor{dialinecolor}{rgb}{0.000000, 0.000000, 0.000000}
\pgfsetstrokecolor{dialinecolor}
\pgfsetstrokeopacity{1.000000}
\definecolor{diafillcolor}{rgb}{0.000000, 0.000000, 0.000000}
\pgfsetfillcolor{diafillcolor}
\pgfsetfillopacity{1.000000}
\node[anchor=base west,inner sep=0pt,outer sep=0pt,color=dialinecolor] at (45.5500000\du,10.300000\du){$\cv_2$};
% setfont left to latex
\definecolor{dialinecolor}{rgb}{0.000000, 0.000000, 0.000000}
\pgfsetstrokecolor{dialinecolor}
\pgfsetstrokeopacity{1.000000}
\definecolor{diafillcolor}{rgb}{0.000000, 0.000000, 0.000000}
\pgfsetfillcolor{diafillcolor}
\pgfsetfillopacity{1.000000}
\node[anchor=base west,inner sep=0pt,outer sep=0pt,color=dialinecolor] at (46.000000\du,10.000000\du){};
\pgfsetlinewidth{0.100000\du}
\pgfsetdash{}{0pt}
\pgfsetbuttcap
\pgfsetmiterjoin
\pgfsetlinewidth{0.100000\du}
\pgfsetbuttcap
\pgfsetmiterjoin
\pgfsetdash{}{0pt}
\definecolor{dialinecolor}{rgb}{0.803922, 0.000000, 0.000000}
\pgfsetstrokecolor{dialinecolor}
\pgfsetstrokeopacity{1.000000}
\pgfpathellipse{\pgfpoint{43.000000\du}{10.000000\du}}{\pgfpoint{1.000000\du}{0\du}}{\pgfpoint{0\du}{1.000000\du}}
\pgfusepath{stroke}
% setfont left to latex
\definecolor{dialinecolor}{rgb}{0.000000, 0.000000, 0.000000}
\pgfsetstrokecolor{dialinecolor}
\pgfsetstrokeopacity{1.000000}
\definecolor{diafillcolor}{rgb}{0.000000, 0.000000, 0.000000}
\pgfsetfillcolor{diafillcolor}
\pgfsetfillopacity{1.000000}
\node[anchor=base west,inner sep=0pt,outer sep=0pt,color=dialinecolor] at (42.5500000\du,10.300000\du){$\cv_1$};
\pgfsetlinewidth{0.100000\du}
\pgfsetdash{}{0pt}
\pgfsetbuttcap
\pgfsetmiterjoin
\pgfsetlinewidth{0.100000\du}
\pgfsetbuttcap
\pgfsetmiterjoin
\pgfsetdash{}{0pt}
\definecolor{dialinecolor}{rgb}{0.803922, 0.000000, 0.000000}
\pgfsetstrokecolor{dialinecolor}
\pgfsetstrokeopacity{1.000000}
\pgfpathellipse{\pgfpoint{55.000000\du}{10.000000\du}}{\pgfpoint{1.000000\du}{0\du}}{\pgfpoint{0\du}{1.000000\du}}
\pgfusepath{stroke}
% setfont left to latex
\definecolor{dialinecolor}{rgb}{0.000000, 0.000000, 0.000000}
\pgfsetstrokecolor{dialinecolor}
\pgfsetstrokeopacity{1.000000}
\definecolor{diafillcolor}{rgb}{0.000000, 0.000000, 0.000000}
\pgfsetfillcolor{diafillcolor}
\pgfsetfillopacity{1.000000}
\node[anchor=base west,inner sep=0pt,outer sep=0pt,color=dialinecolor] at (55.000000\du,10.000000\du){};
\pgfsetlinewidth{0.100000\du}
\pgfsetdash{{\pgflinewidth}{0.200000\du}}{0cm}
\pgfsetbuttcap
{
\definecolor{diafillcolor}{rgb}{0.000000, 0.000000, 0.000000}
\pgfsetfillcolor{diafillcolor}
\pgfsetfillopacity{1.000000}
% was here!!!
\definecolor{dialinecolor}{rgb}{0.000000, 0.000000, 0.000000}
\pgfsetstrokecolor{dialinecolor}
\pgfsetstrokeopacity{1.000000}
\draw (47.500000\du,10.000000\du)--(53.451200\du,10.000000\du);
}
% setfont left to latex
\definecolor{dialinecolor}{rgb}{0.000000, 0.000000, 0.000000}
\pgfsetstrokecolor{dialinecolor}
\pgfsetstrokeopacity{1.000000}
\definecolor{diafillcolor}{rgb}{0.000000, 0.000000, 0.000000}
\pgfsetfillcolor{diafillcolor}
\pgfsetfillopacity{1.000000}
\node[anchor=base west,inner sep=0pt,outer sep=0pt,color=dialinecolor] at (53.000000\du,4.000000\du){};
% setfont left to latex
\definecolor{dialinecolor}{rgb}{0.000000, 0.000000, 0.000000}
\pgfsetstrokecolor{dialinecolor}
\pgfsetstrokeopacity{1.000000}
\definecolor{diafillcolor}{rgb}{0.000000, 0.000000, 0.000000}
\pgfsetfillcolor{diafillcolor}
\pgfsetfillopacity{1.000000}
\node[anchor=base west,inner sep=0pt,outer sep=0pt,color=dialinecolor] at (52.842400\du,3.886210\du){};
% setfont left to latex
\definecolor{dialinecolor}{rgb}{0.000000, 0.000000, 0.000000}
\pgfsetstrokecolor{dialinecolor}
\pgfsetstrokeopacity{1.000000}
\definecolor{diafillcolor}{rgb}{0.000000, 0.000000, 0.000000}
\pgfsetfillcolor{diafillcolor}
\pgfsetfillopacity{1.000000}
\node[anchor=base west,inner sep=0pt,outer sep=0pt,color=dialinecolor] at (52.787800\du,4.036160\du){};
% setfont left to latex
\definecolor{dialinecolor}{rgb}{0.000000, 0.000000, 0.000000}
\pgfsetstrokecolor{dialinecolor}
\pgfsetstrokeopacity{1.000000}
\definecolor{diafillcolor}{rgb}{0.000000, 0.000000, 0.000000}
\pgfsetfillcolor{diafillcolor}
\pgfsetfillopacity{1.000000}
\node[anchor=base west,inner sep=0pt,outer sep=0pt,color=dialinecolor] at (52.787800\du,3.899840\du){};
\pgfsetlinewidth{0.100000\du}
\pgfsetdash{{\pgflinewidth}{0.200000\du}}{0cm}
\pgfsetmiterjoin
\pgfsetbuttcap
{
\definecolor{diafillcolor}{rgb}{0.000000, 0.000000, 0.000000}
\pgfsetfillcolor{diafillcolor}
\pgfsetfillopacity{1.000000}
% was here!!!
\pgfsetarrowsend{latex}
\definecolor{dialinecolor}{rgb}{0.000000, 0.000000, 0.000000}
\pgfsetstrokecolor{dialinecolor}
\pgfsetstrokeopacity{1.000000}
\pgfpathmoveto{\pgfpoint{49.000000\du}{8.000000\du}}
\pgfpathcurveto{\pgfpoint{49.000000\du}{6.000000\du}}{\pgfpoint{53.007600\du}{7.150000\du}}{\pgfpoint{53.007600\du}{5.150000\du}}
\pgfusepath{stroke}
}
% setfont left to latex
\definecolor{dialinecolor}{rgb}{0.000000, 0.000000, 0.000000}
\pgfsetstrokecolor{dialinecolor}
\pgfsetstrokeopacity{1.000000}
\definecolor{diafillcolor}{rgb}{0.000000, 0.000000, 0.000000}
\pgfsetfillcolor{diafillcolor}
\pgfsetfillopacity{1.000000}
\node[anchor=base west,inner sep=0pt,outer sep=0pt,color=dialinecolor] at (50.375100\du,5.440180\du){};
% setfont left to latex
\definecolor{dialinecolor}{rgb}{0.000000, 0.000000, 0.000000}
\pgfsetstrokecolor{dialinecolor}
\pgfsetstrokeopacity{1.000000}
\definecolor{diafillcolor}{rgb}{0.000000, 0.000000, 0.000000}
\pgfsetfillcolor{diafillcolor}
\pgfsetfillopacity{1.000000}
\node[anchor=base west,inner sep=0pt,outer sep=0pt,color=dialinecolor] at (50.991400\du,1.856260\du){};
% setfont left to latex
\definecolor{dialinecolor}{rgb}{0.000000, 0.000000, 0.000000}
\pgfsetstrokecolor{dialinecolor}
\pgfsetstrokeopacity{1.000000}
\definecolor{diafillcolor}{rgb}{0.000000, 0.000000, 0.000000}
\pgfsetfillcolor{diafillcolor}
\pgfsetfillopacity{1.000000}
\node[anchor=base west,inner sep=0pt,outer sep=0pt,color=dialinecolor] at (50.991400\du,1.856260\du){};
% setfont left to latex
\definecolor{dialinecolor}{rgb}{0.000000, 0.000000, 0.000000}
\pgfsetstrokecolor{dialinecolor}
\pgfsetstrokeopacity{1.000000}
\definecolor{diafillcolor}{rgb}{0.000000, 0.000000, 0.000000}
\pgfsetfillcolor{diafillcolor}
\pgfsetfillopacity{1.000000}
\node[anchor=base west,inner sep=0pt,outer sep=0pt,color=dialinecolor] at (55.933500\du,4.199240\du){};
% setfont left to latex
\definecolor{dialinecolor}{rgb}{0.000000, 0.000000, 0.000000}
\pgfsetstrokecolor{dialinecolor}
\pgfsetstrokeopacity{1.000000}
\definecolor{diafillcolor}{rgb}{0.000000, 0.000000, 0.000000}
\pgfsetfillcolor{diafillcolor}
\pgfsetfillopacity{1.000000}
\node[anchor=base west,inner sep=0pt,outer sep=0pt,color=dialinecolor] at (55.879000\du,4.349190\du){};
% setfont left to latex
\definecolor{dialinecolor}{rgb}{0.000000, 0.000000, 0.000000}
\pgfsetstrokecolor{dialinecolor}
\pgfsetstrokeopacity{1.000000}
\definecolor{diafillcolor}{rgb}{0.000000, 0.000000, 0.000000}
\pgfsetfillcolor{diafillcolor}
\pgfsetfillopacity{1.000000}
\node[anchor=base west,inner sep=0pt,outer sep=0pt,color=dialinecolor] at (55.879000\du,4.212870\du){};
% setfont left to latex
\definecolor{dialinecolor}{rgb}{0.000000, 0.000000, 0.000000}
\pgfsetstrokecolor{dialinecolor}
\pgfsetstrokeopacity{1.000000}
\definecolor{diafillcolor}{rgb}{0.000000, 0.000000, 0.000000}
\pgfsetfillcolor{diafillcolor}
\pgfsetfillopacity{1.000000}
\node[anchor=base west,inner sep=0pt,outer sep=0pt,color=dialinecolor] at (55.993100\du,4.198980\du){};
% setfont left to latex
\definecolor{dialinecolor}{rgb}{0.313726, 0.313726, 0.313726}
\pgfsetstrokecolor{dialinecolor}
\pgfsetstrokeopacity{1.000000}
\definecolor{diafillcolor}{rgb}{0.313726, 0.313726, 0.313726}
\pgfsetfillcolor{diafillcolor}
\pgfsetfillopacity{1.000000}
\node[anchor=base,inner sep=0pt, outer sep=0pt,color=dialinecolor] at (49.49\du,3.68\du){generative};
% setfont left to latex
\definecolor{dialinecolor}{rgb}{0.313726, 0.313726, 0.313726}
\pgfsetstrokecolor{dialinecolor}
\pgfsetstrokeopacity{1.000000}
\definecolor{diafillcolor}{rgb}{0.313726, 0.313726, 0.313726}
\pgfsetfillcolor{diafillcolor}
\pgfsetfillopacity{1.000000}
\node[anchor=base,inner sep=0pt, outer sep=0pt,color=dialinecolor] at (49.49\du,4.253\du){process};
\pgfsetlinewidth{0.050000\du}
\pgfsetdash{{\pgflinewidth}{0.200000\du}}{0cm}
\pgfsetmiterjoin
\pgfsetbuttcap
{
\definecolor{diafillcolor}{rgb}{0.313726, 0.313726, 0.313726}
\pgfsetfillcolor{diafillcolor}
\pgfsetfillopacity{1.000000}
% was here!!!
\pgfsetarrowsend{latex}
\definecolor{dialinecolor}{rgb}{0.313726, 0.313726, 0.313726}
\pgfsetstrokecolor{dialinecolor}
\pgfsetstrokeopacity{1.000000}
\pgfpathmoveto{\pgfpoint{49.542045\du}{4.388634\du}}
\pgfpathcurveto{\pgfpoint{50.185815\du}{6.341510\du}}{\pgfpoint{50.621598\du}{4.452949\du}}{\pgfpoint{51.257956\du}{6.425600\du}}
\pgfusepath{stroke}
}
% setfont left to latex
\definecolor{dialinecolor}{rgb}{0.000000, 0.000000, 0.000000}
\pgfsetstrokecolor{dialinecolor}
\pgfsetstrokeopacity{1.000000}
\definecolor{diafillcolor}{rgb}{0.000000, 0.000000, 0.000000}
\pgfsetfillcolor{diafillcolor}
\pgfsetfillopacity{1.000000}
\node[anchor=base west,inner sep=0pt,outer sep=0pt,color=dialinecolor] at (49.429010\du,3.713715\du){};
\end{tikzpicture}

	\end{center}
\end{figure}

A masked component failure represents the notion of a diagnostician who is able to, with varying degrees of success, isolate the failed component to some subset of the system.
We consider a particular generative model given by the following definition.
\begin{definition}
\label{def:rv_A}
An $\alpha$-masked component failure $\rvcand$ contains the failed component with probability $\alpha \cdot 100 \%$, where $\alpha$ is denoted its \emph{accuracy}.
Let $\RV{A}$ denote the random accuracy of a masked component failure with a sample space $\Set{A} \subseteq [0,1]$.
The distribution of a random masked component failure $\rvcand$ given $\RV{A} = \alpha$ is denoted a random $\alpha$-masked component failure.
\end{definition}

The accuracy of a masked component failure could theoretically convey information about the system.
As a simplification, we make the following assumption.
\begin{assumption}
\label{asm:indep_A_S_K}
The random accuracy $\RV{A}$ of masked component failures is independent of the random system lifetime $\sv$ and the random component cause $\RV{K}$ and therefore does not carry information about the true parameter index $\tparam\sysparam$.
\end{assumption}

A random masked component failure may have an uncertain cardinality.
We model this uncertainty with the random variable given by the following definition.
\begin{definition}
\label{def:rv_W}
Let $\RV{W} = \Card{\rvcand}$ denote the random cardinality of a random masked component failure with a sample space $\{1,\ldots,m-1\}$.
$\RV{W}$ may never realize $m$ since that does not isolate a proper subset of the components and it may never realize $0$ since that does not estimate the component cause of failure.
\end{definition}
Note that if $\RV{W}=m$ were to be realized, contrary to the above definition, then a consistent point estimator for the lifetime distribution $\sv$ is possible but no consistent estimator of the component lifetimes $\cv_1,\ldots,\cv_m$ is possible.
In \cref{dummy}, we provide a point estimator in such a situation that is inconsistent but minimizes the \emph{mean squared error}.

The cardinality of a masked component failure could theoretically convey information about the system.
As a simplification, we make the following assumption.
\begin{assumption}
\label{asm:indep_W_S}
The random cardinality $\RV{W}$ of masked component failures is independent of the random system lifetime $\sv$ and the random component cause $\RV{K}$ and therefore does not carry information about the true parameter index $\tparam\sysparam$.
\end{assumption}

We make the following simplifying assumption.
\begin{assumption}
\label{asm:indep_C_given_K_S}
The random masked component failure $\rvcand$ conditioned $\RV{K}$ is independent of $\sv$.
\end{assumption}
The $\alpha$-masked component model is somewhat naive since $\rvcand$ given $\RV{K}$ is conditionally independent of $\sv$.
By \cref{asm:indep_C_given_K_S}, an $\alpha$-masked component failure carries information about $\tparam\sysparam$ through its dependency on $\RV{K}$.

The indicator function and the $k$-subset functions are needed for subsequent material and are given by the following definitions.
\begin{definition}[Indicator function]
	Given a subset $\Set{A}$ of a set $\Set{B}$, the indicator function $\SetIndicator{\Set{A}} \colon \Set{B} \mapsto \{0,1\}$
	is equal to $1$ if $x \in \Set{A}$ and $0$ otherwise.
\end{definition}

\begin{definition}
	The subsets of $\Set{A}$ that are of cardinality $k$ is defined as
	\begin{equation}
	\left[\Set{A}\right]^k \coloneqq \SetBuilder{\Set{X} \in \Set{A}}{\Card{\Set{A}} = k}\,,
	\end{equation}
	which has a cardinality
	\begin{equation}
	\binom{\Card{\Set{A}}}{k}\,.
	\end{equation}
\end{definition}

The distribution of candidate sets conditioned on a specific cardinality is given by the following definition.
\begin{definition}
	The random candidate set $\rvcand$ conditioned on $\RV{W} = w$ has a sample space given by
	\begin{equation}
	\candsw{w} = \left[\PS{\RV{K}}\right]^w\,.
	\end{equation}
\end{definition}
By \cref{asm:indep_W_S,asm:indep_A_S_K}, the distributions of $\RV{W}$ and $\RV{A}$ is not dependent on the true parameter index $\tparam\sysparam$.
Any inferences about the parametric model will be \emph{independent} of $\RV{W}$ and $\RV{A}$ except in the sense that conditioning on larger $\alpha$-masked component failures or less accurate accuracy $\alpha$ generates estimators with greater variance.
This, of course, is quite important, but the fact that a particular $\RV{W}$ or $\RV{A}$ is realized says nothing by itself about $\tparam\sysparam$.

%The \emph{probability space} of the model conditioned on $\RV{W} = w$ is given by the triple
%\begin{equation}
%	\left((0,\infty) \times \candsw{w}, \Set{B} \times \PS{\candsw{w}}, \Fun{P}\right)
%\end{equation}
%where $\Set{B}$ is the \emph{Borel space}, the set of half-open intervals, and $\Fun{P} \colon \Set{B} \times \PS{\candsw{w}} \mapsto [0,1]$ is the %probability measure.

\begin{figure}
	\label{fig:gen_model_alpha}
	\caption{Graphical model of random variables where $\sv = \min\{\cv_1,\cv_2\ldots,\cv_m\}$, $\rvcand$ is a random $\alpha$-masked candidate set of cardinality $w$, $\RV{K}$ is the random component failure, $\RV{W}$ is the cardinality of random candidate sets, and $\RV{A}$ is the random $\alpha$.}
	\begin{center}
		% Graphic for TeX using PGF
% Title: /home/spinoza/papers/series_system_estimation_from_masked_system_failures/research/graph_model_simplified.dia
% Creator: Dia v0.97+git
% CreationDate: Fri Feb 21 13:48:45 2020
% For: spinoza
% \usepackage{tikz}
% The following commands are not supported in PSTricks at present
% We define them conditionally, so when they are implemented,
% this pgf file will use them.
\ifx\du\undefined
  \newlength{\du}
\fi
\setlength{\du}{15\unitlength}
\begin{tikzpicture}[even odd rule]
\pgftransformxscale{1.000000}
\pgftransformyscale{-1.000000}
\definecolor{dialinecolor}{rgb}{0.000000, 0.000000, 0.000000}
\pgfsetstrokecolor{dialinecolor}
\pgfsetstrokeopacity{1.000000}
\definecolor{diafillcolor}{rgb}{1.000000, 1.000000, 1.000000}
\pgfsetfillcolor{diafillcolor}
\pgfsetfillopacity{1.000000}
\pgfsetlinewidth{0.100000\du}
\pgfsetdash{{\pgflinewidth}{0.200000\du}}{0cm}
\pgfsetmiterjoin
\pgfsetbuttcap
\definecolor{dialinecolor}{rgb}{0.498039, 0.498039, 0.498039}
\pgfsetstrokecolor{dialinecolor}
\pgfsetstrokeopacity{1.000000}
\pgfpathmoveto{\pgfpoint{41.000000\du}{10.000000\du}}
\pgfpathcurveto{\pgfpoint{41.000000\du}{10.000000\du}}{\pgfpoint{41.000000\du}{8.000000\du}}{\pgfpoint{44.000000\du}{8.000000\du}}
\pgfpathcurveto{\pgfpoint{47.000000\du}{8.000000\du}}{\pgfpoint{48.000000\du}{8.000000\du}}{\pgfpoint{49.000000\du}{8.000000\du}}
\pgfpathcurveto{\pgfpoint{50.000000\du}{8.000000\du}}{\pgfpoint{51.000000\du}{8.000000\du}}{\pgfpoint{54.000000\du}{8.000000\du}}
\pgfpathcurveto{\pgfpoint{57.000000\du}{8.000000\du}}{\pgfpoint{57.000000\du}{10.000000\du}}{\pgfpoint{57.000000\du}{10.000000\du}}
\pgfpathcurveto{\pgfpoint{57.000000\du}{10.000000\du}}{\pgfpoint{57.000000\du}{12.000000\du}}{\pgfpoint{54.000000\du}{12.000000\du}}
\pgfpathcurveto{\pgfpoint{51.000000\du}{12.000000\du}}{\pgfpoint{50.000000\du}{12.000000\du}}{\pgfpoint{49.000000\du}{12.000000\du}}
\pgfpathcurveto{\pgfpoint{48.000000\du}{12.000000\du}}{\pgfpoint{47.000000\du}{12.000000\du}}{\pgfpoint{44.000000\du}{12.000000\du}}
\pgfpathcurveto{\pgfpoint{41.000000\du}{12.000000\du}}{\pgfpoint{41.000000\du}{10.000000\du}}{\pgfpoint{41.000000\du}{10.000000\du}}
\pgfpathclose
\pgfusepath{stroke}
% setfont left to latex
\definecolor{dialinecolor}{rgb}{0.000000, 0.000000, 0.000000}
\pgfsetstrokecolor{dialinecolor}
\pgfsetstrokeopacity{1.000000}
\definecolor{diafillcolor}{rgb}{0.000000, 0.000000, 0.000000}
\pgfsetfillcolor{diafillcolor}
\pgfsetfillopacity{1.000000}
\node[anchor=base west,inner sep=0pt,outer sep=0pt,color=dialinecolor] at (43.000000\du,10.000000\du){};
% setfont left to latex
\definecolor{dialinecolor}{rgb}{0.000000, 0.000000, 0.000000}
\pgfsetstrokecolor{dialinecolor}
\pgfsetstrokeopacity{1.000000}
\definecolor{diafillcolor}{rgb}{0.000000, 0.000000, 0.000000}
\pgfsetfillcolor{diafillcolor}
\pgfsetfillopacity{1.000000}
\node[anchor=base west,inner sep=0pt,outer sep=0pt,color=dialinecolor] at (43.000000\du,10.000000\du){};
% setfont left to latex
\definecolor{dialinecolor}{rgb}{0.000000, 0.000000, 0.000000}
\pgfsetstrokecolor{dialinecolor}
\pgfsetstrokeopacity{1.000000}
\definecolor{diafillcolor}{rgb}{0.000000, 0.000000, 0.000000}
\pgfsetfillcolor{diafillcolor}
\pgfsetfillopacity{1.000000}
\node[anchor=base west,inner sep=0pt,outer sep=0pt,color=dialinecolor] at (44.938600\du,4.724999\du){};
\pgfsetlinewidth{0.100000\du}
\pgfsetdash{}{0pt}
\pgfsetbuttcap
\pgfsetmiterjoin
\pgfsetlinewidth{0.100000\du}
\pgfsetbuttcap
\pgfsetmiterjoin
\pgfsetdash{}{0pt}
\definecolor{dialinecolor}{rgb}{0.000000, 0.639216, 0.000000}
\pgfsetstrokecolor{dialinecolor}
\pgfsetstrokeopacity{1.000000}
\pgfpathellipse{\pgfpoint{43.983709\du}{4.524260\du}}{\pgfpoint{0.897140\du}{0\du}}{\pgfpoint{0\du}{0.897140\du}}
\pgfusepath{stroke}
% setfont left to latex
\definecolor{dialinecolor}{rgb}{0.000000, 0.000000, 0.000000}
\pgfsetstrokecolor{dialinecolor}
\pgfsetstrokeopacity{1.000000}
\definecolor{diafillcolor}{rgb}{0.000000, 0.000000, 0.000000}
\pgfsetfillcolor{diafillcolor}
\pgfsetfillopacity{1.000000}
\node[anchor=base west,inner sep=0pt,outer sep=0pt,color=dialinecolor] at (43.725219\du,4.849559\du){S};
% setfont left to latex
\definecolor{dialinecolor}{rgb}{0.000000, 0.000000, 0.000000}
\pgfsetstrokecolor{dialinecolor}
\pgfsetstrokeopacity{1.000000}
\definecolor{diafillcolor}{rgb}{0.000000, 0.000000, 0.000000}
\pgfsetfillcolor{diafillcolor}
\pgfsetfillopacity{1.000000}
\node[anchor=base west,inner sep=0pt,outer sep=0pt,color=dialinecolor] at (48.925000\du,4.724999\du){};
% setfont left to latex
\definecolor{dialinecolor}{rgb}{0.000000, 0.000000, 0.000000}
\pgfsetstrokecolor{dialinecolor}
\pgfsetstrokeopacity{1.000000}
\definecolor{diafillcolor}{rgb}{0.000000, 0.000000, 0.000000}
\pgfsetfillcolor{diafillcolor}
\pgfsetfillopacity{1.000000}
\node[anchor=base west,inner sep=0pt,outer sep=0pt,color=dialinecolor] at (48.925000\du,4.724999\du){};
% setfont left to latex
\definecolor{dialinecolor}{rgb}{0.000000, 0.000000, 0.000000}
\pgfsetstrokecolor{dialinecolor}
\pgfsetstrokeopacity{1.000000}
\definecolor{diafillcolor}{rgb}{0.000000, 0.000000, 0.000000}
\pgfsetfillcolor{diafillcolor}
\pgfsetfillopacity{1.000000}
\node[anchor=base west,inner sep=0pt,outer sep=0pt,color=dialinecolor] at (48.925000\du,4.724999\du){};
\pgfsetlinewidth{0.100000\du}
\pgfsetdash{}{0pt}
\pgfsetbuttcap
\pgfsetmiterjoin
\pgfsetlinewidth{0.100000\du}
\pgfsetbuttcap
\pgfsetmiterjoin
\pgfsetdash{}{0pt}
\definecolor{dialinecolor}{rgb}{0.000000, 0.639216, 0.000000}
\pgfsetstrokecolor{dialinecolor}
\pgfsetstrokeopacity{1.000000}
\pgfpathellipse{\pgfpoint{49.009114\du}{5.526263\du}}{\pgfpoint{0.903074\du}{0\du}}{\pgfpoint{0\du}{0.903074\du}}
\pgfusepath{stroke}
% setfont left to latex
\definecolor{dialinecolor}{rgb}{0.000000, 0.000000, 0.000000}
\pgfsetstrokecolor{dialinecolor}
\pgfsetstrokeopacity{1.000000}
\definecolor{diafillcolor}{rgb}{0.000000, 0.000000, 0.000000}
\pgfsetfillcolor{diafillcolor}
\pgfsetfillopacity{1.000000}
\node[anchor=base west,inner sep=0pt,outer sep=0pt,color=dialinecolor] at (48.587431\du,5.875700\du){C};
% setfont left to latex
\definecolor{dialinecolor}{rgb}{0.000000, 0.000000, 0.000000}
\pgfsetstrokecolor{dialinecolor}
\pgfsetstrokeopacity{1.000000}
\definecolor{diafillcolor}{rgb}{0.000000, 0.000000, 0.000000}
\pgfsetfillcolor{diafillcolor}
\pgfsetfillopacity{1.000000}
\node[anchor=base west,inner sep=0pt,outer sep=0pt,color=dialinecolor] at (49.925000\du,4.724999\du){};
% setfont left to latex
\definecolor{dialinecolor}{rgb}{0.000000, 0.000000, 0.000000}
\pgfsetstrokecolor{dialinecolor}
\pgfsetstrokeopacity{1.000000}
\definecolor{diafillcolor}{rgb}{0.000000, 0.000000, 0.000000}
\pgfsetfillcolor{diafillcolor}
\pgfsetfillopacity{1.000000}
\node[anchor=base west,inner sep=0pt,outer sep=0pt,color=dialinecolor] at (54.017398\du,4.529175\du){};
% setfont left to latex
\definecolor{dialinecolor}{rgb}{0.000000, 0.000000, 0.000000}
\pgfsetstrokecolor{dialinecolor}
\pgfsetstrokeopacity{1.000000}
\definecolor{diafillcolor}{rgb}{0.000000, 0.000000, 0.000000}
\pgfsetfillcolor{diafillcolor}
\pgfsetfillopacity{1.000000}
\node[anchor=base west,inner sep=0pt,outer sep=0pt,color=dialinecolor] at (54.400000\du,10.400000\du){Tm};
% setfont left to latex
\definecolor{dialinecolor}{rgb}{0.000000, 0.000000, 0.000000}
\pgfsetstrokecolor{dialinecolor}
\pgfsetstrokeopacity{1.000000}
\definecolor{diafillcolor}{rgb}{0.000000, 0.000000, 0.000000}
\pgfsetfillcolor{diafillcolor}
\pgfsetfillopacity{1.000000}
\node[anchor=base west,inner sep=0pt,outer sep=0pt,color=dialinecolor] at (55.000000\du,10.000000\du){};
\pgfsetlinewidth{0.100000\du}
\pgfsetdash{{\pgflinewidth}{0.200000\du}}{0cm}
\pgfsetmiterjoin
\pgfsetbuttcap
{
\definecolor{diafillcolor}{rgb}{0.000000, 0.000000, 0.000000}
\pgfsetfillcolor{diafillcolor}
\pgfsetfillopacity{1.000000}
% was here!!!
\pgfsetarrowsend{latex}
\definecolor{dialinecolor}{rgb}{0.000000, 0.000000, 0.000000}
\pgfsetstrokecolor{dialinecolor}
\pgfsetstrokeopacity{1.000000}
\pgfpathmoveto{\pgfpoint{49.000000\du}{7.950147\du}}
\pgfpathcurveto{\pgfpoint{49.000000\du}{5.950147\du}}{\pgfpoint{44.000467\du}{8.181144\du}}{\pgfpoint{43.984620\du}{5.571397\du}}
\pgfusepath{stroke}
}
\pgfsetlinewidth{0.100000\du}
\pgfsetdash{}{0pt}
\pgfsetbuttcap
\pgfsetmiterjoin
\pgfsetlinewidth{0.100000\du}
\pgfsetbuttcap
\pgfsetmiterjoin
\pgfsetdash{}{0pt}
\definecolor{dialinecolor}{rgb}{0.803922, 0.000000, 0.000000}
\pgfsetstrokecolor{dialinecolor}
\pgfsetstrokeopacity{1.000000}
\pgfpathellipse{\pgfpoint{46.000000\du}{10.000000\du}}{\pgfpoint{1.000000\du}{0\du}}{\pgfpoint{0\du}{1.000000\du}}
\pgfusepath{stroke}
% setfont left to latex
\definecolor{dialinecolor}{rgb}{0.000000, 0.000000, 0.000000}
\pgfsetstrokecolor{dialinecolor}
\pgfsetstrokeopacity{1.000000}
\definecolor{diafillcolor}{rgb}{0.000000, 0.000000, 0.000000}
\pgfsetfillcolor{diafillcolor}
\pgfsetfillopacity{1.000000}
\node[anchor=base west,inner sep=0pt,outer sep=0pt,color=dialinecolor] at (45.400000\du,10.400000\du){T2};
% setfont left to latex
\definecolor{dialinecolor}{rgb}{0.000000, 0.000000, 0.000000}
\pgfsetstrokecolor{dialinecolor}
\pgfsetstrokeopacity{1.000000}
\definecolor{diafillcolor}{rgb}{0.000000, 0.000000, 0.000000}
\pgfsetfillcolor{diafillcolor}
\pgfsetfillopacity{1.000000}
\node[anchor=base west,inner sep=0pt,outer sep=0pt,color=dialinecolor] at (46.000000\du,10.000000\du){};
\pgfsetlinewidth{0.100000\du}
\pgfsetdash{}{0pt}
\pgfsetbuttcap
\pgfsetmiterjoin
\pgfsetlinewidth{0.100000\du}
\pgfsetbuttcap
\pgfsetmiterjoin
\pgfsetdash{}{0pt}
\definecolor{dialinecolor}{rgb}{0.803922, 0.000000, 0.000000}
\pgfsetstrokecolor{dialinecolor}
\pgfsetstrokeopacity{1.000000}
\pgfpathellipse{\pgfpoint{43.000000\du}{10.000000\du}}{\pgfpoint{1.000000\du}{0\du}}{\pgfpoint{0\du}{1.000000\du}}
\pgfusepath{stroke}
% setfont left to latex
\definecolor{dialinecolor}{rgb}{0.000000, 0.000000, 0.000000}
\pgfsetstrokecolor{dialinecolor}
\pgfsetstrokeopacity{1.000000}
\definecolor{diafillcolor}{rgb}{0.000000, 0.000000, 0.000000}
\pgfsetfillcolor{diafillcolor}
\pgfsetfillopacity{1.000000}
\node[anchor=base west,inner sep=0pt,outer sep=0pt,color=dialinecolor] at (42.400000\du,10.400000\du){T1};
\pgfsetlinewidth{0.100000\du}
\pgfsetdash{}{0pt}
\pgfsetbuttcap
\pgfsetmiterjoin
\pgfsetlinewidth{0.100000\du}
\pgfsetbuttcap
\pgfsetmiterjoin
\pgfsetdash{}{0pt}
\definecolor{dialinecolor}{rgb}{0.803922, 0.000000, 0.000000}
\pgfsetstrokecolor{dialinecolor}
\pgfsetstrokeopacity{1.000000}
\pgfpathellipse{\pgfpoint{55.000000\du}{10.000000\du}}{\pgfpoint{1.000000\du}{0\du}}{\pgfpoint{0\du}{1.000000\du}}
\pgfusepath{stroke}
% setfont left to latex
\definecolor{dialinecolor}{rgb}{0.000000, 0.000000, 0.000000}
\pgfsetstrokecolor{dialinecolor}
\pgfsetstrokeopacity{1.000000}
\definecolor{diafillcolor}{rgb}{0.000000, 0.000000, 0.000000}
\pgfsetfillcolor{diafillcolor}
\pgfsetfillopacity{1.000000}
\node[anchor=base west,inner sep=0pt,outer sep=0pt,color=dialinecolor] at (55.000000\du,10.000000\du){};
\pgfsetlinewidth{0.100000\du}
\pgfsetdash{{\pgflinewidth}{0.200000\du}}{0cm}
\pgfsetbuttcap
{
\definecolor{diafillcolor}{rgb}{0.000000, 0.000000, 0.000000}
\pgfsetfillcolor{diafillcolor}
\pgfsetfillopacity{1.000000}
% was here!!!
\definecolor{dialinecolor}{rgb}{0.000000, 0.000000, 0.000000}
\pgfsetstrokecolor{dialinecolor}
\pgfsetstrokeopacity{1.000000}
\draw (47.500000\du,10.000000\du)--(53.451200\du,10.000000\du);
}
% setfont left to latex
\definecolor{dialinecolor}{rgb}{0.000000, 0.000000, 0.000000}
\pgfsetstrokecolor{dialinecolor}
\pgfsetstrokeopacity{1.000000}
\definecolor{diafillcolor}{rgb}{0.000000, 0.000000, 0.000000}
\pgfsetfillcolor{diafillcolor}
\pgfsetfillopacity{1.000000}
\node[anchor=base west,inner sep=0pt,outer sep=0pt,color=dialinecolor] at (54.017398\du,4.529175\du){};
% setfont left to latex
\definecolor{dialinecolor}{rgb}{0.000000, 0.000000, 0.000000}
\pgfsetstrokecolor{dialinecolor}
\pgfsetstrokeopacity{1.000000}
\definecolor{diafillcolor}{rgb}{0.000000, 0.000000, 0.000000}
\pgfsetfillcolor{diafillcolor}
\pgfsetfillopacity{1.000000}
\node[anchor=base west,inner sep=0pt,outer sep=0pt,color=dialinecolor] at (52.767400\du,4.611209\du){};
\pgfsetlinewidth{0.100000\du}
\pgfsetdash{}{0pt}
\pgfsetbuttcap
\pgfsetmiterjoin
\pgfsetlinewidth{0.100000\du}
\pgfsetbuttcap
\pgfsetmiterjoin
\pgfsetdash{}{0pt}
\definecolor{dialinecolor}{rgb}{0.803922, 0.000000, 0.000000}
\pgfsetstrokecolor{dialinecolor}
\pgfsetstrokeopacity{1.000000}
\pgfpathellipse{\pgfpoint{54.017398\du}{4.529175\du}}{\pgfpoint{0.901439\du}{0\du}}{\pgfpoint{0\du}{0.901439\du}}
\pgfusepath{stroke}
% setfont left to latex
\definecolor{dialinecolor}{rgb}{0.000000, 0.000000, 0.000000}
\pgfsetstrokecolor{dialinecolor}
\pgfsetstrokeopacity{1.000000}
\definecolor{diafillcolor}{rgb}{0.000000, 0.000000, 0.000000}
\pgfsetfillcolor{diafillcolor}
\pgfsetfillopacity{1.000000}
\node[anchor=base west,inner sep=0pt,outer sep=0pt,color=dialinecolor] at (53.558866\du,4.913235\du){K};
% setfont left to latex
\definecolor{dialinecolor}{rgb}{0.000000, 0.000000, 0.000000}
\pgfsetstrokecolor{dialinecolor}
\pgfsetstrokeopacity{1.000000}
\definecolor{diafillcolor}{rgb}{0.000000, 0.000000, 0.000000}
\pgfsetfillcolor{diafillcolor}
\pgfsetfillopacity{1.000000}
\node[anchor=base west,inner sep=0pt,outer sep=0pt,color=dialinecolor] at (52.712800\du,4.761159\du){};
% setfont left to latex
\definecolor{dialinecolor}{rgb}{0.000000, 0.000000, 0.000000}
\pgfsetstrokecolor{dialinecolor}
\pgfsetstrokeopacity{1.000000}
\definecolor{diafillcolor}{rgb}{0.000000, 0.000000, 0.000000}
\pgfsetfillcolor{diafillcolor}
\pgfsetfillopacity{1.000000}
\node[anchor=base west,inner sep=0pt,outer sep=0pt,color=dialinecolor] at (52.712800\du,4.624839\du){};
\pgfsetlinewidth{0.100000\du}
\pgfsetdash{{\pgflinewidth}{0.200000\du}}{0cm}
\pgfsetmiterjoin
\pgfsetbuttcap
{
\definecolor{diafillcolor}{rgb}{0.000000, 0.000000, 0.000000}
\pgfsetfillcolor{diafillcolor}
\pgfsetfillopacity{1.000000}
% was here!!!
\pgfsetarrowsend{latex}
\definecolor{dialinecolor}{rgb}{0.000000, 0.000000, 0.000000}
\pgfsetstrokecolor{dialinecolor}
\pgfsetstrokeopacity{1.000000}
\pgfpathmoveto{\pgfpoint{49.000000\du}{7.950116\du}}
\pgfpathcurveto{\pgfpoint{49.000000\du}{5.950116\du}}{\pgfpoint{54.005641\du}{8.126699\du}}{\pgfpoint{54.016744\du}{5.580612\du}}
\pgfusepath{stroke}
}
% setfont left to latex
\definecolor{dialinecolor}{rgb}{0.000000, 0.000000, 0.000000}
\pgfsetstrokecolor{dialinecolor}
\pgfsetstrokeopacity{1.000000}
\definecolor{diafillcolor}{rgb}{0.000000, 0.000000, 0.000000}
\pgfsetfillcolor{diafillcolor}
\pgfsetfillopacity{1.000000}
\node[anchor=base west,inner sep=0pt,outer sep=0pt,color=dialinecolor] at (50.300100\du,6.165179\du){};
\pgfsetlinewidth{0.100000\du}
\pgfsetdash{{\pgflinewidth}{0.200000\du}}{0cm}
\pgfsetmiterjoin
\pgfsetbuttcap
{
\definecolor{diafillcolor}{rgb}{0.000000, 0.000000, 0.000000}
\pgfsetfillcolor{diafillcolor}
\pgfsetfillopacity{1.000000}
% was here!!!
\pgfsetarrowsend{latex}
\definecolor{dialinecolor}{rgb}{0.000000, 0.000000, 0.000000}
\pgfsetstrokecolor{dialinecolor}
\pgfsetstrokeopacity{1.000000}
\pgfpathmoveto{\pgfpoint{53.066002\du}{4.539794\du}}
\pgfpathcurveto{\pgfpoint{50.597546\du}{4.567345\du}}{\pgfpoint{52.278410\du}{5.513140\du}}{\pgfpoint{49.912188\du}{5.526263\du}}
\pgfusepath{stroke}
}
% setfont left to latex
\definecolor{dialinecolor}{rgb}{0.000000, 0.000000, 0.000000}
\pgfsetstrokecolor{dialinecolor}
\pgfsetstrokeopacity{1.000000}
\definecolor{diafillcolor}{rgb}{0.000000, 0.000000, 0.000000}
\pgfsetfillcolor{diafillcolor}
\pgfsetfillopacity{1.000000}
\node[anchor=base west,inner sep=0pt,outer sep=0pt,color=dialinecolor] at (51.002880\du,3.107163\du){};
% setfont left to latex
\definecolor{dialinecolor}{rgb}{0.000000, 0.000000, 0.000000}
\pgfsetstrokecolor{dialinecolor}
\pgfsetstrokeopacity{1.000000}
\definecolor{diafillcolor}{rgb}{0.000000, 0.000000, 0.000000}
\pgfsetfillcolor{diafillcolor}
\pgfsetfillopacity{1.000000}
\node[anchor=base west,inner sep=0pt,outer sep=0pt,color=dialinecolor] at (51.002880\du,3.107163\du){};
% setfont left to latex
\definecolor{dialinecolor}{rgb}{0.000000, 0.000000, 0.000000}
\pgfsetstrokecolor{dialinecolor}
\pgfsetstrokeopacity{1.000000}
\definecolor{diafillcolor}{rgb}{0.000000, 0.000000, 0.000000}
\pgfsetfillcolor{diafillcolor}
\pgfsetfillopacity{1.000000}
\node[anchor=base west,inner sep=0pt,outer sep=0pt,color=dialinecolor] at (55.858500\du,4.924239\du){};
\pgfsetlinewidth{0.100000\du}
\pgfsetdash{}{0pt}
\pgfsetbuttcap
\pgfsetmiterjoin
\pgfsetlinewidth{0.100000\du}
\pgfsetbuttcap
\pgfsetmiterjoin
\pgfsetdash{}{0pt}
\definecolor{dialinecolor}{rgb}{0.000000, 0.639216, 0.000000}
\pgfsetstrokecolor{dialinecolor}
\pgfsetstrokeopacity{1.000000}
\pgfpathellipse{\pgfpoint{51.002880\du}{3.107163\du}}{\pgfpoint{0.900893\du}{0\du}}{\pgfpoint{0\du}{0.900893\du}}
\pgfusepath{stroke}
% setfont left to latex
\definecolor{dialinecolor}{rgb}{0.000000, 0.000000, 0.000000}
\pgfsetstrokecolor{dialinecolor}
\pgfsetstrokeopacity{1.000000}
\definecolor{diafillcolor}{rgb}{0.000000, 0.000000, 0.000000}
\pgfsetfillcolor{diafillcolor}
\pgfsetfillopacity{1.000000}
\node[anchor=base west,inner sep=0pt,outer sep=0pt,color=dialinecolor] at (50.459842\du,3.508592\du){W};
% setfont left to latex
\definecolor{dialinecolor}{rgb}{0.000000, 0.000000, 0.000000}
\pgfsetstrokecolor{dialinecolor}
\pgfsetstrokeopacity{1.000000}
\definecolor{diafillcolor}{rgb}{0.000000, 0.000000, 0.000000}
\pgfsetfillcolor{diafillcolor}
\pgfsetfillopacity{1.000000}
\node[anchor=base west,inner sep=0pt,outer sep=0pt,color=dialinecolor] at (55.804000\du,5.074189\du){};
% setfont left to latex
\definecolor{dialinecolor}{rgb}{0.000000, 0.000000, 0.000000}
\pgfsetstrokecolor{dialinecolor}
\pgfsetstrokeopacity{1.000000}
\definecolor{diafillcolor}{rgb}{0.000000, 0.000000, 0.000000}
\pgfsetfillcolor{diafillcolor}
\pgfsetfillopacity{1.000000}
\node[anchor=base west,inner sep=0pt,outer sep=0pt,color=dialinecolor] at (55.804000\du,4.937869\du){};
% setfont left to latex
\definecolor{dialinecolor}{rgb}{0.000000, 0.000000, 0.000000}
\pgfsetstrokecolor{dialinecolor}
\pgfsetstrokeopacity{1.000000}
\definecolor{diafillcolor}{rgb}{0.000000, 0.000000, 0.000000}
\pgfsetfillcolor{diafillcolor}
\pgfsetfillopacity{1.000000}
\node[anchor=base west,inner sep=0pt,outer sep=0pt,color=dialinecolor] at (55.918100\du,4.923979\du){};
\pgfsetlinewidth{0.100000\du}
\pgfsetdash{{\pgflinewidth}{0.200000\du}}{0cm}
\pgfsetmiterjoin
\pgfsetbuttcap
{
\definecolor{diafillcolor}{rgb}{0.000000, 0.000000, 0.000000}
\pgfsetfillcolor{diafillcolor}
\pgfsetfillopacity{1.000000}
% was here!!!
\pgfsetarrowsend{latex}
\definecolor{dialinecolor}{rgb}{0.000000, 0.000000, 0.000000}
\pgfsetstrokecolor{dialinecolor}
\pgfsetstrokeopacity{1.000000}
\pgfpathmoveto{\pgfpoint{50.101987\du}{3.107163\du}}
\pgfpathcurveto{\pgfpoint{48.809384\du}{3.114000\du}}{\pgfpoint{49.052540\du}{3.713785\du}}{\pgfpoint{49.009114\du}{4.623189\du}}
\pgfusepath{stroke}
}
% setfont left to latex
\definecolor{dialinecolor}{rgb}{0.000000, 0.000000, 0.000000}
\pgfsetstrokecolor{dialinecolor}
\pgfsetstrokeopacity{1.000000}
\definecolor{diafillcolor}{rgb}{0.000000, 0.000000, 0.000000}
\pgfsetfillcolor{diafillcolor}
\pgfsetfillopacity{1.000000}
\node[anchor=base west,inner sep=0pt,outer sep=0pt,color=dialinecolor] at (46.608442\du,3.430142\du){A};
% setfont left to latex
\definecolor{dialinecolor}{rgb}{0.000000, 0.000000, 0.000000}
\pgfsetstrokecolor{dialinecolor}
\pgfsetstrokeopacity{1.000000}
\definecolor{diafillcolor}{rgb}{0.000000, 0.000000, 0.000000}
\pgfsetfillcolor{diafillcolor}
\pgfsetfillopacity{1.000000}
\node[anchor=base west,inner sep=0pt,outer sep=0pt,color=dialinecolor] at (46.983124\du,3.137845\du){};
% setfont left to latex
\definecolor{dialinecolor}{rgb}{0.000000, 0.000000, 0.000000}
\pgfsetstrokecolor{dialinecolor}
\pgfsetstrokeopacity{1.000000}
\definecolor{diafillcolor}{rgb}{0.000000, 0.000000, 0.000000}
\pgfsetfillcolor{diafillcolor}
\pgfsetfillopacity{1.000000}
\node[anchor=base west,inner sep=0pt,outer sep=0pt,color=dialinecolor] at (46.983124\du,3.137845\du){};
\pgfsetlinewidth{0.100000\du}
\pgfsetdash{}{0pt}
\pgfsetbuttcap
\pgfsetmiterjoin
\pgfsetlinewidth{0.100000\du}
\pgfsetbuttcap
\pgfsetmiterjoin
\pgfsetdash{}{0pt}
\definecolor{dialinecolor}{rgb}{0.000000, 0.639216, 0.000000}
\pgfsetstrokecolor{dialinecolor}
\pgfsetstrokeopacity{1.000000}
\pgfpathellipse{\pgfpoint{46.983124\du}{3.137845\du}}{\pgfpoint{0.899010\du}{0\du}}{\pgfpoint{0\du}{0.899010\du}}
\pgfusepath{stroke}
\pgfsetlinewidth{0.100000\du}
\pgfsetdash{{\pgflinewidth}{0.200000\du}}{0cm}
\pgfsetmiterjoin
\pgfsetbuttcap
{
\definecolor{diafillcolor}{rgb}{0.000000, 0.000000, 0.000000}
\pgfsetfillcolor{diafillcolor}
\pgfsetfillopacity{1.000000}
% was here!!!
\pgfsetarrowsend{latex}
\definecolor{dialinecolor}{rgb}{0.000000, 0.000000, 0.000000}
\pgfsetstrokecolor{dialinecolor}
\pgfsetstrokeopacity{1.000000}
\pgfpathmoveto{\pgfpoint{47.037290\du}{4.085412\du}}
\pgfpathcurveto{\pgfpoint{47.080406\du}{4.839666\du}}{\pgfpoint{46.766873\du}{5.367246\du}}{\pgfpoint{48.106040\du}{5.526263\du}}
\pgfusepath{stroke}
}
\pgfsetlinewidth{0.100000\du}
\pgfsetdash{{\pgflinewidth}{0.200000\du}}{0cm}
\pgfsetmiterjoin
\pgfsetbuttcap
{
\definecolor{diafillcolor}{rgb}{0.000000, 0.000000, 0.000000}
\pgfsetfillcolor{diafillcolor}
\pgfsetfillopacity{1.000000}
% was here!!!
\pgfsetarrowsstart{latex}
\pgfsetarrowsend{latex}
\definecolor{dialinecolor}{rgb}{0.000000, 0.000000, 0.000000}
\pgfsetstrokecolor{dialinecolor}
\pgfsetstrokeopacity{1.000000}
\pgfpathmoveto{\pgfpoint{50.193417\du}{2.608947\du}}
\pgfpathcurveto{\pgfpoint{49.627359\du}{2.260545\du}}{\pgfpoint{48.338815\du}{2.258910\du}}{\pgfpoint{47.776967\du}{2.623173\du}}
\pgfusepath{stroke}
}
\end{tikzpicture}

	\end{center}
\end{figure}
In general, a sample of masked system failures may have variable sized candidate sets. In \cref{dummyref}, we derive the sampling distribution of the maximum likelihood estimator of this kind of sample using the inverse-variance weighted mean.

The \emph{primary} data point is the \emph{masked system failure} as given by the following definition.
\begin{definition}
An $\alpha$-\emph{masked system failure} consists of a system failure time $\ti$, a corresponding set of failed candidate components $\cand$, and an expected accuracy $\alpha$ of the $\cand$, denoted by $\Tuple{\ti,\cand,\alpha}$.
\end{definition}
\begin{assumption}
\label{asm:cand_prob}
A random $\alpha$-\emph{masked system failure} is a jointly distributed random system lifetime $\sv$, random failed candidate set $\rvcand$, and random $\alpha$-accuracy $\RV{A}$.
\end{assumption}

\begin{assumption}
\label{def:msfs}
A random sample of $n$ \emph{masked system failures}, denoted by
\begin{equation}
\mfs = \left( \Tuple{\sv_1,\rvcand_1,\RV{A}_1},\ldots,\Tuple{\sv_n,\rvcand_n,\RV{A}_n}\right)\,,
\end{equation}
consists of $n$ independent and identically jointly distributed pairs of system failure times and component sets.
\end{assumption}

\begin{assumption}
The only information we have about the system and its true parameter index $\tparam\sysparam$ is given by observing a random sample of $n$ masked component failures,
\begin{equation}
    \left(
    	\Tuple{\sv_1=\ti_1,\rvcand_1=\Set{C}[1], \RV{A}_1=\alpha_1}, \ldots, 
    	\Tuple{\sv_n=\ti_n,\rvcand_n=\Set{C}[n], \RV{A}_n=\alpha_n}
    \right)\,.
\end{equation}
\end{assumption}

Given the assumed model, the object of statistical interest is the (unknown) true parameter index $\tparam\sysparam$.
Provided an estimate of $\tparam\sysparam$, any characteristic that is a function of $\tparam\sysparam$ can be estimated as a result, thus the primary objective is to use the \emph{information} in a sample of $n$ masked system failures to estimate $\tparam\sysparam$ with some quantifiable uncertainty.
\end{document}