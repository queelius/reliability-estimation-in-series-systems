% Options for packages loaded elsewhere
\PassOptionsToPackage{unicode}{hyperref}
\PassOptionsToPackage{hyphens}{url}
%
\documentclass[
]{article}
\usepackage{lmodern}
\usepackage{amssymb,amsmath}
\usepackage{ifxetex,ifluatex}
\ifnum 0\ifxetex 1\fi\ifluatex 1\fi=0 % if pdftex
  \usepackage[T1]{fontenc}
  \usepackage[utf8]{inputenc}
  \usepackage{textcomp} % provide euro and other symbols
\else % if luatex or xetex
  \usepackage{unicode-math}
  \defaultfontfeatures{Scale=MatchLowercase}
  \defaultfontfeatures[\rmfamily]{Ligatures=TeX,Scale=1}
\fi
% Use upquote if available, for straight quotes in verbatim environments
\IfFileExists{upquote.sty}{\usepackage{upquote}}{}
\IfFileExists{microtype.sty}{% use microtype if available
  \usepackage[]{microtype}
  \UseMicrotypeSet[protrusion]{basicmath} % disable protrusion for tt fonts
}{}
\usepackage{xcolor}
\IfFileExists{xurl.sty}{\usepackage{xurl}}{} % add URL line breaks if available
\IfFileExists{bookmark.sty}{\usepackage{bookmark}}{\usepackage{hyperref}}
\hypersetup{
  pdftitle={Bootstrapping confidence intervals (BCa) of the maximum likelihood estimator of components in a series systems from masked failure data},
  pdfauthor={Alex Towell},
  hidelinks,
  pdfcreator={LaTeX via pandoc}}
\urlstyle{same} % disable monospaced font for URLs
\usepackage[margin=1in]{geometry}
\usepackage{graphicx}
\makeatletter
\def\maxwidth{\ifdim\Gin@nat@width>\linewidth\linewidth\else\Gin@nat@width\fi}
\def\maxheight{\ifdim\Gin@nat@height>\textheight\textheight\else\Gin@nat@height\fi}
\makeatother
% Scale images if necessary, so that they will not overflow the page
% margins by default, and it is still possible to overwrite the defaults
% using explicit options in \includegraphics[width, height, ...]{}
\setkeys{Gin}{width=\maxwidth,height=\maxheight,keepaspectratio}
% Set default figure placement to htbp
\makeatletter
\def\fps@figure{htbp}
\makeatother
\setlength{\emergencystretch}{3em} % prevent overfull lines
\providecommand{\tightlist}{%
  \setlength{\itemsep}{0pt}\setlength{\parskip}{0pt}}
\setcounter{secnumdepth}{5}
\hypersetup{linktoc=all}
\AtBeginDocument{\renewcommand{\refname}{References}}
\usepackage{hyperref}
\usepackage{graphicx}
\usepackage{amsthm}
\usepackage{amsmath}
\usepackage{natbib}
\usepackage{tikz}
\usepackage[]{natbib}
\bibliographystyle{apalike}

\title{Bootstrapping confidence intervals (BCa) of the maximum
likelihood estimator of components in a series systems from masked
failure data}
\author{Alex Towell}
\date{}

\begin{document}
\maketitle
\begin{abstract}
We estimate the parameters of a series system with Weibull component
lifetimes from relatively small samples consisting of right-censored
system lifetimes and masked component cause of failure. Under a set of
conditions that permit us to ignore how the component cause of failures
are masked, we assess the bias and variance of the estimator. Then, we
assess the accuracy of the boostrapped variance and calibration of the
confidence intervals of the MLE under a variety of scenarios.
\end{abstract}

{
\setcounter{tocdepth}{2}
\tableofcontents
}
\newcommand{\T}{T}
\newtheorem{definition}{Definition}
\newtheorem{theorem}{Theorem}
\newtheorem{corollary}{Corollary}
\newtheorem{condition}{Condition}
\renewcommand{\v}[1]{\boldsymbol{#1}}
\numberwithin{equation}{section}

\hypertarget{conditional-distribution-of-k_i-given-t_i-and-mathcalc_i}{%
\subsubsection{\texorpdfstring{Conditional distribution of \(K_i\) given
\(T_i\) and
\(\mathcal{C}_i\)}{Conditional distribution of K\_i given T\_i and \textbackslash mathcal\{C\}\_i}}\label{conditional-distribution-of-k_i-given-t_i-and-mathcalc_i}}

This subsection is not necessary in our likelihood model, but it derives
a useful result for making predictions about the component cause of
failure.

Suppose we have observed a candidate set and a series system failure and
we are interested in the probability that a particular component is the
cause of failure.

\begin{theorem}
Assuming Conditions \ref{cond:c_contains_k} and \ref{cond:equal_prob_failure_cause},
the conditional probability of the component cause of failure is component $j$ ($K_i = j$) given a masked component cause of failure ($\mathcal{C}_i = c_i$) and system lifetime ($T_i = t_i$)
is given by
\begin{equation}
\label{eq:cond_prob_k_given_t_and_c}
\Pr\{K_i = j|T_i=t_i,\mathcal{C}_i=c_i\} =
 \frac{h_j(t_i|\boldsymbol{\theta_j})}{\sum_{l \in c_i} h_l(t_i|\boldsymbol{\theta_l})} 1_{\{j \in c_i\}}.
\end{equation}
\end{theorem}
\begin{proof}
The conditional probability $\Pr\{K_i = j|T_i=t_i,\mathcal{C}_i=c_i\}$ may be
written as
$$
\Pr\{K_i = j|T_i=t_i,\mathcal{C}_i=c_i\} =
    \frac{\Pr{}_{\!\boldsymbol{\theta}}\{\mathcal{C}_i=c_i|K_i = j,T_i=t_i\} f_{K_i,T_i}(j,t_i;\boldsymbol{\theta})}
    {\sum_{j=1}^m \Pr{}_{\!\boldsymbol{\theta}}
        \{\mathcal{C}_i=c_i|K_i = j,T_i=t_i\} f_{K_i,T_i}(j,t_i;\boldsymbol{\theta})}.
$$
By Theorem \ref{thm:f_k_and_t},
$f_{K_i,T_i}(j,t_i;\boldsymbol{\theta}) = h_j(t_i;\boldsymbol{\theta})R_{T_i}(t_i;\boldsymbol{\theta})$.
We may make this substitution and simplify:
$$
\Pr\{K_i = j|T_i=t_i,\mathcal{C}_i=c_i\} =
    \frac{\Pr{}_{\!\boldsymbol{\theta}}\{\mathcal{C}_i=c_i|K_i = j,T_i=t_i\} h_j(t_i;\boldsymbol{\theta_j})}
         {\sum_{j'=1}^m \Pr{}_{\!\boldsymbol{\theta}}\{\mathcal{C}_i=c_i|K_i=j',T_i=t_i\} h_{j'}(t_i;\boldsymbol{\theta_{j'}})}.
$$
Assuming Conditions \ref{cond:c_contains_k} and \ref{cond:equal_prob_failure_cause}, we may rewrite the above
as
$$
\Pr\{K_i = j|T_i=t_i,\mathcal{C}_i=c_i\} =
    \frac{\Pr{}_{\!\boldsymbol{\theta}}\{\mathcal{C}_i=c_i|K_i = j,T_i=t_i\} h_j(t_i;\boldsymbol{\theta_j})}
    {\Pr{}_{\!\boldsymbol{\theta}}\{\mathcal{C}_i=c_i|K_i = j,T_i=t_i\} {\sum_{l \in c_i} h_l(t_i;\boldsymbol{\theta_l})}} =
    \frac{h_j(t_i;\boldsymbol{\theta_j})}
    {\sum_{l \in c_i} h_l(t_i;\boldsymbol{\theta_l})}.
$$
\end{proof}

Frequently, we may not have any information at all about the component
cause of failure. In this case, \(c_i = \{1,\ldots,m\}\), and we obtain
the following corollary.

\begin{corollary}
The probability that the $j$\textsuperscript{th}
component is the cause of system failure given only that we know a system failure
occured at time $t_i$ is given by
$$
\Pr\{K_i = j|T_i=t_i\} = \frac{h_j(t_i;\boldsymbol{\theta_j})}{\sum_{l=1}^m h_l(t_i;\boldsymbol{\theta_l})}.
$$
\end{corollary}

  \bibliography{refs.bib}

\end{document}
